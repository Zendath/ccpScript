% !TeX root = ..\..\main.tex
\begin{tikzpicture}
    %draw the cylinder and rotate it so that the "opening" is to the left
    \node[yshift=-2.5cm] (c) at (4.5, 0) [cylinder, shape border rotate=180, draw, minimum height=7cm, minimum width=5cm, anchor=shape center, name path=c1] {};
    
    %first step of ratcheted key exchange 
    \node[yshift=-.5cm] (1) at (c.north) {1. Initial Key Exchange};
    %draw both key derivations
    \node[yshift=-.5cm, xshift=-2.5cm] (1a1) at (1) {$\substack{\downarrow\\\scriptstyle k}$};
    \node[yshift=-.5cm, xshift= 2.5cm] (1a2) at (1) {$\substack{\downarrow\\\scriptstyle k}$};
    %connect the two keys with a dashed arrow (the anchors, e.g. 1a1.30, are "border anchors" that use an angle)
    \draw[<->, dashed] (1a1.30) -- (1a2.150);
    
    %second step
    \node[yshift=-1cm] (2) at (1) {2. Ratcheted Key Exchange};
    %draw refreshing of keys
    \node[yshift=-.5cm, xshift=-2.5cm] (2a1) at (2) {$k\circlearrowleft$};
    \node[yshift=-.5cm, xshift= 2.5cm] (2a2) at (2) {$\circlearrowleft k$};

    %third step
    \node[yshift=-1cm] (3) at (2) {3. Secure Channel};
    %first channel cylinder below the third step (anchor is "shape center", since the normal "center" uses the actual height of the 3d cylinder)
    \node[yshift=-.5cm, anchor=shape center] (3Channel1) at (3) [cylinder, shape border rotate=180, draw, minimum height=5.2cm, minimum width=.5cm] {};
    %draw refreshing keys
    \node[yshift=-1cm, xshift=-2.5cm] (3a1) at (3) {$k\circlearrowleft$};
    \node[yshift=-1cm, xshift= 2.5cm] (3a2) at (3) {$\circlearrowleft k$};
    %second channel cylinder (no anchor=shape center since we use the shape center anchor of the first cylinder for placement)
    \node[yshift=-1cm] (3Channel2) at (3Channel1) [cylinder, shape border rotate=180, draw, minimum height=5.2cm, minimum width=.5cm] {};

    %arrows from the keys in the second step to the first channel in the third step
    \draw [->] (2a1) -- (3Channel1.after top);
    \draw [->] (2a2) -- (3Channel1.before bottom);


    %ALICE and BOB
    \node (A) at (0, 0) {\textbf{Alice}};
    \node (B) at (9, 0) {\textbf{Bob}};

    %the two messages send from Alice
    \node (A_m1) at (0, -3) {$m$};
    \node (A_m2) at (0, -4) {$m$};

    %draw the arrow connecting the first message and the first channel
    %this path is used to get the intersection
    \path[name path=A_m1path] (A_m1) -- (3Channel1.top);
    %calculate the intersections (they are named A_m1_intersect-1, A_m1_intersect-2, ...) and draw the black part of the arrow
    \draw[-,name intersections={of=c1 and A_m1path, name=A_m1_intersect}] (A_m1) -- (A_m1_intersect-1);
    %draw the gray part of the arrow
    \draw[->, gray] (A_m1_intersect-1) -- (3Channel1.top);

    %draw the arrow connecting the second message and the second channel
    \path[name path=A_m2path] (A_m2) -- (3Channel2.top);
    \draw[<-,name intersections={of=c1 and A_m2path, name=A_m2_intersect}] (A_m2) -- (A_m2_intersect-1);
    \draw[-, gray] (A_m2_intersect-1) -- (3Channel2.top);

    %the messages bob receives
    \node (B_m1) at (9, -3) {$m$};
    \node (B_m2) at (9, -4) {$m$};

    %arrows connecting the first channel to bobs first message
    \path[name path=B_m1path] (B_m1) -- (3Channel1.bottom);
    %the gray and black parts of the arrow are switched
    \draw[-,,gray, name intersections={of=c1 and B_m1path, name=B_m1_intersect}] (3Channel1.bottom) -- (B_m1_intersect-1);
    \draw[->] (B_m1_intersect-1) -- (B_m1);

    %arrows connecting the second channel to bobs second message
    \path[name path=B_m2path] (B_m2) -- (3Channel2.top);
    \draw[<-, gray, name intersections={of=c1 and B_m2path, name=B_m2_intersect}] (3Channel2.bottom) -- (B_m2_intersect-1);
    \draw[-] (B_m2_intersect-1) -- (B_m2);

    %Left side FS-arrows
    \draw[->|, green] (.5, -.5) -- (.5, -1.25) node [midway,left] {FS};
    \node at (.5, -1.5) {\color{red}\Large\Lightning}; % corruption arrow
    \draw[|->|, green] (.5, -1.75) -- (.5, -3.25);
    \draw[|->, green] (.5, -3.75) -- (.5, -5);

    %Right side FS-arrows
    \draw[->|, green] (8.5, -.5) -- (8.5, -1.25);
    \draw[|->|, green] (8.5, -1.75) -- (8.5, -3.25);
    \node at (8.5, -3.5) {\color{red}\Large\Lightning}; %corruption arrow
    \draw[|->, green] (8.5, -3.75) -- (8.5, -5) node [midway, right] {PCS};

\end{tikzpicture}