\documentclass[a4paper,orivec]{llncs}

\newcommand{\alert}[1]{{\color{red}#1}}

\usepackage{fullpage}
\pagestyle{plain}
\setcounter{tocdepth}{2}

%!TEX root=main.tex

\usepackage{lmodern}
\usepackage[T1]{fontenc}
\usepackage[utf8]{inputenc}

\usepackage{amsfonts,amsmath,amssymb,mathtools}

\usepackage{graphicx}

\usepackage{array}

\usepackage{color,xcolor}
\definecolor{gray}{gray}{0.5}
\definecolor{darkblue}{rgb}{0,0,0.5}
\definecolor{darkgreen}{rgb}{0,0.5,0}
\usepackage[colorlinks=true,linkcolor=darkblue,urlcolor=darkblue,citecolor=darkgreen,pdftitle={Cryptographic Communication Protocols: Key Exchange and Channels}]{hyperref}

\usepackage[strings]{underscore}

\newcommand{\codescalefactor}{0.8}
\newcommand{\tikzscalefactor}{0.8}

\usepackage{nicodemus}
\usepackage{bpmarker}

%draw dashed boxes
\usepackage{dashbox}
\setlength{\dashlength}{4pt}
\setlength{\dashdash}{2pt}

%tikz and libraries
\usepackage{tikz}
\usetikzlibrary{arrows.meta, positioning, calc, trees, shapes, intersections, tikzmark, decorations.pathmorphing}

%some symbols (e.g. lightning bolt)
\usepackage{marvosym}

%placing correlated figures next to each other
\usepackage{subcaption}

%nicer in-line fractions (used for in line case distinction for functions with multiple possible outputs)
%https://mirror.physik.tu-berlin.de/pub/CTAN/macros/latex/contrib/xfrac/xfrac.pdf
\usepackage{xfrac}
%!TEX root=main.tex

\newlength\itemizeskip%used for aligning text in itemize

\let\oldparagraph=\paragraph
\renewcommand\paragraph[1]{\oldparagraph{#1.}}

\newcolumntype{C}[1]{>{\centering\arraybackslash\hspace{0pt}}p{#1}}

%provides \intertext analogue for lists
%\listintertext*{text} inserts text flush with left margin
%\listintertext{text} inserts text flush with left margin of the list environment one higher (if it exists)
%copied from https://tex.stackexchange.com/questions/135726/intertext-like-command-in-enumerate-environment
\makeatletter
\newcommand{\listintertext}{\@ifstar\listintertext@\listintertext@@}
\newcommand{\listintertext@}[1]{% \listintertext*{#1}
  \hspace*{-\@totalleftmargin}#1}
\newcommand{\listintertext@@}[1]{% \listintertext{#1}
  \hspace{-\leftmargin}#1}
\makeatother


\newcommand{\algbox}[2]{\fbox{\parbox{#1}{#2}}}
\newcommand{\arrbox}[2]{\parbox{#1}{\centering#2}}

\newcommand{\getsr}{\gets_\$}
\newcommand{\tor}{\to_\$}
\newcommand{\es}{\epsilon}
\newcommand{\T}{\mathtt{tru}}
\newcommand{\F}{\mathtt{fal}}
\newcommand{\secp}{\kappa}

\newcommand{\ZZ}{\mathbb{Z}}
\newcommand{\BB}{\{0,1\}}
\newcommand{\Bool}{\{\T,\F\}}
\newcommand{\NN}{\mathbb{N}}
\newcommand{\PS}{\mathcal{P}}

\newcommand{\advA}{\mathcal{A}}
\newcommand{\advB}{\mathcal{B}}
\newcommand{\advC}{\mathcal{C}}
\newcommand{\advD}{\mathcal{D}}
\newcommand{\Adv}{\mathrm{Adv}}

\newcommand{\APKE}{\mathrm{APKE}}
\renewcommand{\AE}{\mathrm{AE}}
\newcommand{\IKE}{\mathrm{IKE}}


%VARIABLES
\newcommand{\sk}{\mathit{sk}}
\newcommand{\st}{\mathit{st}}
\newcommand{\vk}{\mathit{vk}}
\newcommand{\pk}{\mathit{pk}}
\newcommand{\ek}{\mathit{ek}}
\newcommand{\dk}{\mathit{dk}}
\newcommand{\rk}{\mathit{rk}}
\newcommand{\ck}{\mathit{ck}}
\newcommand{\mk}{\mathit{mk}}
\newcommand{\mtag}{\tau}
\newcommand{\sig}{\sigma}
\newcommand{\ad}{\mathit{ad}}

%SPACES
\newcommand{\sksp}{\mathcal{SK}}
\newcommand{\pksp}{\mathcal{PK}}
\newcommand{\vksp}{\mathcal{VK}}
\newcommand{\eksp}{\mathcal{EK}}
\newcommand{\dksp}{\mathcal{DK}}
\newcommand{\stsp}{\mathcal{ST}}
\newcommand{\ksp}{\mathcal{K}}
\newcommand{\rsp}{\mathcal{R}}
\newcommand{\msp}{\mathcal{M}}
\newcommand{\csp}{\mathcal{C}}
\newcommand{\tsp}{\mathcal{T}}
\newcommand{\sigsp}{\Sigma}
\newcommand{\adsp}{\mathcal{AD}}
\newcommand{\idsp}{\mathcal{ID}}

%ORACLES
\newcommand{\ROh}{\mathrm{H}}

\newcommand{\CORR}{\mathrm{CORR}}
\newcommand{\IND}{{\mathrm{IND}}}
\newcommand{\ind}{{\mathrm{ind}}}
\newcommand{\INDD}{{\mathrm{IND}\$}}
\newcommand{\indd}{{\mathrm{ind}\$}}
\newcommand{\INDCCA}{{\mathrm{IND}\text{-}\mathrm{CCA}}}
\newcommand{\indcca}{{\mathrm{ind}\text{-}\mathrm{cca}}}
\newcommand{\INDCPA}{{\mathrm{IND}\text{-}\mathrm{CPA}}}
\newcommand{\indcpa}{{\mathrm{ind}\text{-}\mathrm{cpa}}}
\newcommand{\OWCCA}{{\mathrm{OW}\text{-}\mathrm{CCA}}}
\newcommand{\owcca}{{\mathrm{ow}\text{-}\mathrm{cca}}}
\newcommand{\OWCPA}{{\mathrm{OW}\text{-}\mathrm{CPA}}}
\newcommand{\owcpa}{{\mathrm{ow}\text{-}\mathrm{cpa}}}
\newcommand{\SUFCMA}{{\mathrm{SUF}\text{-}\mathrm{CMA}}}
\newcommand{\sufcma}{{\mathrm{suf}\text{-}\mathrm{cma}}}
\newcommand{\ANON}{\mathrm{ANON}}

\newcommand{\Ogen}{\mathrm{Gen}}
\newcommand{\Oinit}{\mathrm{Init}}
\newcommand{\Oenc}{\mathrm{Enc}}
\newcommand{\Odec}{\mathrm{Dec}}
\newcommand{\Oup}{\mathrm{Up}}
\newcommand{\OupPK}{\mathrm{UpPK}}
\newcommand{\OupSK}{\mathrm{UpSK}}

\newcommand{\Otag}{\mathrm{Tag}}
\newcommand{\Osig}{\mathrm{Sign}}
\newcommand{\Ovfy}{\mathrm{Vfy}}

\newcommand{\Osnd}{\mathrm{Snd}}
\newcommand{\Orsp}{\mathrm{Rsp}}
\newcommand{\Orcv}{\mathrm{Rcv}}
\newcommand{\Ochall}{\mathrm{Chall}}

\newcommand{\Ocorrupt}{\mathrm{Corrupt}}
\newcommand{\Oexp}{\mathrm{Expose}}
\newcommand{\Oexps}{\mathrm{ExposeS}}
\newcommand{\Oexpr}{\mathrm{ExposeR}}
\newcommand{\Oreveal}{\mathrm{Reveal}}



\iffalse
\newcommand{\full}{\!$\cdot$\,}
\newcommand{\core}{\hphantom{\!$\cdot$\,}}

\newcommand{\getscup}{\overset{\raisebox{-1pt}{\tiny$\;\cup$}}{\gets}}
\newcommand{\getsconcat}{\overset{\raisebox{-1pt}{\tiny$\;\shortparallel$}}{\gets}} % "concat assignment"

\newcommand{\bO}{\mathcal{O}}
\newcommand{\secp}{\lambda}

\fi

%DH
\newcommand{\DHgr}{\mathbb{G}}
\newcommand{\DHg}{g}
\newcommand{\DHp}{p}

\newcommand{\dlp}{\mathrm{dlp}}
\newcommand{\cdh}{\mathrm{cdh}}
\newcommand{\ddh}{\mathrm{ddh}}

%PRG and PRF
\newcommand{\PRF}{\mathrm{PRF}}
\newcommand{\PRG}{\mathrm{PRG}}

\newcommand{\PRGf}{\PRG.\mathrm{f}}
\newcommand{\PRFf}{\PRF.\mathrm{f}}

%KDF
\newcommand{\KDF}{\mathrm{KDF}}

%MAC and DIGITAL SIGNATURE
\newcommand{\MAC}{\mathrm{MAC}}
\newcommand{\SIG}{\mathrm{SIG}}

\newcommand{\MACgen}{\MAC.\mathrm{gen}}
\newcommand{\MACtag}{\MAC.\mathrm{tag}}
\newcommand{\MACvfy}{\MAC.\mathrm{vfy}}

\newcommand{\SIGgen}{\SIG.\mathrm{gen}}
\newcommand{\SIGsig}{\SIG.\mathrm{sig}}
\newcommand{\SIGvfy}{\SIG.\mathrm{vfy}}

%SYMMETRIC ENCRYPTION, PUBLIC-KEY ENCRYPTION, various KEMs, and KEY EXCHANGE
\newcommand{\SYE}{\mathrm{SE}}
\newcommand{\AEAD}{\mathrm{AEAD}}
\newcommand{\PKE}{\mathrm{PKE}}
\newcommand{\KEM}{\mathrm{KEM}}
\newcommand{\IKEM}{\mathrm{IKEM}}
\newcommand{\FKEM}{\mathrm{FKEM}}
\newcommand{\UKEM}{\mathrm{UKEM}}
\newcommand{\KUKEM}{\mathrm{KU}\text{-}\mathrm{KEM}} %not sure if this is the best way to typeset the "-" in KU-KEM 
\newcommand{\TKE}{\mathrm{TKE}}
\newcommand{\AKE}{\mathrm{AKE}}
\newcommand{\URKE}{\mathrm{URKE}}
\newcommand{\HIBE}{\mathrm{HIBE}}

\newcommand{\AEADgen}{\AEAD.\mathrm{gen}}
\newcommand{\AEADenc}{\AEAD.\mathrm{enc}}
\newcommand{\AEADdec}{\AEAD.\mathrm{dec}}

\newcommand{\SYEgen}{\SYE.\mathrm{gen}}
\newcommand{\SYEenc}{\SYE.\mathrm{enc}}
\newcommand{\SYEdec}{\SYE.\mathrm{dec}}

\newcommand{\PKEgen}{\PKE.\mathrm{gen}}
\newcommand{\PKEenc}{\PKE.\mathrm{enc}}
\newcommand{\PKEdec}{\PKE.\mathrm{dec}}

\newcommand{\KEMgen}{\KEM.\mathrm{gen}}
\newcommand{\KEMenc}{\KEM.\mathrm{enc}}
\newcommand{\KEMdec}{\KEM.\mathrm{dec}}

\newcommand{\IKEMgen}{\IKEM.\mathrm{gen}}
\newcommand{\IKEMenc}{\IKEM.\mathrm{enc}}
\newcommand{\IKEMdel}{\IKEM.\mathrm{del}}
\newcommand{\IKEMdec}{\IKEM.\mathrm{dec}}

\newcommand{\UKEMgen}{\UKEM.\mathrm{gen}}
\newcommand{\UKEMenc}{\UKEM.\mathrm{enc}}
\newcommand{\UKEMdec}{\UKEM.\mathrm{dec}}

\newcommand{\KUKEMgen}{\KUKEM.\mathrm{gen}}
\newcommand{\KUKEMenc}{\KUKEM.\mathrm{enc}}
\newcommand{\KUKEMdec}{\KUKEM.\mathrm{dec}}
\newcommand{\KUKEMup}{\KUKEM.\mathrm{up}}

\newcommand{\FKEMgen}{\FKEM.\mathrm{gen}}
\newcommand{\FKEMenc}{\FKEM.\mathrm{enc}}
\newcommand{\FKEMdec}{\FKEM.\mathrm{dec}}

\newcommand{\TKEsnd}{\TKE.\mathrm{snd}}
\newcommand{\TKErsp}{\TKE.\mathrm{rsp}}
\newcommand{\TKErcv}{\TKE.\mathrm{rcv}}

\newcommand{\AKEgen}{\AKE.\mathrm{gen}}
\newcommand{\AKEsnd}{\AKE.\mathrm{snd}}
\newcommand{\AKErsp}{\AKE.\mathrm{rsp}}
\newcommand{\AKErcv}{\AKE.\mathrm{rcv}}

\newcommand{\URKEinit}{\URKE.\mathrm{init}}
\newcommand{\URKEsnd}{\URKE.\mathrm{snd}}
\newcommand{\URKErcv}{\URKE.\mathrm{rcv}}

\newcommand{\HIBEgen}{\HIBE.\mathrm{gen}}
\newcommand{\HIBEenc}{\HIBE.\mathrm{enc}}
\newcommand{\HIBEdec}{\HIBE.\mathrm{dec}}
\newcommand{\HIBEdel}{\HIBE.\mathrm{del}}


\bibliographystyle{alpha}

\makeatletter
\renewcommand*\l@author[2]{}
\renewcommand*\l@title[2]{}
\makeatletter
\renewcommand{\contentsname}{}


\title{Cryptographic Communication Protocols:\\Key Exchange and Channels}
\author{Paul Rösler}

\institute{FAU Erlangen-Nürnberg}

\begin{document}

\maketitle
\begin{center}
    \today
\end{center}

\begingroup
\let\clearpage\relax
\tableofcontents
\endgroup


\section{Administrative Information}
\label{sec:admin}

\paragraph{Exercise}
The exercize is split into two parts.
Some tasks can be submitted online via the interactive StudOn quiz.
Solving these quizzes awards bonus points for the final exam, if it is passed (at most a 0.7 grade improvement).
Some tasks are not part of the quiz and will be discussed in the following exercise sessions.
These tasks do not count towards bonus points, but are still important to form a deeper understanding of the more complex systems.

\paragraph{Recordings}
Lecture recordings from 2024 are available at \url{https://video.cs.fau.de/by-lecture/CCP/2024s}.
The exercise is not recorded.

\section{Preliminary Remarks}
This document is not (yet) a full course script.
Instead, it is meant as an additional resource that systematizes the considered primitives, definitions, and constructions.
The author invites readers to submit comments or pull requests via the GitHub Repository \url{https://github.com/roeslpa/ccpScript}.


\section{Notation}

\begin{tabular}{|l|p{13cm}|}\hline
    \textbf{Symbol} & \textbf{Meaning}\\\hline
    $\T$ / $\F$ & Boolean values true and false\\
    $\gets$ / $\to$ & Assigns a constant expression or the output of a deterministic algorithm\\
    $\getsr$ / $\tor$ & Assigns a random value uniformly sampled from a finite set or the output of a probabilistic algorithm\\
    $\PS(X)$ & Power set of set~$X$\\
    $(X)^+$ & Set of non-trivial concatenations of elements from set~$X$\\
    $X[\cdot]\gets x$ & Assigns all entries of array~$X$ with default value~$x$\\
    Invoke $X$ & Executes algorithm $X$\\
    Stop with $x$ & Terminates the running experiment with final output~$x$\\
    Require $x$ & Ends the algorithm, resp.~oracle, if expression~$x=\F$\\\hline
\end{tabular}


\section{Overview}
\label{sec:overview}

The goal of this course is to \emph{understand how to analyze the security of real-world protocols}.
\emph{Analyzing} comprises
\begin{enumerate}
    \item Methodically Defining Security
    \item Identifying the Cryptographic Core of Protocols
    \item Systematically Finding Attacks against Protocols
    \item Formally Proving Security of Protocols
\end{enumerate}
This course focuses on the following classes of \emph{real-world protocols} and their main building blocks:
\begin{itemize}
    \item Key Exchange Protocols
    \item Simple Communication Channels
    \item Messaging Protocols
    \item Group Membership Management Protocols
\end{itemize}

\subsection{First Informal Example: Diffie-Hellman Key Exchange}
\label{sec:overview:dhke}

Let $\DHgr=(\DHg,\DHp)$ be a group of prime order~$\DHp$ with generator~$\DHg$.
That is, $\DHgr$ is a set with a multiplicativly written operation~$\cdot$. A
simple example is the Group $\DHgr=(\ZZ_5^*,\cdot)$, where $\ZZ_5^* =
\{1,2,3,4\}$. Then a generator for $\DHgr$ is $\DHg=2$, since $2^0=1$, $2^1=2$,
$2^2=4$, $2^3=8\equiv3\mod 5$, and $2^4=16\equiv 1\mod 5$. (Applying the
group operation to the generator~$\DHg$ multiple times gives all group elements.)

Then, the Diffie-Hellman Key Exchange~(DHKE)~\cite{DifHel76}
between two users Alice and Bob works as follows:
\begin{enumerate}
    \item Alice samples $x\getsr\ZZ_\DHp^*$, locally stores $x$, and sends $X\gets(\DHg^x\mod\DHp)$ to Bob
    \item Bob samples $y\getsr\ZZ_\DHp^*$, computes $k\gets(X^y\mod\DHp)$, and sends $Y\gets(\DHg^y\mod\DHp)$ to Alice
    \item Alice computes $k\gets(Y^x\mod\DHp)$, and forgets $x$.
\end{enumerate}
Since $k\equiv X^y\equiv\DHg^{xy}\equiv Y^x\equiv k\mod \DHp$, Alice and Bob compute the same shared key.

\begin{figure}
    \centering
    \begin{tabular}{lC{3cm}l}
        \textbf{Alice} & & \textbf{Bob}\\
        $x\getsr\ZZ_\DHp^*$ & &\\
        $X\gets\DHg^x\mod\DHp$ & $\xrightarrow{X}$ & $y\getsr\ZZ_\DHp^*$\\
        & $\xleftarrow{Y}$ & $Y\gets \DHg^y\mod\DHp$\\
        $k\gets Y^x\mod\DHp$ & & $k\gets X^y\mod\DHp$\\
    \end{tabular}
    \caption{Diffie-Hellman Key Exchange.}
    \label{fig:dhke}
\end{figure}

\paragraph{Security of DHKE}
To prove security of the DHKE protocol, we use the \emph{Decisional Diffie-Hellman}~(DDH) problem (See Figure~\ref{fig:ddh_assumption}).
The advantage of an adversary~$\advA$ against the DDH problem in group~$\DHgr$ is defined as:
% \begin{align*}
%     \Adv_\DHgr^\ddh(\advA)\coloneqq|&\Pr[\advA(\DHg,\DHp,\DHg^x,\DHg^y,\DHg^{xy})=1\mid (x,y)\getsr(\ZZ_\DHp^*)^2]\\
%     &-\Pr[\advA(\DHg,\DHp,\DHg^x,\DHg^y,\DHg^z)=1\mid (x,y,z)\getsr(\ZZ_\DHp^*)^3]|\text{.}
% \end{align*}
%new advantage definition to be consistent with the lecture
\[\Adv_{\DHgr}^\ddh(\advA)\coloneqq\left|\Pr[\text{DDH}^0_{g,p}(\advA)=1]-\Pr[\text{DDH}^1_{g,p}(\advA)=1]\right|\]
\begin{figure}[t]
    \centering
    %!TEX root=../main.tex
\begin{tabular}{lll}
    \algbox{4.5cm}{%
        \textbf{DDH}$^b_{g,p}$\\
        $x,y,z \getsr \ZZ_p^*$\\
        $X\gets g^x, Y\gets g^y,Z_0\gets g^{xy}$\\
        $Z_1\gets g^z$\\~\\
        Return $b'$}&
    \arrbox{2cm}{%    
        ~\\~\\$\xrightarrow{(\DHg,\DHp,X,Y,Z_p)}$\\~\\
        $\xleftarrow{b'}$}&
    \algbox{3cm}{%
        \textbf{Adversary} $\advA$:\\~\\~\\~\\~\\}
\end{tabular}
    \caption{The Decisional Diffie-Hellman~(DDH) problem.}
    \label{fig:ddh_assumption}
\end{figure}
Or 
Intuitively, adversary~$\advA$ plays a game in which it sees both Diffie-Hellman shares $\DHg^x$ and $\DHg^y$ as well as either the real key~$\DHg^{xy}$~$(Z_0)$ or a random group element~$\DHg^z$~$(Z_1)$.
To win this game, the adversary has to \emph{decide} which of the two views they saw.
The advantage $\Adv_\DHgr^\ddh(\advA)$ denotes the probability of $\advA$ winning this game beyond pure guessing.

We define the \emph{Computational DH}~(CDH) and the \emph{Discrete Logarithm}~(DLog) problems for prime-order groups in Section~\ref{sec:asym:assumptions:dh}.

Based on the hardness of the best known attacks against the DDH problem from the long history of work in number theory, we assume that the DDH problem is \emph{hard} for security parameter~$\secp$.
This means that $\Adv_\DHgr^\ddh(\advA)$ is \emph{negligible} for all adversaries~$\advA$ that can be specified as \emph{probabilistic polynomial-time Turing machines} (PPT).

\paragraph{Simplifications}
For simplicity and clarity, we refrain from considering concrete group generation algorithms and we do not elaborate on hardness analyses of the DDH problem.
Furthermore, instead of treating security \emph{asymptotically} with respect to a security parameter~$\secp$, using the definitions of ``negligible'' and ``PPT adversaries'', we consider security \emph{concretely} in this document.
This means that all our reductionist security proofs provide concrete relations between the security of the analyzed protocols and the hardness of the underlying assumptions.

\paragraph{Real-World Remarks}
In practice, Elliptic Curves are often used to instantiate prime-order groups~$\DHgr$.
While the DHKE is one of the most widely used key exchange protocols in the real world, we want to mention that Shor's algorithm~\cite{FOCS:Shor94} solves the DDH problem in polynomial time using a quantum computer.
Thus, we only use the DHKE as a simple, introductory real-world example.

\paragraph{Intuitive Security Definition: Passive Adversaries}
A passive adversary~$\advA$ against the DHKE protocol is \emph{capable} of seeing the transcript of the interaction between Alice and Bob: $X=\DHg^x$ and $Y=\DHg^y$.
The \emph{goal} of the adversary is to derive information about the exchanged key~$k=\DHg^{xy}$ from that seen transcript.
To capture this goal, we challenge~$\advA$ eventually on key~$k$:
$\advA$ either sees the real key~$k_0=k=\DHg^{xy}$ or a randomly sampled element from the key space~$k_1\getsr\ksp$, where $k_1\gets\DHg^z$ and $z\getsr\ZZ_\DHp^*$.
If the adversary can distinguish these two cases, this is considered a \emph{successful} attack against the key exchange protocol.
In contrast, if we can prove that distinguishing the real key from a random key is hard, then we can treat the real key as if it was sampled randomly from a uniform distribution;
more intuitively, then we know that the adversary has no information about the real key, even when observing the interaction between Alice and Bob.

We emphasize that the above intuitive description of the adversarial capabilities, the attack goal, and the success criteria should only sketch the expected security requirements for the DHKE protocol.
As we will see, \emph{good} security definitions should not be specified with a concrete protocol design in mind.

\paragraph{Sketch of Security Proof}
The above described passive adversary~$\advA$ against DHKE obtains inputs $(X,Y)$ and $k_b$, where $X=\DHg^x$, $Y=\DHg^y$, $k_0=k=\DHg^{xy}$, $k_1=\DHg^z$, $(x,y,z)\in\ZZ_\DHp^*$, and $b\in\BB$.
If $\advA$, upon processing these inputs, outputs $b'$ such that $b=b'$, $\advA$ breaks the required security goal.

\begin{figure}
    \centering
    %!TEX root=../main.tex
\begin{tabular}{lllll}
    \algbox{3.5cm}{%
        \textbf{Challenger} $\advC_{\DHgr=(\DHg,\DHp)}^{\ddh,b}$:\\
        $(x,y,z)\getsr(\ZZ_\DHp^*)^3$\\
        $(X,Y)\gets(\DHg^x,\DHg^y)$\\
        $(Z_0,Z_1)\gets(\DHg^{xy},\DHg^z)$\\
        $Z\gets Z_b$\\\\
        Stop with $b''$}&
    \arrbox{2cm}{%
        $\xrightarrow{(\DHg,\DHp,X,Y,Z)}$\\~\\~\\
        $\xleftarrow{b''}$}&
    \algbox{3cm}{%
        \textbf{Reduction} $\advB^\ddh$:\\
        $k\gets Z$\\\\
        $b''\gets b'$}&
    \arrbox{2cm}{%
        $\xrightarrow{(\DHg,\DHp)}$\\
        $\xrightarrow{(X,Y)}$\\
        $\xrightarrow{k}$\\
        $\xleftarrow{b'}$}&
    \algbox{3cm}{%
        \textbf{Adversary} $\advA$\\\\\\}\\
\end{tabular}
    \caption{Sketch of reduction from Diffie-Hellman key exchange to DDH problem.}
    \label{fig:dhke:reduction}
\end{figure}

The primary proof method in this course is based on reductions via sequences of games~\cite{EPRINT:Shoup04} that we will specify in pseudo-code~\cite{EC:BelRog06}.
However, for the proof sketch of our simple example in which we reduce the security of DHKE to the hardness of the DDH problem, the schematic overview of reduction $\advB^\ddh$ in Figure~\ref{fig:dhke:reduction} suffices:
The reduction uses its DDH-challenge~$Z$ as the challenge key~$k$.
Once the DHKE-adversary~$\advA$ solves this challenge by successfully guessing bit~$b'$, the reduction can use this guess to break the DDH-challenge via bit~$b''$.
A formal proof remains a task for the interested reader.

\subsection{Second Example: Public-Key Encryption (PKE)}
\label{sec:overview:pke}

\paragraph{Syntax}
A public-key encryption scheme $\PKE=(\PKEgen,\PKEenc,\PKEdec)$ is a tuple of three algorithms with encryption key space~$\eksp$, decryption key space~$\dksp$, message space~$\msp$, and ciphertext space~$\csp$:

\begin{itemize}
    \item $\PKEgen: \emptyset \tor \dksp\times\eksp$
    \item $\PKEenc: \eksp\times\msp \tor \csp$
    \item $\PKEdec: \dksp\times\csp \to \msp$
\end{itemize}

\paragraph{Correctness}
A public-key encryption scheme $\PKE$ is correct if
\[
\Pr[\PKEdec(\dk,\PKEenc(\ek,m))=m\mid (\dk,\ek)\getsr\PKEgen,m\getsr\msp]=1\text{.}
\]

\paragraph{Passive Security}
We begin with a notion of passive security for PKE that is called \emph{Indistinguishability of Ciphertexts under Chosen Plaintext Attacks} (IND-CPA).
This models that adversaries only eavesdrop victims' ciphertexts for adversarially chosen messages---hence, \emph{chosen plaintext attacks}.
However, the adversary never observes the decryption of ciphertexts that were not created by the victims themselves.
The goal of the adversary is to decide which of two possible messages was encrypted in a challenge ciphertext---hence, \emph{indistinguishability of ciphertext}.

We introduce this passive notion of security mainly for didactic reasons.
Yet, we want to mention that the Fujisaki-Okamoto transform~\cite{C:FujOka99} lifts passively secure PKE schemes to actively secure ones.
Hence, this notion is indeed relevant in practice.

The advantage of an adversary~$\advA$ against public-key encryption scheme $\PKE$ in game $\INDCPA$ from Figure~\ref{fig:pke:ind:cpa} is defined as:
\[
\Adv_\PKE^\indcpa(\advA)\coloneqq\left|\Pr[\INDCPA_{\PKE}^0(\advA)=1]-\Pr[\INDCPA_{\PKE}^1(\advA)=1]\right|\text{.}
\]

\begin{figure}[!ht]
    \centering
    \nicoresetlinenr%
    \fbox{%
        \scalebox{\codescalefactor}{%
            %!TEX root=../main.tex
\markersetlen{ndL}{110pt}%
\markersetlen{ndR}{120pt}%
\newcommand{\CC}{\mathit{CC}}%
\begin{tabular}[t]{ll}
    \nicodemusbox{\markerlenndL}{%
        \textbf{Game} $\INDCPA_{\PKE}^b(\advA)$
        \begin{nicodemus}
            \item $(\dk,\ek)\getsr\PKEgen$
            \item $b'\getsr\advA(\ek)$
            \item Stop with~$b'$
        \end{nicodemus}%
    }%
    &
    \nicodemusbox{\markerlenndR}{%
        \textbf{Oracle} $\Ochall(m_0,m_1)$
        \begin{nicodemus}
            \item Require $\{m_0,m_1\}\subseteq\msp$
            \item $c\getsr\PKEenc(\ek,m_b)$
            \item Return~$c$
        \end{nicodemus}%
    }%
\end{tabular}%%
        }%
    }
    \caption{%
        Games $\INDCPA$ for public-key encryption scheme~$\PKE$.
    }
    \label{fig:pke:ind:cpa}
\end{figure}

\paragraph{Active Security}
The ability to actively manipulate ciphertexts and observe the victims' reaction on the decryption is modeled by providing a decryption oracle.
This additional oracle returns decrypted messages for all ciphertext queries except for those that correspond to challenge ciphertexts.
If also decryptions of challenge ciphertexts were given to the adversary, the adversary could trivially win the game for any functional PKE construction.
Therefore, we call such attack strategies \emph{trivial attacks} or \emph{trivial winning conditions}.

\begin{figure}[!ht]
    \centering
    \nicoresetlinenr%
    \fbox{%
        \scalebox{\codescalefactor}{%
            %!TEX root=../main.tex
\markersetlen{ndL}{110pt}%
\markersetlen{ndR}{120pt}%
\newcommand{\CC}{\mathit{CC}}%
\begin{tabular}[t]{ll}
    \nicodemusbox{\markerlenndL}{%
        \textbf{Game} $\INDCCA_{\PKE}^b(\advA)$
        \begin{nicodemus}
            \item $\CC\gets\emptyset$
            \item $(\dk,\ek)\getsr\PKEgen$
            \item $b'\getsr\advA(\ek)$
            \item Stop with~$b'$
        \end{nicodemus}%
    }%
    &
    \nicodemusbox{\markerlenndR}{%
        \textbf{Oracle} $\Ochall(m_0,m_1)$
        \begin{nicodemus}
            \item Require $\{m_0,m_1\}\subseteq\msp$
            \item $c\getsr\PKEenc(\ek,m_b)$
            \item $\CC\gets\CC\cup\{c\}$
            \item Return~$c$
        \end{nicodemus}%
        \medskip
        
        \textbf{Oracle} $\Odec(c)$
        \begin{nicodemus}
            \item Require $c\notin\CC$
            \item $m\gets\PKEdec(\dk,c)$
            \item Return $m$
        \end{nicodemus}%
    }%
\end{tabular}%%
        }%
    }
    \caption{%
        Games $\INDCCA$ for public-key encryption scheme~$\PKE$.
    }
    \label{fig:pke:ind}
\end{figure}

The advantage of an adversary~$\advA$ against public-key encryption scheme $\PKE$ in game $\INDCCA$ from Figure~\ref{fig:pke:ind} is defined as:
\[
\Adv_\PKE^\indcca(\advA)\coloneqq\left|\Pr[\INDCCA_{\PKE}^0(\advA)=1]-\Pr[\INDCCA_{\PKE}^1(\advA)=1]\right|\text{.}
\]

\paragraph{ElGamal Encryption}
As shown in Figure~\ref{fig:pke:const:elgamal}, the ElGamal public-key encryption scheme~\cite{ElGamal85} at its core uses the Diffie-Hellman key exchange.
Intuitively, the ElGamal PKE key pair~$(\dk,\ek)$ is Alice's key pair in the DHKE.
For ElGamal encryption, one samples a fresh, ephemeral DHKE key pair, which corresponds to Bob's key in the DHKE.
Using Alice's DHKE public key and Bob's DHKE secret key, one computes the shared key and multiplies the message to that shared key.
On ElGamal decryption, the same shared key is computed and, by multiplying with its inverse, the original, encrypted message is derived.

That decryption yields the original message can be seen from the following equation:
\[m\gets\frac{c_2}{c_1^\dk}=\frac{\ek^x\cdot m}{(g^x)^\dk}=\frac{g^{x\dk}\cdot m}{g^{x\dk}}=m\]
\begin{figure}[!ht]
    \centering
    \nicoresetlinenr%
    \fbox{%
        \scalebox{\codescalefactor}{%
            %!TEX root=../main.tex
\markersetlen{ndL}{110pt}%
\markersetlen{ndM}{120pt}%
\markersetlen{ndR}{120pt}%
\begin{tabular}[t]{lll}
    \nicodemusbox{\markerlenndL}{%
        \textbf{Proc} $\PKEgen$
        \begin{nicodemus}
            \item $\dk\getsr\ZZ_\DHp^*$
            \item $\ek\gets\DHg^\dk$
            \item Return $(\dk,\ek)$
        \end{nicodemus}%
    }%
    &
    \nicodemusbox{\markerlenndM}{%
        \textbf{Proc} $\PKEenc(\ek,m)$
        \begin{nicodemus}
            \item $x\getsr\ZZ_\DHp^*$
            \item $c_1\gets\DHg^x$
            \item $c_2\gets\ek^x\cdot m$
            \item $c\gets(c_1,c_2)$
            \item Return~$c$
        \end{nicodemus}%
    }%
    &
    \nicodemusbox{\markerlenndR}{%
        \textbf{Proc} $\PKEdec(\dk,c)$
        \begin{nicodemus}
            \item $(c_1,c_2)\gets c$
            \item $m\gets c_2/c_1^\dk$
            \item Return $m$
        \end{nicodemus}%
    }%
\end{tabular}%%
        }%
    }
    \caption{%
        ElGamal public-key encryption scheme~$\PKE$~\cite{ElGamal85}.
    }
    \label{fig:pke:const:elgamal}
\end{figure}

\paragraph{Security of ElGamal}
To prove IND-CPA security of the ElGamal encryption scheme, we assume that there is a successful adversary~$\advA$ against the IND-CPA security of $\PKE$ (Figure~\ref{fig:elgamal:ind}).
We then construct another game $G_1^b$ (Figure~\ref{fig:elgamal:g1}) that is a small modification of the ElGamal IND-CPA game.
There can be no adversary against $G_1^b$ that does better than just guessing the bit $b$.
This is because every value exposed to the adversary is independent of $b$.
For $\pk$ and $c_1$ this is obvious and since $z$ is uniformly sampled from $\ZZ_p^*$ $c_2$ is also random.

To show that games $G_1^b$ and $\INDCPA_\PKE^b$ are indistinguishable, we turn every successful distinguisher~$\advD$ into a successful adversary against the DDH problem (Figure~\ref{fig:elgamal:reduction}).
The distinguisher~$\advD$ is an algorithm that outputs $d\gets\{0,1\}$ depending on which game ($\INDCPA_\PKE^b$ or $G_1^b$) it is playing against.

\begin{figure}[!ht]
    \centering
    \begin{minipage}{0.49\textwidth}
        \centering
        %!TEX root=../main.tex
\begin{tabular}{lll}
    \algbox{3.5cm}{%
        \textbf{Game} $\INDCPA_\PKE^b$:\\
        $\dk\gets\ZZ_p^*,\pk\gets\DHg^\dk$\\
        $y\getsr\ZZ_p^*$\\
        $c_1\gets\DHg^y, c_2^1\gets\DHg^{\dk\cdot y}$\\
        $c_2\gets c_2^1\cdot m_b$\\
        $c^*\gets(c_1,c_2)$\\
        Return $b'$}&
    \arrbox{1cm}{%
        $\xrightarrow{\pk}$\\   
        $\xleftarrow{m_0,m_1}$\\
        $\xrightarrow{c^*}$\\
        $\xleftarrow{b'}$}&
    \algbox{2.5cm}{%
        \textbf{Adversary} $\advA$\\~\\~\\~\\~\\~\\}
\end{tabular}
        \caption{Game $\INDCPA_\PKE^b$}
        \label{fig:elgamal:ind}
    \end{minipage}
    \hfill
    \begin{minipage}{0.49\textwidth}
        \centering
        %!TEX root=../main.tex
\begin{tabular}{lll}
    \algbox{3.5cm}{%
        \textbf{Game} $G_1^b$:\\
        $\dk\gets\ZZ_p^*,\pk\gets\DHg^\dk$\\
        $y,{\color{red}z}\getsr\ZZ_p^*$\\
        $c_1\gets\DHg^y, c_2^1\gets{\color{red}\DHg^z}$\\
        $c_2\gets c_2^1\cdot m_b$\\
        $c^*\gets(c_1,c_2)$\\
        Return $b'$}&
    \arrbox{1cm}{%
        $\xrightarrow{\pk}$\\
        $\xleftarrow{m_0,m_1}$\\
        $\xrightarrow{c^*}$\\
        $\xleftarrow{b'}$}&
    \algbox{2.5cm}{%
        \textbf{Adversary} $\advA$\\~\\~\\~\\~\\~\\}
\end{tabular}
        \caption{Game $G_1^b$}
        \label{fig:elgamal:g1}
    \end{minipage}
\end{figure}
\begin{figure}[!ht]
    \centering
    %!TEX root=../main.tex
\begin{tabular}{lllll}
    \algbox{4cm}{%
        \textbf{Game} $\advC_{\DHgr=(\DHg,\DHp)}^{\ddh,b}$:\\
        $x,y,z\getsr(\ZZ_\DHp^*)^3$\\
        $X\gets\DHg^x,Y\gets\DHg^y, Z_0\gets\DHg^{xy}$\\
        $Z_1\gets\DHg^z, Z\gets Z_{b'}$\\~\\~\\
        Return $b''$
        }&
    \arrbox{2cm}{%
        $\xrightarrow{(\DHg,\DHp,X,Y,Z)}$\\~\\~\\
        $\xleftarrow{b''}$}&
    \algbox{3cm}{%
        \textbf{Reduction} $\advB^\ddh$:\\
        $\pk\gets X$\\
        $c_1\gets Y, c_2^1\gets Z$\\
        $c_2\gets c_2^1\cdot m_b$\\
        $c^*\gets(c_1,c_2)$\\
        $b''\gets d$\\
        Return $b''$}&
    \arrbox{2cm}{%
        $\xrightarrow{\pk}$\\
        $\xleftarrow{m_0,m_1}$\\
        $\xrightarrow{c^*}$\\
        $\xleftarrow{d}$}&
    \algbox{3cm}{%
        \textbf{Adversary} $\advD$\\~\\~\\~\\~\\~\\} 
\end{tabular}
    \caption{Sketch of reduction from distinguisher $\advD$ to DDH problem.}
    \label{fig:elgamal:reduction}
\end{figure}



\section{Game-Based Definitions}
As one can see from the minimal differences between the \emph{passive} security definition and the \emph{active} security definition for PKE from Section~\ref{sec:overview:pke}, definitions of security can be modular with respect to a pattern that is based on a small core.
We will use the example of \emph{Key Encapsulation Mechanisms}~(KEM) to introduce a methodology for defining security systematically.
This methodology is closely related to a concept called \emph{Indistinguishability up to Correctness} by Rogaway and Zhang~\cite{C:RogZha18}.

\subsection{Syntax}
The first step to formally consider a cryptographic primitive is defining its \emph{syntax}.
The syntax specifies the algorithms of a primitive by defining their inputs and outputs.

A key encapsulation mechanism $\KEM=(\KEMgen,\KEMenc,\KEMdec)$ is a tuple of three algorithms with encapsulation key space~$\eksp$, decapsulation key space~$\dksp$, ciphertext space~$\csp$, and symmetric key space~$\ksp$:
\begin{itemize}
    \item $\KEMgen: \emptyset \tor \dksp\times\eksp$
    \item $\KEMenc: \eksp \tor \ksp\times\csp$
    \item $\KEMdec: \dksp\times\csp \to \ksp$
\end{itemize}

Intuitively a KEM can be considered a special case of PKE that is limited to always encrypt (aka.\ encapsulate) fresh symmetric keys.
Instead of taking the symmetric keys as input, the encapsulation algorithm of a KEM generates them internally.

\paragraph{Construction: ElGamal KEM}
For illustration, we give the ElGamal KEM as a simple example of a KEM in Figure~\ref{fig:kem:const:elgamal}.

\begin{figure}[!ht]
    \centering
    \nicoresetlinenr%
    \fbox{%
        \scalebox{\codescalefactor}{%
            %!TEX root=../main.tex
\markersetlen{ndL}{110pt}%
\markersetlen{ndM}{120pt}%
\markersetlen{ndR}{120pt}%
\begin{tabular}[t]{lll}
    \nicodemusbox{\markerlenndL}{%
        \textbf{Proc} $\KEMgen$
        \begin{nicodemus}
            \item $\dk\getsr\ZZ_\DHp^*$
            \item $\ek\gets\DHg^\dk$
            \item Return $(\dk,\ek)$
        \end{nicodemus}%
    }%
    &
    \nicodemusbox{\markerlenndM}{%
        \textbf{Proc} $\KEMenc(\ek)$
        \begin{nicodemus}
            \item $x\getsr\ZZ_\DHp^*$
            \item $k\gets\ek^x$
            \item $c\gets\DHg^x$
            \item Return~$(k,c)$
        \end{nicodemus}%
    }%
    &
    \nicodemusbox{\markerlenndR}{%
        \textbf{Proc} $\KEMdec(\dk,c)$
        \begin{nicodemus}
            \item $k\gets c^\dk$
            \item Return $k$
        \end{nicodemus}%
    }%
\end{tabular}%%
        }%
    }
    \caption{%
        ElGamal key encapsulation mechanism~$\KEM$~\cite{ElGamal85}.
    }
    \label{fig:kem:const:elgamal}
\end{figure}

\subsection{Correctness}
Before specifying the expected security guarantees, we define the \emph{correctness} requirements.
Typically, these requirements capture the functionality that a primitive should offer when being executed in an \emph{honest} environment in which all traffic is delivered as intended.
The literature on distributed systems developed two complementary notions of correctness:
\begin{itemize}
    \item \textbf{Safety:} Nothing ``bad'' should happen (e.g., encapsulation and decapsulation for the same ciphertext should not compute different symmetric keys)
    \item \textbf{Liveness:} The ``good'' event always happens (e.g., encapsulation and decapsulation always output a non-trivial key)
\end{itemize}

A correctness definition in the sense of \emph{safety} typically suffices for the ultimate goal of defining and analyzing security.
Furthermore, specifying \emph{liveness} requirements for sophisticated primitives sometimes leads to complex, incomprehensible definitions.

A key encapsulation mechanism $\KEM$ is correct if
\[
\Pr[\KEMdec(\dk,c)=k\mid (\dk,\ek)\getsr\KEMgen,(k,c)\getsr\KEMenc(\ek))=m]=1\text{.}
\]
Equivalently, a key encapsulation mechanism $\KEM$ is correct if $\Pr[\CORR_{\KEM}(\advA)=0]=1$ for all adversaries~$\advA$, where game~$\CORR$ is defined in Figure~\ref{fig:kem:corr}.

\begin{figure}[!ht]
    \centering
    \nicoresetlinenr%
    \fbox{%
        \scalebox{\codescalefactor}{%
            %!TEX root=../main.tex
\markersetlen{ndL}{110pt}%
\markersetlen{ndR}{100pt}%
\newcommand{\CK}{\mathit{CK}}%
\begin{tabular}[t]{ll}
    \nicodemusbox{\markerlenndL}{%
        \textbf{Game} $\CORR_{\KEM}(\advA)$
        \begin{nicodemus}
            \item $\CK[\cdot]\gets\bot$
            \item $(\dk,\ek)\getsr\KEMgen$
            \item Invoke $\advA(\ek)$
            \item Stop with~$0$
        \end{nicodemus}%
        \medskip
        
        \textbf{Oracle} $\Oenc$
        \begin{nicodemus}
            \item $(k,c)\getsr\KEMenc(\ek)$
            \item $\CK[c]\gets k$
            \item Return~$c$
        \end{nicodemus}%
    }%
    &
    \nicodemusbox{\markerlenndR}{%
        \textbf{Oracle} $\Odec(c)$
        \begin{nicodemus}
            \item $k'\gets\KEMdec(\dk,c)$
            \item If $\CK[c]\notin\{k',\bot\}$:
            \item \quad Stop with $1$
            \item Return $k'$
        \end{nicodemus}%
    }%
\end{tabular}%%
        }%
    }
    \caption{%
        Game $\CORR$ for key encapsulation mechanism~$\KEM$.
    }
    \label{fig:kem:corr}
\end{figure}

\subsection{Adversarial Capabilities}
Defining \emph{security} can be divided into three components:
modeling the \emph{capabilities of adversaries},
specifying the \emph{adversaries' goal}, and
identifying attack strategies that are \emph{trivially} successful against every possible construction.
The former two components are typically influenced by intuitive purpose of the considered primitive and how this primitive is used in practice;
the latter one is almost entirely determined by all remaining definitional steps.

\subsubsection{Typical Capabilities}
We begin with considering typical adversarial capabilities against KEMs:
\begin{itemize}
    \item \textbf{Chosen-Plaintext Attacks:}
    The adversary can see the ciphertexts of encapsulated keys
    \item \textbf{Chosen-Ciphertext Attacks:}
    The adversary can see the decapsulated keys for chosen ciphertexts
    \item \textbf{Secret Key Corruption:}
    The adversary can learn the entire decapsulation key
    \item \textbf{Leakage of Information during Algorithm Execution:}
    The adversary can learn secret bits of variables processed by the encapsulation or decapsulation algorithm
    \item \textbf{Subversion of Algorithms:}
    The adversary can modify the encapsulation or decapsulation algorithm
    \item Etc.
\end{itemize}

\paragraph{Chosen-Plaintext Attacks}
The most standard adversarial capabilities that are considered realistic and defendable for KEMs are chosen-plaintext and chosen-ciphertext attacks (CPA resp.\ CCA).

Consider the adversarial capabilities shown in Figure~\ref{fig:adv_capabilities}.
Here, the adversary can see the values in the dashed boxes, as well as send values (in red).
Observing the values $\ek$ and $c$ is possible by eavesdropping, but the fact that $k'$ is visible might be surprising at first.
This is because an active adversary can send $c'$ and observe certain \emph{side-channels} to learn information about $k'$.
E.g. the adversary can measure the time it takes to decapsulate $c'$ or send an invalid $c'$ and observe the error message.
To model this capability we provide the adversary with the ability to choose ciphertexts for decapsulation and provide it with the decryption oracle (CCA).

\begin{figure}[!ht]
    \centering
    %!TEX root=../main.tex
\begin{tabular}{lcl}
    \underline{Alice} & \underline{$\advA$} & \underline{Bob}\\
    $(k,c)\getsr\KEMenc(\ek)$ & \arrbox{2cm}{$\xleftarrow{\dbox{$\ek$}}$} & $(\dk,\ek)\getsr\KEMgen$\\
    & $\xrightarrow{\dbox{$c$}}$ & $k\gets\KEMdec(\dk,c)$\\
    & {\color{red}$\xrightarrow{c'}$} & $\dbox{$k'$}\gets\KEMdec(\dk,c')$
\end{tabular}%
    \caption{%
        A sketch of common adversarial capabilities.
    }
    \label{fig:adv_capabilities}
\end{figure}

\paragraph{Chosen-Ciphertext Attacks}
The game $\INDCCA$ is defined in Figure~\ref{fig:kem:ind}.

\begin{figure}[!ht]
    \centering
    \nicoresetlinenr%
    \fbox{%
        \scalebox{\codescalefactor}{%
            %!TEX root=../main.tex
\markersetlen{ndL}{110pt}%
\markersetlen{ndR}{120pt}%
\newcommand{\CC}{\mathit{CC}}%
\begin{tabular}[t]{ll}
    \nicodemusbox{\markerlenndL}{%
        \textbf{Game} $\INDCCA_{\KEM}^b(\advA)$
        \begin{nicodemus}
            \item $\CC\gets\emptyset$
            \item $(\dk,\ek)\getsr\KEMgen$
            \item $b'\getsr\advA(\ek)$
            \item Stop with~$b'$
        \end{nicodemus}%
    }%
    &
    \nicodemusbox{\markerlenndR}{%
        \textbf{Oracle} $\Ochall$
        \begin{nicodemus}
            \item $(k_0,c)\getsr\KEMenc(\ek)$
            \item $k_1\getsr\ksp$
            \item $\CC\gets\CC\cup\{c\}$
            \item Return~$(k_b,c)$
        \end{nicodemus}%
        \medskip
        
        \textbf{Oracle} $\Odec(c)$
        \begin{nicodemus}
            \item Require $c\notin\CC$
            \item $k\gets\KEMdec(\dk,c)$
            \item Return $k$
        \end{nicodemus}%
    }%
\end{tabular}%%
        }%
    }
    \caption{%
        Games $\INDCCA$ for key encapsulation mechanism~$\KEM$.
    }
    \label{fig:kem:ind}
\end{figure}

\subsection{Adversarial Goal}
The adversarial goal is the second component of defining security.
It describes what the adversary is trying to learn. 
The adversarial goal is typically denoted in the first part of the name of the security game (e.g. \emph{IND}-CPA or \emph{OW}-CCA)

\subsubsection{Typical Goals}
For KEMs, the typical adversarial goals are either indistinguishability of ciphertexts or one-way security.

\paragraph{One-Way Security}
One-way security~(OW-security) is the intuitive definition of the adversarial goal. 
For a KEM, OW-security defines this to be learning the encapsulated key $k$.
The security game $\OWCCA$ is defined in Figure~\ref{fig:kem:ow:cca} and the $\OWCPA$ game follows from it by excluding the decryption oracle.

\begin{figure}[!ht]
    \centering
    \nicoresetlinenr%
    \fbox{%
        \scalebox{\codescalefactor}{%
            %!TEX root=../main.tex
\markersetlen{ndL}{110pt}%
\markersetlen{ndR}{120pt}%
\newcommand{\CC}{\mathit{CC}}%
\begin{tabular}[t]{ll}
    \nicodemusbox{\markerlenndL}{%
        \textbf{Game} $\OWCCA_{\KEM}(\advA)$
        \begin{nicodemus}
            \item $\CC\gets\emptyset$
            \item $K\gets\emptyset$
            \item $(\dk,\ek)\getsr\KEMgen$
            \item $k'\getsr\advA(\ek)$
            \item Stop with~$k'\in K$
        \end{nicodemus}%
    }%
    &
    \nicodemusbox{\markerlenndR}{%
        \textbf{Oracle} $\Ochall$
        \begin{nicodemus}
            \item $(k,c)\getsr\KEMenc(\ek)$
            \item $K\gets K\cup\{k\}$
            \item $\CC\gets\CC\cup\{c\}$
            \item Return~$(c)$
        \end{nicodemus}%
        \medskip
        
        \textbf{Oracle} $\Odec(c)$
        \begin{nicodemus}
            \item Require $c\notin\CC$
            \item $k\gets\KEMdec(\dk,c)$
            \item Return $k$
        \end{nicodemus}%
    }%
\end{tabular}%%
        }%
    }
    \caption{%
        One-Way CCA security game for $\KEM$.
    }
    \label{fig:kem:ow:cca}
\end{figure}

\paragraph{Indistinguishability of Ciphertexts}
As we have seen in Section~\ref{sec:overview:pke}, we sometimes want \emph{indistinguishability} of the generated ciphertext and a randomly chosen ciphertext.
The reason for this is that it gives us a stronger security guarantee. 
Namely, if a construction is IND secure, the adversary is unable to learn a single bit of the secret.
The $\INDCCA$ game is defined in Figure~\ref{fig:kem:ind}

\subsection{Trivial Winning Strategies}\label{sec:kem:trivial_attacks}
So far, we have done some unexplained bookkeeping in our security games. The reason for that are \emph{trivial winning strategies}.
These are strategies or attacks that work against every correct construction, simply because the constructions have to give sensible responses.
For our KEM there are two trivial winning conditions we need to guard against (Figure~\ref{kem:triv}).

\begin{figure}[!ht]%
    \centering
    %!TEX root=../main.tex
\parbox[t]{3.5cm}{%
    $\OWCCA$:
    \begin{enumerate}[topsep=0pt]
        \item $c\getsr\Ochall$
        \item $k\gets\Odec(c)$
        \item Stop with $k$
    \end{enumerate}    
}\parbox[t]{3.5cm}{%
    $\INDCCA$:
    \begin{enumerate}[topsep=0pt]
        \item $(k,c)\getsr\Ochall$
        \item $k'\gets\Odec(c)$
        \item Stop with $k\not= k'$
    \end{enumerate}
} 
    \caption{Trivial winning strategies against $\OWCCA$ and $\INDCCA$ $\KEM$ games.}
    \label{kem:triv}
\end{figure}

Now that the trivial winning conditions are defined, we can specify the advantage terms for each game.
For the OW-CPA and OW-CCA games that is:
\begin{align*}
    \Adv_\KEM^\owcpa(\advA)\coloneqq&\Pr[\OWCPA_{\KEM}(\advA)=1]\\
    \Adv_\KEM^\owcca(\advA)\coloneqq&\Pr[\OWCCA_{\KEM}(\advA)=1]
\end{align*}
And the advantages for the IND games are defined as:
\begin{align*}
    \Adv_\KEM^\indcpa(\advA)\coloneqq&\left|\Pr[\INDCPA_{\KEM}^0(\advA)=1]-\Pr[\INDCPA_{\KEM}^1(\advA)=1]\right|\\
    \Adv_\KEM^\indcca(\advA)\coloneqq&\left|\Pr[\INDCCA_{\KEM}^0(\advA)=1]-\Pr[\INDCCA_{\KEM}^1(\advA)=1]\right|
\end{align*}

If there exists a provably secure construction for a definition, then there are no more open trivial winning conditions.

\paragraph{Multi-Instance Security}
Definitions of security, such as the IND-CCA game, differ depending on the context.
One example of that would be the multi-instance security game (Figure~\ref{fig:kem:ind:mi:corrupt}).
This game is called \emph{multi-instance} because it captures a setting in which many instances are running concurrently.
This can be the case in TLS for example, where many connections are established at the same time.
If we were to use the IND-CCA game from Figure~\ref{fig:kem:ind}, then we would have to use the \emph{hybrid argument} to prove security.
The hybrid argument can be used to prove security for multiple instances, by showing step by step that each instance is secure and then showing the indistinguishability between each step.
Doing so however, yields a factor $q$ in the final security bound, where $q$ is the number of instances.
To avoid this degeneration of the security bound, we can use the multi-instance game, for which sometimes tighter security bounds exist.

\begin{figure}[!ht]
    \centering
    \nicoresetlinenr%
    \fbox{%
        \scalebox{\codescalefactor}{%
            %!TEX root=../main.tex
\markersetlen{ndL}{130pt}%
\markersetlen{ndR}{120pt}%
\newcommand{\CC}{\mathit{CC}}%
\newcommand{\CR}{\mathit{CR}}%
\begin{tabular}[t]{ll}
    \nicodemusbox{\markerlenndL}{%
        \textbf{Game} $\INDCCA_{\KEM}^b(\advA)$
        \begin{nicodemus}
            \item $n\gets0$
            \item $\CR\gets\emptyset$
            \item $b'\getsr\advA$
            \item Require $\forall i\in\CR:\CC_i=\emptyset$
            \item Stop with~$b'$
        \end{nicodemus}%
        \medskip
        
        \textbf{Oracle} $\Ogen$
        \begin{nicodemus}
            \item $\CC_n\gets\emptyset$
            \item $(\dk_n,\ek_n)\getsr\KEMgen$
            \item $n\gets n+1$
            \item Return $\ek_{n-1}$
        \end{nicodemus}%
    }%
    &
    \nicodemusbox{\markerlenndR}{%
        \textbf{Oracle} $\Ochall(i)$
        \begin{nicodemus}
            \item Require $i\in[n]$
            \item $(k_0,c)\getsr\KEMenc(\ek_i)$
            \item $k_1\getsr\ksp$
            \item $\CC_i\gets\CC_i\cup\{c\}$
            \item Return~$(k_b,c)$
        \end{nicodemus}%
        \medskip
        
        \textbf{Oracle} $\Odec(i,c)$
        \begin{nicodemus}
            \item Require $i\in[n]\land c\notin\CC_i$
            \item $k\gets\KEMdec(\dk_i,c)$
            \item Return $k$
        \end{nicodemus}%
        \medskip
        
        \textbf{Oracle} $\Ocorrupt(i)$
        \begin{nicodemus}
            \item Require $i\in[n]$
            \item $\CR\gets\CR\cup i$
            \item Return $\dk$
        \end{nicodemus}%
    }%
\end{tabular}%%
        }%
    }
    \caption{%
        Multi-instance games $\INDCCA$ for key encapsulation mechanism~$\KEM$.
    }
    \label{fig:kem:ind:mi:corrupt}
\end{figure}

\section{Forward Security}
\label{sec:forward-sec}

\paragraph{Motivation} Until now, we have used a secret key that was only known by one or more honest parties participating in the various protocols.
However in the real world, the secret key often gets compromised, and the adversary learns some information about it.
This might happen if the device gets stolen or confiscated, if the honest parties get hacked or just through human error.
If we use our KEM from Figure~\ref{fig:kem:const:elgamal} as an example, this would enable the adversary to decapsulate every single ciphertext that was sent (if the adversary recorded the ciphertexts).
Forward Security guarantees that messages before the compromise remain secure (Figure~\ref{fig:fs:sketch}).

In this section, we will discuss how to achieve security even if the secret key is compromised in the future, using our KEM as an example.

\begin{figure}[!ht]
    \centering
    \begin{tikzpicture}%
    \draw [green, -{>Bar}] (0,0) -- (4.8,0) node [midway, above] {secure};
    \node at (5,0) {\color{red}\LARGE\Lightning};
    \node at (5,-.5) {$\dk$ compromised};
    \draw [-{>Bar}] (5.2,0) -- (10,0) node [midway, above] {not secure};
\end{tikzpicture}%
    \caption{A sketch of forward security}
    \label{fig:fs:sketch}
\end{figure}

\subsection{Forward Secure KEMs}

\alert{add \cite{SP:GreMie15,EC:CanHalKat03,EC:GHJL17}}

\paragraph{Syntax} To provide security even if the secret key is leaked, we have to change the secret key over time.
This means we also need to change the syntax of the decapsulation algorithm.
The forward-secure KEM $\FKEM=(\FKEMgen,\FKEMenc,\FKEMdec)$ is a tuple of three algorithms with encapsulation key space~$\eksp$, decapsulation key space~$\dksp$, ciphertext space~$\csp$, and symmetric key space~$\ksp$:
\begin{itemize}
    \item $\KEMgen: \emptyset \tor \dksp\times\eksp$
    \item $\KEMenc: \eksp \tor \ksp\times\csp$
    \item $\KEMdec: \dksp\times\csp \to \dksp\times\ksp$
\end{itemize}

\paragraph{Correctness} A forward-secure KEM $\FKEM$ is correct if $\Pr[\CORR_{\FKEM}(\advA)=0]=1$ for all adversaries~$\advA$, where game~$\CORR$ is defined in Figure~\ref{fig:fkem:corr}.

\paragraph{Security}
The advantage of an adversary~$\advA$ against forward-secure KEM $\FKEM$ in game $\INDCCA$ from Figure~\ref{fig:fkem:ind} is defined as:
\[
\Adv_\FKEM^\indcca(\advA)\coloneqq\left|\Pr[\INDCCA_{\FKEM}^0(\advA)=1]-\Pr[\INDCCA_{\FKEM}^1(\advA)=1]\right|\text{.}
\]

\begin{figure}[!ht]
    \begin{subfigure}{.49\textwidth}
        \centering
        \nicoresetlinenr%
        \fbox{%
            \scalebox{\codescalefactor}{%
                % !TeX root = ..\..\main.tex
\markersetlen{ndL}{120pt}%
\markersetlen{ndR}{130pt}%
\newcommand{\CK}{\mathit{CK}}%
\begin{tabular}[t]{ll}
    \nicodemusbox{\markerlenndL}{%
        \textbf{Game} $\CORR_{\FKEM}(\advA)$
        \begin{nicodemus}
            \item $\CK[\cdot]\gets\bot$
            \item $(\dk,\ek)\getsr\FKEMgen$
            \item Invoke $\advA(\ek)$
            \item Stop with~$0$
        \end{nicodemus}%
        \medskip
        
        \textbf{Oracle} $\Oenc$
        \begin{nicodemus}
            \item $(k,c)\getsr\FKEMenc(\ek)$
            \item If $\CK[c]\neq\diamond$: $\CK[c]\gets k$
            \item Return~$c$
        \end{nicodemus}%
    }%
    &
    \nicodemusbox{\markerlenndR}{%
        \textbf{Oracle} $\Odec(c)$
        \begin{nicodemus}
            \item $(\dk,k')\gets\FKEMdec(\dk,c)$
            \item If $\CK[c]\notin\{k',\bot,\diamond\}$:
            \item \quad Stop with $1$
            \item If $\CK[c]\neq\bot$: $\CK[c]\gets \diamond$
            \item Return $k'$
        \end{nicodemus}%
    }%
\end{tabular}%%
            }%
        }
        \caption{%
            $\CORR$ game.
        }
        \label{fig:fkem:corr}    
    \end{subfigure}
    \hfill
    \begin{subfigure}{.49\textwidth}
        \centering
        \nicoresetlinenr%
        \fbox{%
            \scalebox{\codescalefactor}{%
                % !TeX root = ..\..\main.tex
\markersetlen{ndL}{130pt}%
\markersetlen{ndR}{130pt}%
\newcommand{\CC}{\mathit{CC}}%
\begin{tabular}[t]{ll}
    \nicodemusbox{\markerlenndL}{%
        \textbf{Game} $\INDCCA_{\FKEM}^b(\advA)$
        \begin{nicodemus}
            \item $x\gets 0$
            \item $\CC\gets\emptyset$
            \item $(\dk,\ek)\getsr\FKEMgen$
            \item $b'\getsr\advA(\ek)$
            \item Stop with~$b'$
        \end{nicodemus}%
        \medskip
        
        \textbf{Oracle} $\Ochall$
        \begin{nicodemus}
            \item If $x=1$: Return $\bot$
            \item $(k_0,c)\getsr\FKEMenc(\ek)$
            \item $k_1\getsr\ksp$
            \item $\CC\gets\CC\cup\{c\}$
            \item Return~$(k_b,c)$
        \end{nicodemus}%
    }%
    &
    \nicodemusbox{\markerlenndR}{%
        \textbf{Oracle} $\Odec(c)$
        \begin{nicodemus}
            \item $(\dk,k)\gets\FKEMdec(\dk,c)$
            \item If $c\in\CC$: $k\gets\bot$
            \item $\CC\gets\CC\setminus\{c\}$
            \item Return $k$
        \end{nicodemus}%
        \medskip
        
        \textbf{Oracle} $\Ocorrupt$
        \begin{nicodemus}
            \item Require $\CC=\emptyset$
            \item Return~$\dk$
        \end{nicodemus}%
    }%
\end{tabular}%%
            }%
        }
        \caption{%
            $\INDCCA$ game.
        }
        \label{fig:fkem:ind}
    \end{subfigure}
    \caption{Games for forward-secure KEM~$\FKEM$.}
\end{figure}

\paragraph{Trivial winning strategies} In addition to the trivial winning strategies for Game~$\INDCCA^b_\KEM$ from Section~\ref{sec:kem:trivial_attacks}, two new trivial attacks have to be considered (Figure~\ref{fig:fkem:triv}).

\begin{figure}[!ht]%
    \centering
    % !TeX root = ..\..\main.tex
\parbox[t]{5cm}{%
    \begin{enumerate}[topsep=0pt]
        \item $\Ocorrupt\to\dk$
        \item $\Ochall\to(k,c)$
        \item $(\dk',k')\gets\FKEMdec(\dk,c)$
    \end{enumerate}
}\parbox[t]{5cm}{%
    \begin{enumerate}[topsep=0pt]
        \item $\Ochall\to(k,c)$
        \item $\Ocorrupt\to\dk$
        \item $(\dk',k')\gets\FKEMdec(\dk,c)$
    \end{enumerate}
}
    \caption{Additional Trivial winning strategies for forward-secure KEMs.}
    \label{fig:fkem:triv}
\end{figure}

\subsubsection{Hierarchical Identity-Based Encryption}

To construct a forward-secure KEM, we can use \emph{Hierarchical Identity-Based Encryption}~($\HIBE$) schemes.
A $\HIBE$ scheme consists of the following algorithms: $\HIBE = (\HIBEgen,\HIBEenc,\HIBEdec, \HIBEdel)$.

We motivate $\HIBE$ with the example of domain based infrastructure.
Due to its inefficiency, $\HIBE$ is not used in practice for this, but it is a good example to understand the concept.
In our example Alice wants to send a key to Bob to use with public encryption.
Ideally, Alice would not have to know Bob's public key, but only the public key of the top domain (i.e.\ \texttt{.de}).
Through a series of delegations~($\HIBEdel$) Bob's secret key~$\sk''''$ is derived from the top domain's secret key~$\sk$.

\begin{figure}[!ht]
    \centering
    % !TeX root = ..\main.tex
\begin{tikzpicture}
    \node (A) at (0,0) {\underline{Alice}};
    \node (DENIC) at (7.5,0){\underline{DENIC}};

    \node [below = .5cm of DENIC, anchor=north](DE) {\texttt{.de}};
    \node [anchor=east] (KEYGEN) at ($ (DENIC)!.5!(DE) $) {$(\sk,\pk)\getsr\HIBEgen$}; %place at the middle of DENIC and DE

    \node (ENC) at (2, -1){$(k,c)\getsr\HIBEenc(\pk,\text{''}\texttt{bob@chaac.tf.fau.de}\text{''})$};

    \node [below left = .5cm and .5cm of DE, anchor=north] (ROESLPA) {\texttt{.roeslpa.de}};
    \node [below right =.5cm and .5cm of DE, anchor=north] (FAU) {\texttt{.fau.de}};

    \node [below left = .5cm and .5cm of FAU, anchor=north] (MED) {\texttt{.med.fau.de}};
    \node [below right = .5cm and .5cm of FAU, anchor=north] (TF) {\texttt{.tf.fau.de}};

    \node [below = of TF, anchor=north] (BOB) {Bob: ''\texttt{bob@chaac.tf.fau.de}''};

    \node [below = 3.75cm of KEYGEN] (DEC) {$k\gets\HIBEdec(\sk'''',c)$};

    %ARROWS
    %DE
    \draw [->] (KEYGEN) -- ({0,0} |- KEYGEN) node [midway, above] {$\pk$}; %draw from keygen to (0,keygen.y) using intersection
    \draw (DE) -- (ROESLPA);
    \draw (DE) -- (FAU) node [midway, right] {$\sk'\gets\HIBEdel(\sk,\text{''fau''})$};
    
    %FAU
    \draw (FAU) -- (MED);
    \draw (FAU) -- (TF) node [midway, right] {$\sk''\gets\HIBEdel(\sk',\text{''tf''})$};

    %TF
    \draw[densely dotted] (TF) -- (BOB) node [midway, right] {$\sk''''\gets\HIBEdel(\sk''',\text{''bob''})$};

    \draw[->] (0,-1.5) -- (DEC) node [midway, above] {$c$};
\end{tikzpicture}
    \caption{Using $\HIBE$ for domain based key infrastructure.}
    \label{fig:hibe:example}
\end{figure}

\paragraph{Syntax} Concretely, a $\HIBE$ scheme is defined as follows:
\begin{itemize}
    \item $\HIBEgen : \emptyset \tor \sksp\times\pksp$
    \item $\HIBEenc: \pksp\times\{0,1\}^{l\cdot\lambda} \tor \csp\times\ksp$
    \item $\HIBEdec: \sksp\times\csp \to \ksp$
    \item $\HIBEdel: \sksp\times\{0,1\}^\lambda \to \sksp$
\end{itemize}

\paragraph{Correctness} A $\HIBE$ scheme is correct if the secret key $\sk$ delegated to string $\mathit{id}=\mathit{id}_1||\dots||\mathit{id}_l$ can decapsulate the ciphertext $c$ in $(k,c) \gets \HIBEenc(\pk,\mathit{id})$.
For brevity, we do not give a formal definition of correctness.
% \[\Pr[\HIBEdec(\sk, c)=k \mid \HIBEenc(\pk, \mathit{id})\tor(k,c), \HIBEgen\tor(\sk_0,\pk)]=1\]
% where there is a chain of delegations $\HIBEdel(\sk_0,\mathit{id_1})\to\sk_1,\dots,\HIBEdel(\sk_{l-1},\mathit{id_l})\to\sk$

\paragraph{Security} Once again, we only give an informal definition:
A $\HIBE$ scheme is secure if $k$ is indistinguishable from a random key, with $(k,c)\getsr\HIBEenc(\pk,\mathit{id})$, as long as none of the secret keys delegated to prefixes of $\mathit{id}$ are corrupted.

\subsubsection{Construction of a forward-secure KEM from HIBE}

An overview of how a construction of a forward-secure KEM from $\HIBE$ works is shown in Figure~\ref{fig:fkem:hibe:interaction}.
The $\mathit{nonce}$ is a random number with bit length $l$ and serves as the identity for the $\HIBE$ scheme. 
The $\mathit{nonce}$ should be unique for every encapsulation, i.e.\ it should be only used once.

\begin{figure}[!ht]
    \centering
    % !TeX root = ..\..\main.tex
\begin{tabular}{lcl}
    \underline{Alice} &  & \underline{Bob}\\
    $\FKEMenc(\pk):$ & \arrbox{2cm}{$\xlongleftarrow{\pk}$} & $\FKEMgen:$\\
    \qquad $\mathit{nonce}\getsr\{0,1\}^l$ & & \qquad $(\sk,\pk)\getsr\HIBEgen$\\
    \qquad $\mathit{nonce} \gets b_0||\dots||b_{l-1}$ & &\qquad Return $(\sk,\pk)$\\
    \qquad $(k,c)\getsr\HIBEenc(\pk,\mathit{nonce})$  & & \\
    \qquad $c \gets (\mathit{nonce}, c)$ & & \\
    & $\xlongrightarrow{c}$ & $(\sk, k)\gets\FKEMdec(\sk_{\mathit{nonce}},c)$
\end{tabular}
    \caption{Overview of the interaction between Alice and Bob in a $\FKEM$ exchange.}
    \label{fig:fkem:hibe:interaction}
\end{figure}

\paragraph{Delegation} In Figure~\ref{fig:fkem:hibe:interaction}, the delegation of the secret key $\sk_{\mathit{nonce}}$ from $\sk$ still needs to be addressed.
As an example, let $\mathit{nonce}=01\dots10$. The delegation is done in 5 steps:
\begin{enumerate}
    \item Delegate along path of $\mathit{nonce}$~(Figure~\ref{fig:fkem:hibe:delegation1})\\
        This gives the secret key $\sk_{01\dots10}$.
    \item Decapsulate $c$ with resulting secret key $\sk_{\mathit{nonce}}$
    \item Delegate co-path of $\mathit{nonce}$~(Figure~\ref{fig:fkem:hibe:delegation3})\\
        The co-path contains all siblings of nodes on the original path.
    \item Remove secret keys on the path of $\mathit{nonce}$~(Figure~\ref{fig:fkem:hibe:delegation4})\\
        The keys $(\sk_\epsilon,\sk_0,\sk_{01},\sk_{01\dots1},\sk_{01\dots10})$ are removed.
    \item Remember only secret keys on co-path\\
        The new secret keys are $(\sk_1,\sk_{00},\sk_{01\dots11})$ and those along the co-path of $\sk_{01}$ and $\sk_{01\dots1}$.
\end{enumerate}

It should be noted, that the number of secret keys grows logarithmically in the number of decryptions, which makes these $\HIBE$ constructions inefficient. 

\begin{figure}[!ht]
    \centering
    \begin{subfigure}{.33\textwidth}
        \centering
        % !TeX root = ..\..\main.tex
\begin{tikzpicture}[
    level 1/.style = {level distance = .75cm, sibling distance = 2cm},
    level 2/.style = {level distance = .75cm, sibling distance = 1.25cm}]
    \node at (0,0) {$\sk_\epsilon$}
        child  {node {$\sk_{0}$}                            
            child [missing]                    
            child {node {$\sk_{01}$}
                child[dotted] {node {$\sk_{01\dots1}$}
                    child[solid] {node {$\sk_{01\dots10}$}}
                    child[missing]}}}
        child  [missing];
\end{tikzpicture}
        \caption{Delegating $\sk_\mathit{nonce}$}
        \label{fig:fkem:hibe:delegation1}
    \end{subfigure}\hfill
    \begin{subfigure}{.33\textwidth}
        \centering
        % !TeX root = ..\..\main.tex
\begin{tikzpicture}[
    level 1/.style = {level distance = .75cm, sibling distance = 2cm},
    level 2/.style = {level distance = .75cm, sibling distance = 1.25cm}]
    \node at (0,0) {$\sk_\epsilon$}
        child  {node {$\sk_{0}$}                            
            child {node[orange] {$\sk_{00}$}}                       
            child {node {$\sk_{01}$}
                child[dotted] {node {$\sk_{01\dots1}$}
                    child[solid] {node {$\sk_{01\dots10}$}}
                    child[solid] {node[orange] {$\sk_{01\dots11}$}}}}}
        child  {node[orange] {$\sk_{1}$}};
\end{tikzpicture}
        \caption{Delegating co-path}
        \label{fig:fkem:hibe:delegation3}
    \end{subfigure}\hfill
    \begin{subfigure}{.33\textwidth}
        \centering
        % !TeX root = ..\..\main.tex
\begin{tikzpicture}[
    level 1/.style = {level distance = .75cm, sibling distance = 2cm},
    level 2/.style = {level distance = .75cm, sibling distance = 1.25cm}]
    \node[cross out, draw, inner sep=0pt, red, text=black] at (0,0) {$\sk_\epsilon$}
        child  {node[cross out, draw, inner sep=0pt, red, text=black] {$\sk_{0}$}                            
            child {node {$\sk_{00}$}}                       
            child {node[cross out, draw, inner sep=0pt, red, text=black] {$\sk_{01}$}
                child[dotted] {node[solid, cross out, draw, inner sep=0pt, red, text=black] {$\sk_{01\dots1}$}
                    child[solid] {node[cross out, draw, inner sep=0pt, red, text=black] {$\sk_{01\dots10}$}}
                    child[solid] {node {$\sk_{01\dots11}$}}}}}
        child  {node {$\sk_{1}$}
            child {node {$\sk_{10}$}}
            child {node {$\sk_{11}$}}};
\end{tikzpicture}
        \caption{Deleting path}
        \label{fig:fkem:hibe:delegation4}
    \end{subfigure}
    \caption{Delegation of the secret key $\sk_{\mathit{nonce}}$}
    \label{fig:fkem:hibe:delegation}
\end{figure}

\paragraph{Example} We provide a short example of how a $\FKEM_\HIBE$ can be used to decapsulate two messages.

Let Bob receive two messages $c_1=(c_1',``10")$ and $c_2=(c_2',``00")$ in that order.
For this, he sets the $\FKEM$~$\sk$ to the $\sk_\epsilon$ generated by the $\HIBEgen$ algorithm.
He then delegates $\sk_\epsilon$ to $\sk_{10}$ using the five steps described above:
\begin{enumerate}
    \item Delegate along path of $\mathit{nonce}_1=10$\\
        This results in the tree shown in Figure~\ref{fig:fkem:hibe:example1-1} and the secret key $\sk_{10}$.
    \item Decapsulate $c_1$ with resulting secret key $\sk_{10}$\\
        This gives the encapsulated key $k_1$.
    \item Delegate co-path of $10$\\
        This results in the tree shown in Figure~\ref{fig:fkem:hibe:example1-3} and the secret keys $\sk_{00}$.
    \item Remove the secret keys on the path of $10$\\
        The secret keys $\sk_\epsilon$, $\sk_0$, $\sk_{10}$ are removed (Figure~\ref{fig:fkem:hibe:example1-4}).
    \item The new secret keys are $\sk_0$, and $\sk_{11}$.
\end{enumerate}

If Bob wants to decapsulate $c_2$ now, he uses the secret key that corresponds to the longest prefix of $\mathit{nonce}$.
Here, this is $\sk_0$.
He then repeats the five steps for decapsulation, which yield the encapsulated key $k_2$ and the new secret keys $(\sk_{01},\sk_{11})$ (figures~\ref{fig:fkem:hibe:example2-1}-\ref{fig:fkem:hibe:example2-4}).

\begin{figure}[!ht]
    \centering
    \begin{subfigure}{.33\textwidth}
        \centering
        % !TeX root = ..\..\main.tex
\begin{tikzpicture}[
    level 1/.style = {level distance = .75cm, sibling distance = 2cm},
    level 2/.style = {level distance = .75cm, sibling distance = 1.25cm}]
    \node at (0,0) {$\sk_\epsilon$}
        child  [missing]              
        child  {node {$\sk_1$}
            child {node {$\sk_{10}$}}
        child [missing]};
\end{tikzpicture}
        \caption{Delegating $\sk_{10}$}
        \label{fig:fkem:hibe:example1-1}
    \end{subfigure}\hfill
    \begin{subfigure}{.33\textwidth}
        \centering
        % !TeX root = ..\..\main.tex
\begin{tikzpicture}[
    level 1/.style = {level distance = .75cm, sibling distance = 2cm},
    level 2/.style = {level distance = .75cm, sibling distance = 1.25cm}]
    \node at (0,0) {$\sk_\epsilon$}
        child  {node[orange] {$\sk_0$}}              
        child  {node {$\sk_1$}
            child {node {$\sk_{10}$}}
        child {node[orange] {$\sk_{11}$}}};
\end{tikzpicture}
        \caption{Delegating co-path}
        \label{fig:fkem:hibe:example1-3}
    \end{subfigure}
    \begin{subfigure}{.33\textwidth}
        \centering
        % !TeX root = ..\..\main.tex
\begin{tikzpicture}[
    level 1/.style = {level distance = .75cm, sibling distance = 2cm},
    level 2/.style = {level distance = .75cm, sibling distance = 1.25cm}]
    \node[cross out, draw, inner sep=0pt, red, text=black] at (0,0) {$\sk_\epsilon$}
        child  {node {$\sk_0$}}              
        child  {node[cross out, draw, inner sep=0pt, red, text=black] {$\sk_1$}
            child {node[cross out, draw, inner sep=0pt, red, text=black] {$\sk_{10}$}}
        child {node {$\sk_{11}$}}};
\end{tikzpicture}
        \caption{Removing used secret keys}
        \label{fig:fkem:hibe:example1-4}
    \end{subfigure}
    %
    \begin{subfigure}{.33\textwidth}
        \centering
        % !TeX root = ..\..\main.tex
\begin{tikzpicture}[
    level 1/.style = {level distance = .75cm, sibling distance = 2cm},
    level 2/.style = {level distance = .75cm, sibling distance = 1.25cm}]
    \node at (0,0) {$\sk_0$}
        child  {node {$\sk_{00}$}}              
        child  [missing];
\end{tikzpicture}
        \caption{Delegating $\sk_{00}$}
        \label{fig:fkem:hibe:example2-1}
    \end{subfigure}\hfill
    \begin{subfigure}{.33\textwidth}
        \centering
        % !TeX root = ..\..\main.tex
\begin{tikzpicture}[
    level 1/.style = {level distance = .75cm, sibling distance = 2cm},
    level 2/.style = {level distance = .75cm, sibling distance = 1.25cm}]
    \node at (0,0) {$\sk_0$}
        child  {node {$\sk_{00}$}}              
        child  {node[orange] {$\sk_{01}$}};
\end{tikzpicture}
        \caption{Delegating co-path}
        \label{fig:fkem:hibe:example2-3}
    \end{subfigure}
    \begin{subfigure}{.33\textwidth}
        \centering
        % !TeX root = ..\..\main.tex
\begin{tikzpicture}[
    level 1/.style = {level distance = .75cm, sibling distance = 2cm},
    level 2/.style = {level distance = .75cm, sibling distance = 1.25cm}]
    \node[cross out, draw, inner sep=0pt, red, text=black] at (0,0) {$\sk_0$}
        child  {node[cross out, draw, inner sep=0pt, red, text=black] {$\sk_{00}$}}              
        child  {node {$\sk_{01}$}};
\end{tikzpicture}
        \caption{Removing used secret keys}
        \label{fig:fkem:hibe:example2-4}
    \end{subfigure}
    \caption{Delegation for $c_1$~(\subref{fig:fkem:hibe:example1-1}-\subref{fig:fkem:hibe:example1-4}) and $c_2$~(\subref{fig:fkem:hibe:example2-1}-\subref{fig:fkem:hibe:example2-4})}
\end{figure}

\section{Authenticated Key Exchange}
\label{sec:AKE}

In practice, our key exchange protocol should provide authentication.
This means that after the key exchange, both parties should be sure that their communication partner is indeed the intended party and that no other party has knowledge of their shared symmetric key.

\paragraph{Key Exchange with Interaction} Consider the \emph{key exchange with interaction}~(IKE) protocol from the exercise. 
As a reminder, the protocol (here called two-pass key exchange) is defined as follows:

A two-pass key exchange $\TKE=(\TKEsnd,\TKErsp,\TKErcv)$ is a tuple of three algorithms with state space~$\stsp$, initial ciphertext space~$\csp_I$, response ciphertext space~$\csp_R$, and symmetric key space~$\ksp$:

\begin{itemize}
    \item $\TKEsnd: \emptyset \tor \stsp\times\csp_I$
    \item $\TKErsp: \csp_I \tor \ksp\times\csp_R$
    \item $\TKErcv: \stsp\times\csp_R \to \ksp$
\end{itemize}

A normal protocol execution can be seen in Figure~\ref{fig:tke:overview}.

The problem with a two-pass key exchange is that it is vulnerable to a \emph{Man-in-the-Middle} attack, where the adversary $\advA$ intercepts the messages between Alice and Bob and replaces them with its own messages (and therefore its own keys)~(Figure~\ref{fig:tke:MitM}).

\begin{figure}[!ht]
    \centering
    \begin{subfigure}{.49\textwidth}
        \centering
        % !TeX root = ..\..\main.tex
\begin{tabular}{cC{1cm}c}
    \textbf{Alice} &  & \textbf{Bob}\\
    & $\xlongrightarrow{c_A}$ & \\
    $\downarrow$& $\xlongleftarrow{c_B}$ & $\downarrow$\\
    $k$ & & $k$
\end{tabular}
        \caption{$\TKE$ protocol execution.}
        \label{fig:tke:overview}
    \end{subfigure}\hfill
    \begin{subfigure}{.49\textwidth}
        \centering
        % !TeX root = ..\..\main.tex
\begin{tabular}{cC{1cm}r|lC{1cm}c}
    \textbf{Alice} &  & \multicolumn{2}{c}{$\advA$} & & \textbf{Bob}\\
    & $\xlongrightarrow{c_A}$ &  &  & $\xlongrightarrow{c_A'}$ & \\
    $\downarrow$ & $\xlongleftarrow{c_B'}$ & & & $\xlongleftarrow{c_B}$ & $\downarrow$\\
    $k_1$ & & $k_1$ & $k_2$ & & $k_2$ \\
\end{tabular}
        \caption{$\TKE$ Man-in-the-Middle attack.}
        \label{fig:tke:MitM}
    \end{subfigure}
    \caption{Two-pass key exchange protocol.}
\end{figure}

\paragraph{Key Exchange with Authentication} To prevent this and to provide actual authentication, we use \emph{Authenticated Key Exchange}~(AKE) protocols.
An authenticated key exchange $\AKE=(\AKEgen,\allowbreak\AKEsnd,\allowbreak\AKErsp_R,\allowbreak\AKErsp_I,\allowbreak\AKErcv)$ is a tuple of five algorithms with secret key space~$\sksp$, public key space~$\pksp$, state space~$\stsp$, ciphertext space~$\csp$, and symmetric key space~$\ksp$:
\begin{itemize}
    \item $\AKEgen: \emptyset \tor \sksp\times\pksp$
    \item $\AKEsnd: \sksp\times\pksp \tor \stsp\times\csp$
    \item $\AKErsp_R: \sksp\times\csp \tor \stsp\times\csp$
    \item $\AKErsp_I: \stsp\times\csp \tor \ksp\times\csp$
    \item $\AKErcv: \sksp\times\stsp\times\csp \to \pksp\times\ksp$
\end{itemize}

An overview of normal protocol execution is given in Figure~\ref{fig:ake:overview}. 

\begin{figure}[!ht]
    \centering
    % !TeX root = ..\..\main.tex
\begin{tabular}{lC{3cm}l}
    \textbf{Alice} &  & \textbf{Bob}\\
    $(\sk_A, \pk_A)\getsr\AKEgen$ & $\xleftrightarrow{\pk_A,\pk_B}$ & $(\sk_B, \pk_B)\getsr\AKEgen$ \\
    $(\st, c_1)\getsr\AKEsnd(\sk_A, \pk_B)$ & $\xlongrightarrow{c_1}$ & \\
    & $\xlongleftarrow{c_2}$ & $(\st, c_2)\getsr\AKErsp_R(\sk_B, c_1)$\\
    $(k,c_3)\getsr\AKErsp_I(\st, c_2)$ & $\xlongrightarrow{c_3}$ & \\
    & & $(k, \pk_A)\gets\AKErcv(\st, c_3)$
\end{tabular}
    \caption{Authenticated key exchange protocol.}
    \label{fig:ake:overview}
\end{figure}

\paragraph{Correctness} An $\AKE$ protocol is correct if Alice and Bob compute the same key $k$ for every honest protocol execution.

\paragraph{Security} The adversary has access to the following oracles with the goal of using the $\Ochall$ oracle to output bit $b$.

\setlength\itemizeskip{3.5cm}
\begin{itemize}
    \item $\Ogen() \to \pk$:\hfill\makebox[\linewidth-\itemizeskip][l]{Generate new party}
    \item $\Osnd(i, s, \pk) \to c$:\hfill\makebox[\linewidth-\itemizeskip][l]{Lets instance $s$ of party $i$ start a session with party $\pk$}
    \item $\Orsp(i,s,c)\to c'$:\hfill\parbox[t][][l]{\linewidth-\itemizeskip}{Lets instance $s$ of party $i$ respond to $c$\\Also used for final receipt}
    \item $\Ocorrupt(i)\to\sk_i$:\hfill\makebox[\linewidth-\itemizeskip][l]{Corrupts party $i$'s long-term secret key}
    \item $\Oreveal(i,s)\to k_{i,s}$:\hfill\makebox[\linewidth-\itemizeskip][l]{Reveals final session key computed by instance $s$ of party $i$} 
    \item $\Ochall(i,s)\to k_{i,s,b}$:\hfill\makebox[\linewidth-\itemizeskip][l]{Either outputs real or random session key of instance $c$ of party $i$}
\end{itemize}

\paragraph{Trivial winning conditions} The trivial attacks are listed in Figure~\ref{fig:ake:trivial}. 
Note that the names of the trivial attacks give the general strategy and that the provided pseudocode can sometimes be changed (e.g., by switching the order of $\Ochall$ and $\Oreveal$ in the first attack).
The third attack in particular requires only that $\pk_j$ is the intended partner of $(i,s)$ and that $(i,s)$ has no session partners.

\begin{figure}[!ht]
    \centering
    % !TeX root = ..\..\main.tex
\begin{minipage}[t]{.5\textwidth}
    Attack 1: $\Ochall(i,s)$ and $\Oreveal(i,s)$:
    \begin{enumerate}
    \item Choose $(i,s)$ fittingly.
    \item $\Ochall(i,s)\to k_{i,s}$
    \item $\Oreveal(i,s)\to k_{i,s}'$
    \item Stop with $k_{i,s}\not= k_{i,s}'$
    \end{enumerate}
    Attack 2: $\Ochall(i,s)$ and $\Oreveal(j,t)$:
    \begin{enumerate}
    \item Choose $(i,s)$ fittingly.
    \item $\Ochall(i,s)\to k_{i,s}$
    \item Let $(j,t)$ be a \emph{session partner} of $(i,s)$.
    \item $\Oreveal(j,t)\to k_{j,t}$
    \item Stop with $k_{i,s}\not= k_{j,t}$
    \end{enumerate}
\end{minipage}\hfill
\begin{minipage}[t]{.5\textwidth}
    Attack 3: $\Ochall(i,s)$ and $\Ocorrupt(j)$:
    \begin{enumerate}
    \item $\Ogen\to\pk_i$
    \item $\Ogen\to\pk_j$
    \item $\Osnd(i,s,\pk_j)\to c_1$
    \item $\Ocorrupt(j)\to\sk_j$
    \item $\AKErsp_R(\sk_j,c_1)\tor(\st,c_2)$
    \item $\Orsp(i,s,c_2)\to c_3$
    \item $\AKErcv(\st, c_3)\to (k,\pk_i)$
    \item $\Ochall(i,s)\to k'$
    \item Stop with $k\not= k'$
    \end{enumerate}
\end{minipage}

    \caption{Trivial attacks on authenticated key exchange.}
    \label{fig:ake:trivial}
\end{figure}

\paragraph{Session partners} When talking about $\AKE$ it is helpful to formalize the concept of \emph{session partners}. 
In literature however, there are different definitions of session partners, which can be summarized as follows:
\begin{itemize}
    \item \emph{Matching Conversations} \cite{C:BelRog93}: ``Both instances saw the same transcript''
    \item \emph{Session identifiers}: ``Both instances saw the same relevant parts of the transcript''
    \item \emph{Original Key Partnering} \cite{EPRINT:LiSch17}: ``Using the same randomness, $(i,s)$ and $(j,t)$ would have computed the same key in an honest execution''
\end{itemize}
This lecture uses the \emph{Matching Conversations} definition when talking about session partners.

\paragraph{Constructions} We present two constructions of authenticated key exchange protocols. 
The first one (Figure~\ref{fig:ake:signed_dh}) is based on the Diffie-Hellman key exchange and uses signatures to provide integrity.

The second AKE construction (Figure~\ref{fig:ake:kem}) uses a KEM and a multi-key PRF (Figure~\ref{fig:prf:multi_key:ind}). 
In Figure~\ref{fig:ake:kem}, the secrets that can be learned by the adversary $\advA$ after corrupting Alice and Bob are colored orange and blue, respectively.
For example, after corrupting Bob and learning the secret key $\sk_B$, $\advA$ can decapsulate $c_1$ (which can be learned by eavesdropping on the communication) and obtain $k_1$.
The symmetric key $k$ will still remain secure, since $k_3$ will not be known to $\advA$, even if both Alice and Bob were corrupted.
The reason for this is the usage of fresh keys $\sk',\pk'$ for each new execution.

\begin{figure}[!ht]
    \centering
    % !TeX root = ..\..\main.tex
\begin{tabular}{lC{3cm}l}
    \textbf{Alice} &  & \textbf{Bob}\\
    $a\getsr\ZZ_p^*$ & $\xlongrightarrow{g^a}$ & $b\getsr\ZZ_p^*$ \\
     & $\xlongleftarrow{g^b,\sigma_B}$ & $\sigma_B\gets\SIGsig(\sk_B,(g^a, g^b))$ \\
     $\sigma_A\getsr\SIGsig(\sk_A,(g^a, g^b, \sigma_B))$ & $\xlongrightarrow{\sigma_A}$ & \\
\end{tabular}
    \caption{$\AKE$ based on Diffie-Hellman key exchange and signature scheme $\SIG$.}
    \label{fig:ake:signed_dh}
\end{figure}

\begin{figure}[!ht]
    \centering
    % !TeX root = ..\..\main.tex
\begin{tabular}{lC{3cm}l}
    \textbf{Alice}$(\sk_A, \pk_A)$ &  & \textbf{Bob}$(\sk_B, \pk_B)$\\
    $(k_1,c_1)\getsr\KEMenc(\pk_B)$ & &\\
    $(\sk',\pk')\getsr\AKEgen$ & $\xrightrightarrow{c_1,\pk'}$ & $k_1\gets\KEMdec(sk_B,c_1)$ \\
    & & $(k_2,c_2)\getsr\KEMenc(\pk_A)$\\
    & & $(k_3,c_3)\getsr\KEMenc(\pk')$\\
    & $\xlongleftarrow{c_2,c_3,\tau_B}$ & $(\tau_B, k_B)\gets\PRFf(k_1,\mathit{tag}=(c_1,\pk',c_2,c_3))$\\
    $(k,\tau_A)\gets\PRFf(k_)
\end{tabular}
    \caption{$\AKE$ based on an $\INDCCA$ secure $\KEM$ and $\IND$ secure $\PRF$. Colored secrets are known to $\advA$ after corruption of \textcolor{orange}{Alice} and \textcolor{blue}{Bob}.}
    \label{fig:ake:kem}
\end{figure}

\subsection{Real-World Authenticated Key Exchange}
Even though there are many more real world examples, this script will focus on three examples:

\subsubsection{Transport Layer Security~(TLS)}
TLS key exchange is similar to the signed Diffie-Hellman construction in Figure~\ref{fig:ake:signed_dh}.
An overview of the protocol is given in Figure~\ref{fig:ake:tls}.

\begin{figure}[!ht]
    \centering
    % !TeX root = ..\..\main.tex
\begin{tabular}{lC{3cm}l}
    \textbf{Alice} &  & \textbf{Bob}\\
    $x\getsr\ZZ_p^*$ & $\xlongrightarrow{g^x}$ & $y\getsr\ZZ_p^*$ \\
    & $\xlongleftarrow{g^y,\sigma}$ & $\sigma\gets\SIGsig(\sk_B,(g^x, g^y))$ \\
    & $\xlongrightarrow[\text{with }g^{xy}]{\text{Key Confirmation}}$ & \\
    \parbox[b][][t]{2.5cm}{\centering$\downarrow$\\Key derivation} &  & \parbox[b][][t]{2.5cm}{\centering$\downarrow$\\Key derivation}
\end{tabular}
    \caption{TLS key exchange.}
    \label{fig:ake:tls}
\end{figure}

\subsubsection{Noise Framework}
The Noise Framework (\url{https://noiseprotocol.org/noise.html}) is a set of rules to combine Diffie-Hellman key exchange with Authenticated Encryption~(AE).
The framework is very flexible, especially in a pre-configure context.
It is used by WhatsApp for its client-to-server communication and by WireGuard among others.

One protocol defined in the Noise Framework is ``NK''. 
Following the naming convention of Noise protocols, the ``N'' signifies that the initiator has \textbf{n}o static key and the ``K'' signifies that the responder has a static key \textbf{k}nown to the initiator.

In the documentation of the Noise Framework, the protocols are described in a format similar to the one in Figure~\ref{fig:ake:noise:nk}.
Figure~\ref{fig:ake:noise:nk_overview} gives an overview of the protocol execution in the format used in this script.
The $s$ in the figure is the static key of the responder, which is transmitted to the initiator before the protocol begins (shown by the dots afterwards).
An $e$ is an \emph{ephemeral} key (short-lived key) that correspond to a DH public key-share (e.g. $g^x$).
If two letters are combined a combination of the two keys is formed, e.g. $es$ and $g^{x\sk_B}$.

The messages sent during and after the protocol execution offer different levels of security: 
The encryption of $m_1$ offers forward security for the sender but not for the responder, while authenticity is only guaranteed for the responder.
Anonymity is guaranteed for both parties, but for the responder this holds only as long as he is not corrupted.
For $m_i, i\geq 2$ the encryption offers forward security for both parties, while the other security guarantees remain unchanged.

\begin{figure}[!ht]
    \centering
    \begin{subfigure}{.33\textwidth}
        \centering
        % !TeX root = ..\..\main.tex
\begin{tabular}{cl}
    $\leftarrow$ & s\\
    \dots & \\
    $\rightarrow$ & e, es\\
    $\leftarrow$ & e, ee
\end{tabular}
        \caption{Specification of the ``NK'' protocol in the Noise Framework format.}
        \label{fig:ake:noise:nk}
    \end{subfigure}\hfill
    \begin{subfigure}{.66\textwidth}
        \centering
        % !TeX root = ..\..\main.tex
\begin{tabular}{lC{3cm}l}
    \textbf{Alice} & & \textbf{Bob} \\
    $x\getsr\ZZ_p^*$ & & \\
    $k_1\gets\PRFf(\pk_B^x)$ & $\xlongrightarrow{g^x,\SYEenc(k_1,m_1)}$ & $k_1\gets\PRFf((g^x)^{\sk_B})$\\
    & & $y\getsr\ZZ_p^*$ \\
    & & $k_2\gets\PRFf(k_1, g^{xy})$\\
    & $\xlongleftarrow{\substack{\SYEenc(k_1,g^y)\\\SYEenc(k_2,m_2)}}$ & \\
    & $\xleftrightarrow{\SYEenc(k_2,m_i)}$ &
\end{tabular}
        \caption{Overview of ``NK'' protocol execution.}
        \label{fig:ake:noise:nk_overview}
    \end{subfigure}
    \caption{Noise protocol ``NK''.}
    \label{fig:ake:noise}
\end{figure}

\subsubsection{X3DH}
X3DH is an extended and authenticated 2-pass key exchange. 
It is used in the Ratcheted Key Exchange protocol~(Section \ref{sec:rke}).

The X3DH protocol is designed for asynchronous settings, where two parties want to establish a shared secret key, but one of them might be offline.
For this, one party (here named Bob) regularly sends Diffie-Hellman public keys to the server.
His long term key~(LTK), a prekey and a set of one-time keys. 
The prekey and the one-time keys are renewed regularly, but the one-time keys are renewed more often.

If Alice wants to then establish a public key with Bob, she can fetch Bobs keys from the server.
She also generates an ephemeral Diffie-Hellman exponent, which is then combined with Bobs keys and used to derive the symmetric key through a PRF. (Figure~\ref{fig:ake:x3dh})

If all of Bobs secret keys were used, and he has not renewed them yet, the key $g^{ab}$ is left out.
This reduces the forward security of the protocol, since the prekey is renewed much less often.

The X3DH protocol guarantees strong forward security, authenticity and deniability (both parties can deny their participation).

\begin{figure}[!ht]
    \centering
    \input{figures/ake/ake_X3DH}
    \caption{X3DH key exchange.}
    \label{fig:ake:x3dh}
\end{figure}


\section{Ratcheted Key Exchange}
\label{sec:rke}

We have seen plenty forward secure key exchange protocols in the previous chapters, but there is another security guarantee that is desirable in practice:
Post-compromise security~(PCS), which is the ability to recover from a compromise.
We adapt Figure~\ref{fig:fs:sketch}, which showed the idea of forward security, to show the idea of a protocol that provides both forward and post compromise security (Figure~\ref{fig:rke:pcs_fs_overview}).

\begin{figure}[!ht]
    \centering
    % !TeX root = ..\..\main.tex
\begin{tikzpicture}%
    \draw [green, ->|] (0,0) -- (4.8,0) node [midway, above] {FS}; %FS arrow
    \node at (5,0) {\color{red}\LARGE\Lightning}; % Corruption
    \node at (5,-.5) {secrets compromised}; 
    \draw [green, |->|] (5.2,0) -- (10,0) node [midway, above] {PCS}; % PCS arrow
\end{tikzpicture}%
    \caption{A sketch of forward and post compromise security.}
    \label{fig:rke:pcs_fs_overview}
\end{figure}

A way to achieve post compromise security are the so-called Ratcheted Key Exchange protocols~(RKE).
In Figure~\ref{fig:rke:overview}, an abstract view on a two party RKE protocol is given.
After the initial key exchange, both parties regularly update their keys in a Ratcheted Key Exchange step.
Using the resulting key, a new message can be encrypted.

For the steps different cryptographic building blocks are used.
For example, X3DH might be used for the initial key exchange, the Signal Double Ratchet algorithm for the Ratcheted Key exchange step and AEAD for the encryption of the messages.

\begin{figure}[!ht]
    \centering
    % !TeX root = ..\..\main.tex
\begin{tikzpicture}
    %draw the cylinder and rotate it so that the "opening" is to the left
    \node[yshift=-2.5cm] (c) at (4.5, 0) [cylinder, shape border rotate=180, draw, minimum height=7cm, minimum width=5cm, anchor=shape center, name path=c1] {};
    
    %first step of ratcheted key exchange 
    \node[yshift=-.5cm] (1) at (c.north) {1. Initial Key Exchange};
    %draw both key derivations
    \node[yshift=-.5cm, xshift=-2.5cm] (1a1) at (1) {$\substack{\downarrow\\\scriptstyle k}$};
    \node[yshift=-.5cm, xshift= 2.5cm] (1a2) at (1) {$\substack{\downarrow\\\scriptstyle k}$};
    %connect the two keys with a dashed arrow (the anchors, e.g. 1a1.30, are "border anchors" that use an angle)
    \draw[<->, dashed] (1a1.30) -- (1a2.150);
    
    %second step
    \node[yshift=-1cm] (2) at (1) {2. Ratcheted Key Exchange};
    %draw refreshing of keys
    \node[yshift=-.5cm, xshift=-2.5cm] (2a1) at (2) {$k\circlearrowleft$};
    \node[yshift=-.5cm, xshift= 2.5cm] (2a2) at (2) {$\circlearrowleft k$};

    %third step
    \node[yshift=-1cm] (3) at (2) {3. Secure Channel};
    %first channel cylinder below the third step (anchor is "shape center", since the normal "center" uses the actual height of the 3d cylinder)
    \node[yshift=-.5cm, anchor=shape center] (3Channel1) at (3) [cylinder, shape border rotate=180, draw, minimum height=5.2cm, minimum width=.5cm] {};
    %draw refreshing keys
    \node[yshift=-.5cm, xshift=-2.5cm] (3a1) at (3Channel1) {$k\circlearrowleft$};
    \node[yshift=-.5cm, xshift= 2.5cm] (3a2) at (3Channel1) {$\circlearrowleft k$};
    %second channel cylinder (no anchor=shape center since we use the shape center anchor of the first cylinder for placement)
    \node[yshift=-1cm] (3Channel2) at (3Channel1) [cylinder, shape border rotate=180, draw, minimum height=5.2cm, minimum width=.5cm] {};

    %arrows from the keys in the second step to the first channel in the third step
    \draw [->] (2a1) -- (3Channel1.after top);
    \draw [->] (2a2) -- (3Channel1.before bottom);


    %ALICE and BOB
    \node (A) at (0, 0) {\textbf{Alice}};
    \node (B) at (9, 0) {\textbf{Bob}};

    %the two messages send from Alice
    \node (A_m1) at (0, -3) {$m$};
    \node (A_m2) at (0, -4) {$m$};

    %draw the arrow connecting the first message and the first channel
    %this path is used to get the intersection
    \path[name path=A_m1path] (A_m1) -- (3Channel1.top);
    %calculate the intersections (they are named A_m1_intersect-1, A_m1_intersect-2, ...) and draw the black part of the arrow
    \draw[-,name intersections={of=c1 and A_m1path, name=A_m1_intersect}] (A_m1) -- (A_m1_intersect-1);
    %draw the gray part of the arrow
    \draw[->, gray] (A_m1_intersect-1) -- (3Channel1.top);

    %draw the arrow connecting the second message and the second channel
    \path[name path=A_m2path] (A_m2) -- (3Channel2.top);
    \draw[<-,name intersections={of=c1 and A_m2path, name=A_m2_intersect}] (A_m2) -- (A_m2_intersect-1);
    \draw[-, gray] (A_m2_intersect-1) -- (3Channel2.top);

    %the messages bob receives
    \node (B_m1) at (9, -3) {$m$};
    \node (B_m2) at (9, -4) {$m$};

    %arrows connecting the first channel to bobs first message
    \path[name path=B_m1path] (B_m1) -- (3Channel1.bottom);
    %the gray and black parts of the arrow are switched
    \draw[-,,gray, name intersections={of=c1 and B_m1path, name=B_m1_intersect}] (3Channel1.bottom) -- (B_m1_intersect-1);
    \draw[->] (B_m1_intersect-1) -- (B_m1);

    %arrows connecting the second channel to bobs second message
    \path[name path=B_m2path] (B_m2) -- (3Channel2.top);
    \draw[<-, gray, name intersections={of=c1 and B_m2path, name=B_m2_intersect}] (3Channel2.bottom) -- (B_m2_intersect-1);
    \draw[-] (B_m2_intersect-1) -- (B_m2);

    %Left side FS-arrows
    \draw[->|, green] (.5, -.5) -- (.5, -1.25) node [midway,left] {FS};
    \node at (.5, -1.5) {\color{red}\Large\Lightning}; % corruption arrow
    \draw[|->|, green] (.5, -1.75) -- (.5, -3.25);
    \draw[|->, green] (.5, -3.75) -- (.5, -5);

    %Right side FS-arrows
    \draw[->|, green] (8.5, -.5) -- (8.5, -1.25);
    \draw[|->|, green] (8.5, -1.75) -- (8.5, -3.25);
    \node at (8.5, -3.5) {\color{red}\Large\Lightning}; %corruption arrow
    \draw[|->, green] (8.5, -3.75) -- (8.5, -5) node [midway, right] {PCS};

\end{tikzpicture}
    \caption{Overview of the Ratcheted Key Exchange~(RKE) protocol.}
    \label{fig:rke:overview}
\end{figure}

\paragraph{Types of RKE} RKE protocols can be divided into three types, depending on the directions of communication they support.
Bidirectional RKE protocols is the intuitive case, where full bidirectional communication is supported~(Figure~\ref{fig:RKE:bi}).

Sesquidirectional RKE gives one party full functionality, while the other party can only send ``non-functional'' messages.
Consider for example the ciphertext $c_3$ in Figure~\ref{fig:RKE:sesq}:
It does not encapsulate a key, instead it should be used for efficiency or security (e.g. Bob can use this message to communicate fresh randomness to Alice to recover from earlier compromise).
Every Sesquidirectional RKE protocol can be transformed into a bidirectional RKE protocol, by using two instances of the Sesquidirectional RKE protocol.

\begin{figure}[!ht]
    \centering
    \begin{subfigure}{.5\textwidth}
        \centering
        % !TeX root = ..\..\main.tex

%%%%%%%%%%%%%
% Arrow are drawn using Tikz overlay and tikzmark, since arrows cross rows in the table.
% tikzmark: https://texdoc.org/serve/tikzmark/0
% using tikzmark requires two compilation passes (this should be fine since we already have to do LaTeX->BibTeX->LaTeX)
%%%%%%%%%%%%%
\begin{tabular}{rC{4cm}l}
    \textbf{Alice} &  & \textbf{Bob} \\
    %save the left and right positions for init arrow using tikzmarks
    %this arrow could also be drawn in the table, but then consistent spacing in relation to the arrows below is hard
    $\st_A$\tikzmark{rke_bi_init_l} & & \tikzmark{rke_bi_init_r}$\st_B$ \\
    $\downarrow$ &  & $\downarrow$ \\ 
    $k_1\gets\mathrm{snd}$\tikzmark{rke_bi_c1_l} &  & \tikzmark{rke_bi_c1_r}$\mathrm{rcv}\to k_1$ \\
    $\downarrow$ &  & $\downarrow$ \\  
    $k_2\gets\mathrm{snd}$\tikzmark{rke_bi_c2_l} &  & \tikzmark{rke_bi_c3_r}$\mathrm{snd}\to k_3$ \\
    $\downarrow$ &  & $\downarrow$ \\
    $k_3\gets\mathrm{rcv}$\tikzmark{rke_bi_c3_l} & & \tikzmark{rke_bi_c2_r}$\mathrm{rcv}\to k_2$ \\
    $\downarrow$ &  & $\downarrow$ \\
\end{tabular}
%draw the arrow
%tikzmark coordinates can be accessed with (pic cs:<name>)
%we shift the coordinates up so that the arrows are roughly in the middle of the line
\begin{tikzpicture}[remember picture,overlay]
    \draw[<->] ([xshift=.2cm, yshift=1mm]pic cs:rke_bi_init_l) -- ([xshift=-.2cm, yshift=1mm]pic cs:rke_bi_init_r) node[above, midway] {init};
    \draw[->] ([xshift=.2cm, yshift=1mm]pic cs:rke_bi_c1_l) -- ([xshift=-.2cm, yshift=1mm]pic cs:rke_bi_c1_r) node[above, midway] {$c_1$};
    %edge[out=0,in=180,->] curves the arrow
    %the node is placed using xshift, because using the more readable node[midway, near start] causes placement to break
    \draw ([xshift=.2cm, yshift=1mm]pic cs:rke_bi_c2_l) edge[out=0,in=180,->] ([xshift=-.2cm, yshift=1mm]pic cs:rke_bi_c2_r) node[above, xshift=.5cm] {$c_2$};
    \draw ([xshift=-.2cm, yshift=1mm]pic cs:rke_bi_c3_r) edge[out=180,in=0,->] ([xshift=.2cm, yshift=1mm]pic cs:rke_bi_c3_l) node[above, xshift=-.5cm] {$c_3$};
\end{tikzpicture}
        \caption{Bidirectional RKE protocol.}
        \label{fig:RKE:bi}
    \end{subfigure}\hfill
    \begin{subfigure}{.5\textwidth}
        \centering
        % !TeX root = ..\..\main.tex

%%%%%%%%%%%%%
% Arrow are drawn using Tikz overlay and tikzmark, since arrows cross rows in the table.
% tikzmark: https://texdoc.org/serve/tikzmark/0
% using tikzmark requires two compilation passes (this should be fine since we already have to do LaTeX->BibTeX->LaTeX)
%%%%%%%%%%%%%
\begin{tabular}{rC{4cm}l}
    \textbf{Alice} &  & \textbf{Bob} \\
    %save the left and right positions for init arrow using tikzmarks
    %this arrow could also be drawn in the table, but then consistent spacing in relation to the arrows below is hard
    $\st_A$\tikzmark{rke_sesq_init_l} & & \tikzmark{rke_sesq_init_r}$\st_B$ \\
    $\downarrow$ &  & $\downarrow$ \\ 
    $k_1\gets\mathrm{snd}$\tikzmark{rke_sesq_c1_l} &  & \tikzmark{rke_sesq_c1_r}$\mathrm{rcv}\to k_1$ \\
    $\downarrow$ &  & $\downarrow$ \\  
    $k_2\gets\mathrm{snd}$\tikzmark{rke_sesq_c2_l} &  & \tikzmark{rke_sesq_c3_r}$\mathrm{snd}$ \\
    $\downarrow$ &  & $\downarrow$ \\
    $\mathrm{rcv}$\tikzmark{rke_sesq_c3_l} & & \tikzmark{rke_sesq_c2_r}$\mathrm{rcv}\to k_2$ \\
    $\downarrow$ &  & $\downarrow$ \\
\end{tabular}
%draw the arrows
%tikzmark coordinates can be accessed with (pic cs:<name>)
%we shift the coordinates up so that the arrows are roughly in the middle of the line
\begin{tikzpicture}[remember picture,overlay]
    \draw[<->] ([xshift=.2cm, yshift=1mm]pic cs:rke_sesq_init_l) -- ([xshift=-.2cm, yshift=1mm]pic cs:rke_sesq_init_r) node[above, midway] {init};
    \draw[->] ([xshift=.2cm, yshift=1mm]pic cs:rke_sesq_c1_l) -- ([xshift=-.2cm, yshift=1mm]pic cs:rke_sesq_c1_r) node[above, midway] {$c_1$};
    %edge[out=0,in=180,->] curves the arrow
    %the node is placed using xshift, because using the more readable node[midway, near start] causes placement to break
    \draw ([xshift=.2cm, yshift=1mm]pic cs:rke_sesq_c2_l) edge[out=0,in=180,->] ([xshift=-.2cm, yshift=1mm]pic cs:rke_sesq_c2_r) node[above, xshift=.5cm] {$c_2$};
    \draw ([xshift=-.2cm, yshift=1mm]pic cs:rke_sesq_c3_r) edge[out=180,in=0,->] ([xshift=.2cm, yshift=1mm]pic cs:rke_sesq_c3_l) node[above, xshift=-.5cm] {$c_3$};
\end{tikzpicture}
        \caption{Sesquidirectional RKE protocol.}
        \label{fig:RKE:sesq}
    \end{subfigure}
    \caption{Interaction in bi- and sesquidirectional RKE protocols.}
\end{figure}

The third type of RKE protocols is unidirectional RKE, which only allows one party to send.
For simplicity, this is the type of RKE we will focus on in the following.

\subsection{Unidirectional RKE}

An overview of the unidirectional RKE protocol is given in Figure~\ref{fig:rke:uni}.
It should be noted that Bob is purely deterministic.
He can not distribute fresh randomness, and therefore can not recover from corruptions.

\begin{figure}[!ht]
    \centering
    % !TeX root = ..\..\main.tex
\begin{tabular}{rC{4cm}l}
    \textbf{Alice} &  & \textbf{Bob} \\
    $\st_A$ & $\xleftrightarrow{\makebox[3.5cm]{init}}$ & $\st_B$ \\
    $\downarrow$ &  & $\downarrow$ \\ 
    $k_1\gets\mathrm{snd}$ & $\xrightarrow{\mathmakebox[3.5cm]{c_1}}$ & $\mathrm{rcv}\to k_1$ \\
    $\downarrow$ &  & $\downarrow$ \\  
    $k_2\gets\mathrm{snd}$ & $\xleftarrow{\mathmakebox[3.5cm]{c_2}}$ & $\mathrm{rcv}\to k_2$ \\
    $\downarrow$ &  & $\downarrow$ \\
\end{tabular}
    \caption{Unidirectional RKE protocol.}
    \label{fig:rke:uni}
\end{figure}

\paragraph{Syntax} A unidirectional RKE protocol $\URKE=(\URKEinit, \URKEsnd, \URKErcv)$ has the following syntax:

\begin{itemize}
    \item $\URKEinit: \emptyset\tor\stsp_A\times\stsp_B$
    \item $\URKEsnd: \stsp_A\tor\stsp_A\times\ksp\times\csp$
    \item $\URKErcv: \stsp_B\times\csp\to\stsp_B\times\ksp$
\end{itemize}

\paragraph{Correctness} The correctness of $\URKE$ is defined as follows:
\begin{align*} %too long to fit on one line
    \Pr[k_A^i=k_B^i\mid &(\st_A^0, \st_B^0)\getsr\URKEinit, \\
                        &(\st_A^i, k_A^i, c_A^i)\getsr\URKEsnd(\st_A^{i-1}), \\
                        &(\st_B^i, k_B^i)\gets\URKErcv(\st_B^{i-1}, c_B^i)] = 1
\end{align*}

\paragraph{Security} The advantage of an adversary $\advA$ against the $\IND_\URKE^b$ game shown in Figure~\ref{fig:urke:ind} is defined as follows:
\[\Adv_\URKE^\ind(\advA) = \left|\Pr[\IND_\URKE^0(\advA)=1] - \Pr[\IND_\URKE^1(\advA)=1]\right|\]

\begin{figure}[!ht]
    \centering
    \nicoresetlinenr%
    \fbox{%
        \scalebox{\codescalefactor}{%
            %!TEX root=../../main.tex
\markersetlen{ndL}{130pt}%
\markersetlen{ndC}{155pt}%
\markersetlen{ndR}{150pt}%
\newcommand{\sync}{\mathit{sync}}%
\begin{tabular}[t]{lll}
    \nicodemusbox{\markerlenndL}{%
        \textbf{Game} $\IND_{\URKE}^b(\advA)$
        \begin{nicodemus}
            \item $s\gets0; r\gets0; \sync\gets1$
            \item $X_S\gets\emptyset; X_R\gets0$\quad{\color{gray}T3,T4}
            \item $C[\cdot]\gets\bot$
            \item $(\st_A, \st_B)\getsr\URKEinit$
            \item $b'\getsr\advA$
            \item Stop with~$b'$
        \end{nicodemus}%
        \medskip

        \textbf{Oracle} $\Oexps()$
        \begin{nicodemus}
            \item $X_S\gets X_S\cup\{s\}$\quad{\color{gray}T4}
            \item Return~$\st_A$
        \end{nicodemus}
    }%
    &
    \nicodemusbox{\markerlenndC}{%
        \textbf{Oracle} $\Osnd()$
        \begin{nicodemus}
            \item If $X_A=1$: Return~$(\bot, \bot)$\quad{\color{gray}T3}
            \item $(\st_A, k_0, c)\getsr\URKEsnd(\st_A)$
            \item $k_1\getsr\ksp$
            \item $C[s]\gets c$
            \item $s\gets s+1$
            \item Return~$(k_b,c)$
        \end{nicodemus}%
        \medskip

        \textbf{Oracle} $\Oexpr()$
        \begin{nicodemus}
            \item If $\sync=1\land s>r$:\quad{\color{gray}T2}
            \item \quad Return~$\bot$
            \item If $\sync=1\land s=r$:\quad{\color{gray}T3}
            \item \quad $X_R\gets1$
            \item Return~$\st_B$
        \end{nicodemus}
    }%
    &
    \nicodemusbox{\markerlenndR}{%
        \textbf{Oracle} $\Orcv(c)$
        \begin{nicodemus}
            \item $(\st_B, k_0)\gets\URKErcv(\st_B, c)$
            \item If $c\neq C[r] \land \sync=1$:
            \item \quad $\sync\gets0$
            \item If $r\not\in X_S\land k_0\neq\bot$:\quad{\color{gray}T4}
            \item \qquad $k_1\getsr\ksp$
            \item \qquad Return~$k_b$
            \item $r\gets r+1$
            \item If $\sync=1$:\quad{\color{gray}T1}
            \item \quad Return~$\bot$
            \item Stop with~$b'$
        \end{nicodemus}%
    }%
\end{tabular}%%
        }%
    }
    \caption{%
        Games $\IND$ for unidirectional RKE~$\URKE$. Lines accounting for trivial attacks T1-T4 are marked.
    }
    \label{fig:urke:ind}
\end{figure}

\paragraph{Trivial winning conditions} The four trivial attacks against the game $\IND_\URKE^b$ are shown in Figure~\ref{fig:urke:trivial}.

\begin{figure}[!ht]
    \centering
    \begin{minipage}{.5\textwidth}
        Attack 1: Challenge on both Sides:
        \begin{enumerate}[topsep=\smallskipamount]
            \item $\Osnd()\to(k,c)$
            \item $\Orcv(c)\to(k')$
            \item Stop with $(k\neq k')$
        \end{enumerate}
        \medskip
        Attack 2: Send + Receiver Exposure:
        \begin{enumerate}[topsep=\smallskipamount]
            \item $\Osnd()\to(k,c)$
            \item $\Oexpr()\to\st_R$
            \item $(st_R', k')\gets\URKErcv(st_R, c)$
            \item Stop with $(k\neq k')$
        \end{enumerate}
    \end{minipage}\hfill%
    \begin{minipage}{.5\textwidth}
        Attack 3: Receiver Exposure + Send:
        \begin{enumerate}[topsep=\smallskipamount]
            \item $\Oexpr()\to\st_R$
            \item $\Osnd()\to(k,c)$
            \item $(st_R', k')\gets\URKErcv(st_R, c)$
            \item Stop with $(k\neq k')$
        \end{enumerate}
        \medskip
        Attack 4: Sender Impersonation:
        \begin{enumerate}[topsep=\smallskipamount]
            \item $\Oexps()\to\st_S$
            \item $(st_S', k', c')\getsr\URKEsnd(st_S)$
            \item $\Orcv(c')\to(k)$
            \item Stop with $(k\neq k')$
        \end{enumerate}
    \end{minipage}
    \caption{Trivial winning conditions for game $\IND_\URKE^b$.}
    \label{fig:urke:trivial}
\end{figure}


\section{Key-Updatable Key Encapsulation Mechanism (KU-KEM)}

An overview of the interaction between Alice and Bob in a key-updatable key encapsulation mechanism ($\KUKEM$) is shown in Figure~\ref{fig:kukem:overview}. 

\begin{figure}[!ht]
    \centering
    % !TeX root = ..\..\main.tex
\begin{tabular}{cC{1cm}c}
    \textbf{Alice} &  & \textbf{Bob}\\
    & $\xlongleftarrow{\pk}$ & $(\sk, \pk)\getsr\KUKEMgen$\\
    $\pk\gets\KUKEMup(\pk, \ad)$ & & $\sk\gets\KUKEMup(\sk, \ad)$\\
    $(k, c)\getsr\KUKEMenc(\pk)$ & $\xlongrightarrow{c}$ & \\
    & & $k\gets\KUKEMdec(\sk, c)$\\ 
    $\pk\gets\KUKEMup(\pk, \ad)$ & & $\sk\gets\KUKEMup(\sk, \ad)$
\end{tabular}
    \caption{Overview of a key-updatable key encapsulation mechanism (KU-KEM).}
    \label{fig:kukem:overview}
\end{figure}

\paragraph{Syntax} An updatable KEM $\KUKEM=(\KUKEMgen,\KUKEMenc,\KUKEMdec, \KUKEMup)$ is a tuple of five algorithms ($\KUKEMup$ has two definitions) with secret key space~$\sksp$, public key space~$\pksp$, ciphertext space~$\csp$, symmetric key space~$\ksp$ and associated data space~$\adsp$:

\begin{itemize}
    \item $\KUKEMgen: \emptyset \tor \sksp\times\pksp$
    \item $\KUKEMenc: \pksp \tor \ksp\times\csp$
    \item $\KUKEMdec: \sksp\times\csp \to \dksp\times\ksp$
    \item $\KUKEMup: \pksp\times\adsp\to\pksp$
    \item $\KUKEMup: \sksp\times\adsp\to\sksp$
\end{itemize}

\paragraph{Correctness} A $\KUKEM$ construction is correct, if the following holds:
\begin{align*}
    \Pr[\KUKEMdec(\sk_n,c)=k\mid&(\sk_0,\pk_0)\getsr\KUKEMgen, \ad_i\getsr\adsp, \sk_i\gets\KUKEMup(\sk_{i-1},\ad_i),\\
     &\pk_i\gets\KUKEMup(\pk_{i-1},\ad_i), (k,c)\getsr\KUKEMenc(\pk_n)] = 1
\end{align*}

\paragraph{Security} We define the adversarial capabilities by specifying the oracles:

\begin{itemize}
    \item $\Oenc()\to(c,k_b)$
    \item $\OupPK(\ad)\to\pk$
    \item $\OupSK(\ad)\to\bot$
    \item $\Odec(c)\to\sfrac{k}{\bot}$, where $\bot$ is output when $c$ is a challenge.
    \item $\Ocorrupt\to\sk$
\end{itemize}  

The adversarial goal is to learn the challenge bit $b$. This models the confidentiality of the encapsulated keys.

\paragraph{Non-trivial attacks} In contrast to previous definitions, we do not give the trivial attacks here.
Instead, there are two security properties that should hold for a key-updatable KEM:
\begin{enumerate}
    \item 
    \begin{minipage}[t]{.5\linewidth}
        Previous keys $k$ should stay secure even if later secret keys $\sk$ are corrupted after intermediate updates.
    \end{minipage}\hfill
    \begin{minipage}[t][][b]{.5\linewidth}
        \centering
        \vspace{-.7\baselineskip} %the vertical alignment is not perfect, but skipping .7\baselineskip is pretty arbitrary
        % !TeX root = ..\..\main.tex
\begin{tabular}{cC{3cm}c}
    \textbf{Alice} &  & \textbf{Bob}\\
    $\pk$ & $\xlongrightarrow{\textcolor{green}{c}}$ & $\sk$\\
    & &$\downarrow$\\
    & &$\mathrm{up}(\ad)$\\
    & &$\downarrow$\\
    & &$\sk'$\makebox[0pt][l]{\quad\color{red}\Large\Lightning} %put it in zero width box, to prevent it from changing the centering alignment
\end{tabular}
    \end{minipage}
    \item \begin{minipage}[t]{.5\linewidth}
        Keys $k$ should stay secure even if incompatible secret keys are corrupted.
        Secret keys are incompatible if they are updated with associated data that differs from the public key updates.
    \end{minipage}\hfill
    \begin{minipage}[t][][b]{.5\linewidth}
        \centering
        \vspace{-.7\baselineskip}
        % !TeX root = ..\..\main.tex
\begin{tabular}{cC{3cm}c}
    \textbf{Alice} &  & \textbf{Bob}\\
    $\pk$ & & $\sk$\\
    $\downarrow$& &$\downarrow$\\
    $\mathrm{up}(\ad)$& $\ad\neq\ad^*$ &$\mathrm{up}(\ad^*)$\\
    $\downarrow$ & &$\downarrow$\\
    $\pk'$ & $\xlongrightarrow{\textcolor{green}{c}}$ &$\sk'$\makebox[0pt][l]{\quad\color{red}\Large\Lightning} %put it in zero width box, to prevent it from changing the centering alignment
\end{tabular}
    \end{minipage}
\end{enumerate}

\subsection{Construction} 
The idea of the construction is to use $\HIBE$~(Section~\ref{sec:hibe}) where $\KUKEMup(\pk,\ad)$ keeps a list of update strings and $\KUKEMup(\sk,\ad)$ extends the path in the $\HIBE$ identity hierarchy.
An outline of this idea is given in Figure~\ref{fig:kukem:construction:idea}.

\begin{figure}[!ht]
    \centering
    % !TeX root = ..\..\main.tex
\begin{tikzpicture}

    %code for the left update/pk-renewing operations
    \path let \p1=(-2,0) in %define point p1 --> \x1 is the x variable of p1, this allows for easier adjusting of textnodes x placement
        node[anchor=west] at (\x1, 0)    {$\pk_0=(\pk,\epsilon)$}
        node[anchor=west] at (\x1, -.5)  {$\mathrm{up}(\pk_0, \ad_1)$}
        node[anchor=west] at (\x1, -1)   {$\pk_1=(\pk,\ad_1)$}
        node[anchor=west] at (\x1, -1.5) {$\mathrm{up}(\pk_1, \ad_2)$}
        node[anchor=west] (lastTextNode) at (\x1, -2)   {$\pk_2=(\pk,\ad_1\|\ad_2)$}
        node[yshift=-.25cm] at (lastTextNode.south) {\vdots}; %place three vertical dots centered below the last text node

    %code for the right figure from the lecture (the delegation tree)
    \node at (4,0) {$\sk_0$}
        child [missing] %declare missing children --> otherwise the tree would be a straight line down
        child [missing]
        child[missing]
        child {node {$\sk_1$} 
            child [missing]
            child {node {$sk_2$} edge from parent node[right=5mm, midway] {$\mathrm{up}(\sk_1,\ad_2)$} %edge from parent node places a node at the connecting edge
                                                  node[left=1mm, midway] {$\mathrm{del}(\sk_0, \ad_2$)}} %we want two nodes per edge
            child [missing]
            child [missing] edge from parent node[right=9mm, midway] {$\mathrm{up}(\sk_0,\ad_1)$}
                                             node[left=1mm, midway] {$\mathrm{del}(\sk_0, \ad_2$)}}
        child [missing];
\end{tikzpicture}
    \caption{Basic Idea of constructing a $\KUKEM$ with $\HIBE$.}
    \label{fig:kukem:construction:idea}
\end{figure}

\paragraph{Observations} 
One problem with the construction is that the public key $\pk$ grows, although compression of $\pk$ is possible for some $\HIBE$ schemes.
Another problem is that the depth of the $\HIBE$ is linear in the number of updates. 
As a result, unbounded $\HIBE$ has to be used.

Balli et al.~\cite{AC:BalRosVau20} showed that any $\URKE$ scheme secure against weak randomness is equivalent to a $\KUKEM$ scheme.

At EuroCrypt 2023, Rösler et al.~\cite{EC:RosSlaStr23} presented two new constructions, showing that if the number of updates is bounded then $\KUKEM$ can be build from \emph{identity-based encryption}~(IBE).
If the number of updates is unbounded, then bounded-depth $\HIBE$ can be used. An example tree for the later case is given in Figure~\ref{fig:kukem:upibe}. \alert{JR: I don't quite understand how to interpret the tree. There should be some explanation here.}

\begin{figure}[!ht]
    \centering
    % !TeX root = ..\..\main.tex
\begin{tikzpicture}[
    % !changing the distances requires recalculating the positions p1 in the later draw commands (more information below)
    level 1/.style = {level distance = 1.5cm, sibling distance = 6cm},
    level 2/.style = {level distance = 1cm, sibling distance = 3cm},
    level 3/.style = {level distance = 1cm, sibling distance = 1.5cm},
    level 4/.style = {level distance = 1cm, sibling distance = .75cm},
    every child node/.style = {circle, draw, scale=.5},]

    % draw the nodes
    \node (n) {$\sk$} child foreach \x in {0,1}     %root
            {node {} child foreach \y in {0,1}      %level 1
                {node {} child foreach \z in {0,1}  %level 2
                    {node {} child                  %level 3
                        %draw the leaf nodes with a squiggly arrow
                        {node {} edge from parent[decorate, decoration={coil,aspect=0, amplitude=1pt, segment length=2pt}]}}}};
                    
    %draw the blue arrows (this can be thought of as being done in multiple parts):
    % 1. Draw the path from start node to end (this is the "(\p1) -- (n-X)--...--(n-X.south)" part)
    % 1. a) Since the arrow is shifted (step 4), the first line has to be shorter than the actual edge in the graph
    %       This is done by using "partway modifiers" (https://tikz.dev/tikz-coordinates#autosec-648). We name the start point with let \p1 =($(n-1.center)!<dist>!(n-1-2.center)$) in ...
    %       <dist> has to be calculated s.t. dist=sqrt( (xshift * relLineLength)^2 + (xshift^2) ), where
    %       relLineLength is level distance / (sibling distance / 2)
    %       xshift is the amount of shift in the x-direction (here: -2mm)
    %       e.g. if the line starts below a level 2 node, relLineLength=1/.75, dist=3.34mm
    % 2. Draw the end of the arrow that "bends" around the last node (this is done by adding a relative node with "-- +(.2, -.2)")
    % 3. Draw the rounded start of the arrow by goind .05 units to the left and .075 units down (with "($(\p1)+(-.05,-.075)$)" ) and then rounding the corner with "[rounded corners=1mm]--"
    % Shift everything by xshift with "transform canvas={xshift=-2mm}"

    %draw the three "inner" arrows (arrows 2,3 and 4 from the left) after the rules describe above
    \draw[blue, transform canvas={xshift=-2mm,}, ->] let \p1 =($(n-1-1.center)!3.34mm!(n-1-1-2.center)$) in
        ($(\p1)+(-.05,-.075)$) [rounded corners=1mm]-- (\p1) -- (n-1-1-2.center) -- (n-1-1-2-1.south) -- node[pos=1, below] {$2t$} +(.2, -.2);
    \draw[blue, transform canvas={xshift=-2mm}, ->] let \p1 =($(n-1.center)!2.41mm!(n-1-2.center)$) in
        ($(\p1)+(-.05,-.075)$) [rounded corners=1mm]-- (\p1) -- (n-1-2.center) -- (n-1-2-1.center) -- ( n-1-2-1-1.south) -- node[pos=1, below] {$3t$} +(.2, -.2);
    \draw[blue, transform canvas={xshift=-2mm}, ->] let \p1 =($(n-1-2.center)!3.34mm!(n-1-2-2.center)$) in
        ($(\p1)+(-.05,-.075)$) [rounded corners=1mm]-- (\p1) -- (n-1-2-2.center) -- ( n-1-2-2-1.south) -- node[pos=1, below] {$4t$} +(.2, -.2);
    
    %leftmost arrow; no bend at start
    %this is drawn in two parts: The first part is from the root to to beginning of the squiggly arrow, where a node "1" is placed
    %and the second part is from node 1 to the end of the squiggly arrow, where a node "t" is placed
    %when drawing node 1 we have to reverse the yshift of 1mm by placing a new node with "+(0, -.1)"
    \draw[blue, transform canvas={xshift=-2mm, yshift=1mm}, rounded corners=1mm, -|] (n.center) -- (n-1.center) -- (n-1-1.center) -- (n-1-1-1.center) -- node[pos=1, left] {$1$} +(0, -.1);
    \draw[blue, transform canvas={xshift=-2mm, yshift=1mm}, rounded corners=1mm, ->] ($(n-1-1-1.center)+(0,-.1)$) -- ([yshift=-2mm] n-1-1-1-1.south) -- node[pos=1, below] {$t$} +(.2, -.1);

    %draw the rightmost arrow
    %we place two nodes after the last segment: one node contains vertical dots and the other node contains the label $4t+x$
    \draw[blue, transform canvas={xshift=-2mm}] let \p1 =($(n.center)!4mm!(n-2.center)$) in
        ($(\p1)+(-.05,-.075)$) [rounded corners=1mm]-- (\p1) -- (n-2.center) -- (n-2-1.center) -- ( n-2-1-1.south) -- node[below] {\vdots} +(0, -.2) node[left] {$4t+x$};

    %draw the (l,t)-line to the right 
    %let p1 be the rightmost, lowermost node of the last level before the squiggly arrow shifted 1 unit to the right
    %let p2 be the rightmost node of the last level
    \draw[|-] let \p1=([xshift=1cm] n-2-2-2) in
        (\x1, 0) -- node[right, midway] {$l$} (\p1); %\x1 is x-coordinate of p1
    \draw[|-|] let \p1=($(n-2-2-2)+(1,0)$), \p2=(n-2-2-2-1) in
        (\p1) -- node[right, midway] {$t$} (\x1, \y2);

    \end{tikzpicture}

    \caption{Using bounded-depth HIBE to construct $\KUKEM$.}
    \label{fig:kukem:upibe}
\end{figure}

\section{Symmetric Primitives}


\subsection{Pseudo-Random Generator}


\paragraph{Syntax}
A pseudo-random generator $\PRG=\PRGf$ is an algorithm with key space~$\ksp$ and output space~$\rsp$:

\begin{itemize}
    \item $\PRGf: \ksp \to \rsp$
\end{itemize}

\paragraph{Security}
The advantage of an adversary~$\advA$ against pseudo-random generator $\PRG$ in game $\IND$ from Figure~\ref{fig:prg:ind} is defined as:
\[
\Adv_\PRG^\ind(\advA)\coloneqq\left|\Pr[\IND_{\PRG}^0(\advA)=1]-\Pr[\IND_{\PRG}^1(\advA)=1]\right|\text{.}
\]

\begin{figure}[!ht]
    \centering
    \nicoresetlinenr%
    \fbox{%
        \scalebox{\codescalefactor}{%
            %!TEX root=../main.tex
\markersetlen{ndL}{110pt}%
\newcommand{\CC}{\mathit{CC}}%
\begin{tabular}[t]{l}
    \nicodemusbox{\markerlenndL}{%
        \textbf{Game} $\IND_{\PRG}^b(\advA)$
        \begin{nicodemus}
            \item $k\getsr\ksp$
            \item $r_0\gets\PRGf(k)$
            \item $r_1\getsr\rsp$
            \item $b'\getsr\advA(r_b)$
            \item Stop with~$b'$
        \end{nicodemus}%
    }%
\end{tabular}%%
        }%
    }
    \caption{%
        Games $\IND$ for pseudo-random generator~$\PRG$.
    }
    \label{fig:prg:ind}
\end{figure}

\subsection{Pseudo-Random Function}

\paragraph{Syntax}
A pseudo-random function $\PRF=\PRFf$ is an algorithm with key space~$\ksp$, input space~$\msp$ and output space~$\rsp$:

\begin{itemize}
    \item $\PRFf: \ksp\times\msp \to \rsp$
\end{itemize}

\paragraph{Security}
The advantage of an adversary~$\advA$ against pseudo-random function $\PRF$ in game $\IND$ from Figure~\ref{fig:prf:ind} is defined as:
\[
\Adv_\PRF^\ind(\advA)\coloneqq\left|\Pr[\IND_{\PRF}^0(\advA)=1]-\Pr[\IND_{\PRF}^1(\advA)=1]\right|\text{.}
\]

\begin{figure}[!ht]
    \centering
    \nicoresetlinenr%
    \fbox{%
        \scalebox{\codescalefactor}{%
            %!TEX root=../../main.tex
\markersetlen{ndL}{110pt}%
\markersetlen{ndR}{120pt}%
\newcommand{\CC}{\mathit{CC}}%
\begin{tabular}[t]{ll}
    \nicodemusbox{\markerlenndL}{%
        \textbf{Game} $\IND_{\PRF}^b(\advA)$
        \begin{nicodemus}
            \item $R[\cdot]\gets\bot$
            \item $k\getsr\ksp$
            \item $b'\getsr\advA$
            \item Stop with~$b'$
        \end{nicodemus}%
    }%
    &
    \nicodemusbox{\markerlenndR}{%
        \textbf{Oracle} $\Ochall(m)$
        \begin{nicodemus}
            \item $r_0\gets\PRFf(k,m)$
            \item If $R[m]=\bot$: $R[m]\getsr\rsp$
            \item $r_1\gets R[m]$
            \item Return~$r_b$
        \end{nicodemus}%
    }%
\end{tabular}%%
        }%
    }
    \caption{%
        Games $\IND$ for pseudo-random function~$\PRF$.
    }
    \label{fig:prf:ind}
\end{figure}

\subsubsection{Multi-key PRF}
\paragraph{Syntax} A multi-key $\PRF=\PRFf$ with key space~$\ksp$, input space~$\msp$, and output space~$\rsp$ is an algorithm with the following syntax:

\begin{itemize}
    \item $\PRFf: \ksp^n\times\msp \to \rsp$
\end{itemize}

\paragraph{Security}
The advantage of an adversary~$\advA$ against pseudo-random function $\PRF$ in game $\IND_{F,n,i}$ from Figure~\ref{fig:prf:multi_key:ind} is defined as:
\[
\Adv_\PRF^\ind(\advA)\coloneqq\left|\Pr[\IND_{F,n,i}^0(\advA)=1]-\Pr[\IND_{F,n,i}^1(\advA)=1]\right|\text{.}
\]


\begin{figure}[!ht]
    \centering
    \nicoresetlinenr%
    \fbox{%
        \scalebox{\codescalefactor}{%
            %!TEX root=../../main.tex
\markersetlen{ndL}{110pt}%
\markersetlen{ndR}{130pt}%
\newcommand{\CC}{\mathit{CC}}%
\begin{tabular}[t]{ll}
    \nicodemusbox{\markerlenndL}{%
        \textbf{Game} $\IND_{F,n,i}^b(\advA)$
        \begin{nicodemus}
            \item $R[\cdot]\gets\bot$
            \item $k^i\getsr\ksp$
            \item $b'\getsr\advA$
            \item Stop with~$b'$
        \end{nicodemus}%
    }%
    &
    \nicodemusbox{\markerlenndR}{%
        \textbf{Oracle} $\Ochall((k^j)_{j\in[n]\setminus\{i\}},m)$
        \begin{nicodemus}
            \item $r_0\gets\PRFf(k^1,\dots,k^n,m)$
            \item If $K[(k^j)_{j\in[n]\setminus\{i\}},m]=\bot$:
            \item \quad $r_1\getsr\rsp$
            \item \quad $K[(k^j)_{j\in[n]\setminus\{i\}},m]\gets r_1$
            \item Else: $r_1\gets K[(k^j)_{j\in[n]\setminus\{i\}},m]$
            \item Return~$k_b$
        \end{nicodemus}%
    }%
\end{tabular}%%
        }%
    }
    \caption{%
        Games $\IND$ for multi-key pseudo-random function~$\PRF$.
    }
    \label{fig:prf:multi_key:ind}
\end{figure}

\subsection{Message Authentication Code (MAC)}

\paragraph{Syntax}
A message authentication code $\MAC=(\MACgen,\MACtag,\MACvfy)$ is a tuple of three algorithms with key space~$\ksp$, message space~$\msp$, and tag space~$\tsp$:

\begin{itemize}
    \item $\MACgen: \emptyset \tor \ksp$
    \item $\MACtag: \ksp\times\msp \tor \tsp$
    \item $\MACvfy: \ksp\times\msp\times\tsp \to \Bool$
\end{itemize}

\paragraph{Correctness}
A message authentication code $\MAC$ is correct if
\[
\Pr[\MACvfy(k,m,\MACtag(k,m))=\T\mid k\getsr\MACgen,m\getsr\msp]=1\text{.}
\]

\paragraph{Security: Strong Existential Unforgeability under Chosen-Message Attacks}
The advantage of an adversary~$\advA$ against message authentication code $\MAC$ in game $\SUFCMA$ from Figure~\ref{fig:mac:suf} is defined as:
\[
\Adv_\MAC^\sufcma(\advA)\coloneqq\Pr[\SUFCMA_{\MAC}(\advA)=1]\text{.}
\]

\begin{figure}[!ht]
    \centering
    \nicoresetlinenr%
    \fbox{%
        \scalebox{\codescalefactor}{%
            %!TEX root=../main.tex
\markersetlen{ndL}{110pt}%
\markersetlen{ndR}{120pt}%
\newcommand{\MT}{\mathit{MT}}%
\begin{tabular}[t]{ll}
    \nicodemusbox{\markerlenndL}{%
        \textbf{Game} $\SUFCMA_{\MAC}(\advA)$
        \begin{nicodemus}
            \item $\MT\gets\emptyset$
            \item $k\getsr\MACgen$
            \item Invoke $\advA$
            \item Stop with~$0$
        \end{nicodemus}%
        \medskip
        
        \textbf{Oracle} $\Otag(m)$
        \begin{nicodemus}
            \item $\mtag\getsr\MACtag(k,m)$
            \item $\MT\gets\MT\cup\{(m,\mtag)\}$
            \item Return $\mtag$
        \end{nicodemus}%
    }%
    &
    \nicodemusbox{\markerlenndR}{%
        \textbf{Oracle} $\Ovfy(m,\mtag)$
        \begin{nicodemus}
            \item $b\gets\MACvfy(k,m,\mtag)$
            \item If $(m,\mtag)\notin\MT\land b=\T$:
            \item \quad Stop with~$1$
            \item Return $b$
        \end{nicodemus}%
    }%
\end{tabular}%%
        }%
    }
    \caption{%
        Game $\SUFCMA$ for message authentication code~$\MAC$.
    }
    \label{fig:mac:suf}
\end{figure}


\subsection{Symmetric Encryption (SE)}

\subsubsection{Probabilistic SE}

\paragraph{Syntax}
A probabilistic symmetric encryption scheme $\SYE=(\SYEgen,\SYEenc,\SYEdec)$ is a tuple of three algorithms with key space~$\ksp$, message space~$\msp$, and ciphertext space~$\csp$:

\begin{itemize}
    \item $\SYEgen: \emptyset \tor \ksp$
    \item $\SYEenc: \ksp\times\msp \tor \csp$
    \item $\SYEdec: \ksp\times\csp \to \msp$
\end{itemize}

\paragraph{Correctness}
A symmetric encryption scheme $\SYE$ is correct if 
\[
\Pr[\SYEdec(k,\SYEenc(k,m))=m\mid k\getsr\SYEgen,m\getsr\msp]=1\text{.}
\]
Equivalently, a symmetric encryption scheme $\SYE$ is correct if $\Pr[\CORR_{\SYE}(\advA)=0]=1$ for all adversaries~$\advA$, where game~$\CORR$ is defined in Figure~\ref{fig:sym:enc:corr:prob}.

\begin{figure}[!ht]
    \centering
    \nicoresetlinenr%
    \fbox{%
        \scalebox{\codescalefactor}{%
            %!TEX root=../main.tex
\markersetlen{ndL}{100pt}%
\markersetlen{ndR}{100pt}%
\newcommand{\CM}{\mathit{CM}}%
\begin{tabular}[t]{ll}
    \nicodemusbox{\markerlenndL}{%
        \textbf{Game} $\CORR_{\SYE}(\advA)$
        \begin{nicodemus}
            \item $\CM[\cdot]\gets\bot$
            \item $k\getsr\SYEgen$
            \item Invoke $\advA$
            \item Stop with~$0$
        \end{nicodemus}%
        \medskip
        
        \textbf{Oracle} $\Oenc(m)$
        \begin{nicodemus}
            \item Require $m\in\msp$
            \item $c\getsr\SYEenc(k,m)$
            \item $\CM[c]\gets m$
            \item Return~$c$
        \end{nicodemus}%
    }%
    &
    \nicodemusbox{\markerlenndR}{%
        \textbf{Oracle} $\Odec(c)$
        \begin{nicodemus}
            \item $m'\gets\SYEdec(k,c)$
            \item If $\CM[c]\notin\{m',\bot\}$:
            \item \quad Stop with $1$
            \item Return $m'$
        \end{nicodemus}%
    }%
\end{tabular}%%
        }%
    }
    \caption{%
        Game $\CORR$ for probabilistic encryption scheme~$\SYE$.
    }
    \label{fig:sym:enc:corr:prob}
\end{figure}

\paragraph{Security: One-Wayness under Chosen-Ciphertext Attacks}
The advantage of an adversary~$\advA$ against symmetric encryption scheme $\SYE$ in game $\OWCCA$ from Figure~\ref{fig:sym:enc:ow:prob} is defined as:
\[
\Adv_\SYE^\owcca(\advA)\coloneqq\Pr[\OWCCA_{\SYE}(\advA)=1]\text{.}
\]

\begin{figure}[!ht]
    \centering
    \nicoresetlinenr%
    \fbox{%
        \scalebox{\codescalefactor}{%
            %!TEX root=../main.tex
\markersetlen{ndL}{100pt}%
\markersetlen{ndR}{130pt}%
\newcommand{\CM}{\mathit{CM}}%
\begin{tabular}[t]{ll}
    \nicodemusbox{\markerlenndL}{%
        \textbf{Game} $\OWCCA_{\SYE}(\advA)$
        \begin{nicodemus}
            \item $\CM\gets\emptyset$
            \item $k\getsr\SYEgen$
            \item $(c,m)\getsr\advA$
            \item If $(c,m)\in\CM$:
            \item \quad Stop with~$1$
            \item Stop with~$0$
        \end{nicodemus}%
        \medskip
        
        \textbf{Oracle} $\Oenc(m)$
        \begin{nicodemus}
            \item $c\getsr\SYEenc(k,m)$
            \item Return~$c$
        \end{nicodemus}%
    }%
    &
    \nicodemusbox{\markerlenndR}{%
        \textbf{Oracle} $\Ochall()$
        \begin{nicodemus}
            \item $m\getsr\msp$
            \item $c\getsr\SYEenc(k,m)$
            \item $\CM\gets\CM\cup\{(c,m)\}$
            \item Return~$c$
        \end{nicodemus}%
        \medskip

        \textbf{Oracle} $\Odec(c)$
        \begin{nicodemus}
            \item Require $\nexists m':(c,m')\in\CM$
            \item $m\gets\SYEdec(k,c)$
            \item Return $m$
        \end{nicodemus}%
    }%
\end{tabular}%%
        }%
    }
    \caption{%
        Game $\OWCCA$ for probabilistic symmetric encryption scheme~$\SYE$.
    }
    \label{fig:sym:enc:ow:prob}
\end{figure}

\paragraph{Security: Indistinguishability under Chosen-Ciphertext Attacks}
The advantage of an adversary~$\advA$ against symmetric encryption scheme $\SYE$ in game $\INDCCA$ from Figure~\ref{fig:sym:enc:ind:prob} is defined as:
\[
\Adv_\SYE^\indcca(\advA)\coloneqq\left|\Pr[\INDCCA_{\SYE}^0(\advA)=1]-\Pr[\INDCCA_{\SYE}^1(\advA)=1]\right|\text{.}
\]

\begin{figure}[!ht]
    \centering
    \nicoresetlinenr%
    \fbox{%
        \scalebox{\codescalefactor}{%
            %!TEX root=../main.tex
\markersetlen{ndL}{100pt}%
\markersetlen{ndR}{130pt}%
\newcommand{\CC}{\mathit{CC}}%
\begin{tabular}[t]{ll}
    \nicodemusbox{\markerlenndL}{%
        \textbf{Game} $\INDCCA_{\SYE}^b(\advA)$
        \begin{nicodemus}
            \item $\CC\gets\emptyset$
            \item $k\getsr\SYEgen$
            \item $b'\getsr\advA$
            \item Stop with~$b'$
        \end{nicodemus}%
        \medskip
        
        \textbf{Oracle} $\Oenc(m)$
        \begin{nicodemus}
            \item $c\getsr\SYEenc(k,m)$
            \item Return~$c$
        \end{nicodemus}%
    }%
    &
    \nicodemusbox{\markerlenndR}{%
        \textbf{Oracle} $\Ochall(m_0,m_1)$
        \begin{nicodemus}
            \item Require $\{m_0,m_1\}\subseteq\msp$
            \item $c\getsr\SYEenc(k,m_b)$
            \item $\CC\gets\CC\cup\{c\}$
            \item Return~$c$
        \end{nicodemus}%
        \medskip
        
        \textbf{Oracle} $\Odec(c)$
        \begin{nicodemus}
            \item Require $c\notin\CC$
            \item $m\gets\SYEdec(k,c)$
            \item Return $m$
        \end{nicodemus}%
    }%
\end{tabular}%%
        }%
    }
    \caption{%
        Games $\INDCCA$ for probabilistic symmetric encryption scheme~$\SYE$.
    }
    \label{fig:sym:enc:ind:prob}
\end{figure}


\subsubsection{Authenticated Encryption with Associated Data}

\paragraph{Syntax}
An authenticated encryption scheme with associated data $\AEAD=(\AEADgen,\AEADenc,\AEADdec)$ is a tuple of three algorithms with key space~$\ksp$, message space~$\msp$, associated-data space~$\adsp$, and ciphertext space~$\csp$:

\begin{itemize}
    \item $\AEADgen: \emptyset \tor \ksp$
    \item $\AEADenc: \ksp\times\msp\times\adsp \tor \csp$
    \item $\AEADdec: \ksp\times\adsp\times\csp \to \msp\cup\{\bot\}$
\end{itemize}

\paragraph{Correctness}
An authenticated encryption scheme with associated data $\AEAD$ is correct if 
\[
\Pr[\AEADdec(k,\ad,\AEADenc(k,m,\ad))=m\mid k\getsr\SYEgen,m\getsr\msp,\ad\getsr\adsp]=1\text{.}
\]
Equivalently, an authenticated encryption scheme with associated data $\AEAD$ is correct if $\Pr[\CORR_{\AEAD}(\advA)=0]=1$ for all adversaries~$\advA$, where game~$\CORR$ is defined in Figure~\ref{fig:sym:aenc:corr}.

We define the standard security notions $\INDD$ and $\SUFCMA$ adapted from the seminal work by Rogaway~\cite{CCS:Rogaway02} in the following paragraphs.

\begin{figure}[!ht]
    \centering
    \nicoresetlinenr%
    \fbox{%
        \scalebox{\codescalefactor}{%
            %!TEX root=../main.tex
\markersetlen{ndL}{120pt}%
\markersetlen{ndR}{120pt}%
\newcommand{\CM}{\mathit{CM}}%
\begin{tabular}[t]{ll}
    \nicodemusbox{\markerlenndL}{%
        \textbf{Game} $\CORR_{\AEAD}(\advA)$
        \begin{nicodemus}
            \item $\CM[\cdot]\gets\bot$
            \item $k\getsr\AEADgen$
            \item Invoke $\advA$
            \item Stop with~$0$
        \end{nicodemus}%
        \medskip
        
        \textbf{Oracle} $\Oenc(m,\ad)$
        \begin{nicodemus}
            \item Require $m\in\msp$
            \item $c\getsr\AEADenc(k,m,\ad)$
            \item $\CM[\ad,c]\gets m$
            \item Return~$c$
        \end{nicodemus}%
    }%
    &
    \nicodemusbox{\markerlenndR}{%
        \textbf{Oracle} $\Odec(c,\ad)$
        \begin{nicodemus}
            \item $m'\gets\AEADdec(k,\ad,c)$
            \item If $\CM[\ad,c]\notin\{m',\bot\}$:
            \item \quad Stop with $1$
            \item Return $m'$
        \end{nicodemus}%
    }%
\end{tabular}%%
        }%
    }
    \caption{%
        Game $\CORR$ for authenticated encryption scheme with associated data~$\AEAD$.
    }
    \label{fig:sym:aenc:corr}
\end{figure}

\paragraph{Security: Indistinguishability of Ciphertexts from Randomness under Chosen-Plaintext Attacks}
The advantage of an adversary~$\advA$ against authenticated encryption scheme with associated data $\AEAD$ in game $\INDD$ from Figure~\ref{fig:sym:aenc:indd} is defined as:
\[
\Adv_\AEAD^\indd(\advA)\coloneqq\left|\Pr[\INDD_{\AEAD}^0(\advA)=1]-\Pr[\INDD_{\AEAD}^1(\advA)=1]\right|\text{.}
\]

\begin{figure}[!ht]
    \centering
    \nicoresetlinenr%
    \fbox{%
        \scalebox{\codescalefactor}{%
            %!TEX root=../main.tex
\markersetlen{ndL}{120pt}%
\markersetlen{ndR}{120pt}%
\begin{tabular}[t]{ll}
    \nicodemusbox{\markerlenndL}{%
        \textbf{Game} $\INDD_{\AEAD}^b(\advA)$
        \begin{nicodemus}
            \item $k\getsr\AEADgen$
            \item $b'\getsr\advA$
            \item Stop with~$b'$
        \end{nicodemus}%
        \medskip
        
        \textbf{Oracle} $\Oenc(m,\ad)$
        \begin{nicodemus}
            \item $c\getsr\AEADenc(k,m,\ad)$
            \item Return~$c$
        \end{nicodemus}%
    }%
    &
    \nicodemusbox{\markerlenndR}{%
        \textbf{Oracle} $\Ochall(m,\ad)$
        \begin{nicodemus}
            \item Require $m\in\msp$
            \item $c_0\getsr\AEADenc(k,m,\ad)$
            \item $c_1\getsr\BB^{|c_0|}$
            \item Return~$c_b$
        \end{nicodemus}%
    }%
\end{tabular}%%
        }%
    }
    \caption{%
        Games $\INDD$ for authenticated encryption scheme with associated data~$\AEAD$.
    }
    \label{fig:sym:aenc:indd}
\end{figure}

\paragraph{Security: Strong Existential Unforgeability under Chosen-Message Attacks}
The advantage of an adversary~$\advA$ against authenticated encryption scheme with associated data $\AEAD$ in game $\SUFCMA$ from Figure~\ref{fig:sym:aenc:suf} is defined as:
\[
\Adv_\AEAD^\sufcma(\advA)\coloneqq\Pr[\SUFCMA_{\AEAD}^0(\advA)=1]\text{.}
\]

\begin{figure}[!ht]
    \centering
    \nicoresetlinenr%
    \fbox{%
        \scalebox{\codescalefactor}{%
            %!TEX root=../main.tex
\markersetlen{ndL}{120pt}%
\markersetlen{ndR}{130pt}%
\newcommand{\MC}{\mathit{MC}}%
\begin{tabular}[t]{ll}
    \nicodemusbox{\markerlenndL}{%
        \textbf{Game} $\SUFCMA_{\AEAD}(\advA)$
        \begin{nicodemus}
            \item $\MC\gets\emptyset$
            \item $k\getsr\AEADgen$
            \item Invoke $\advA$
            \item Stop with~$0$
        \end{nicodemus}%
        \medskip
        
        \textbf{Oracle} $\Oenc(m,\ad)$
        \begin{nicodemus}
            \item $c\getsr\AEADenc(k,m,\ad)$
            \item $\MC\gets\MC\cup\{(m,\ad,c)\}$
            \item Return $c$
        \end{nicodemus}%
    }%
    &
    \nicodemusbox{\markerlenndR}{%
        \textbf{Oracle} $\Odec(\ad,c)$
        \begin{nicodemus}
            \item $m\gets\AEADdec(k,\ad,c)$
            \item If $(m,\ad,c)\notin\MC\land m\neq\bot$:
            \item \quad Stop with~$1$
            \item Return $m$
        \end{nicodemus}%
    }%
\end{tabular}%%
        }%
    }
    \caption{%
        Game $\SUFCMA$ for authenticated encryption scheme with associated data~$\AEAD$.
    }
    \label{fig:sym:aenc:suf}
\end{figure}




\section{Asymmetric Primitives}
\label{sec:asym}

\subsection{Assumptions}
\label{sec:asym:assumptions}

\subsubsection{Diffie-Hellman (DH)}
\label{sec:asym:assumptions:dh}
Let $\DHgr=(\DHg,\DHp)$ be a group of prime order~$\DHp$ with generator~$\DHg$.

\paragraph{Discrete Logarithm}
The advantage of an adversary~$\advA$ against the discrete logarithm problem in group~$\DHgr$ is defined as:
\[
\Adv_\DHgr^\dlp(\advA)\coloneqq\Pr[\advA(\DHg,\DHp,\DHg^x)=x\mid x\getsr\ZZ_\DHp^*]\text{.}
\]


\paragraph{Computational DH}
The advantage of an adversary~$\advA$ against the computational Diffie-Hellman problem in group~$\DHgr$ is defined as:
\[
\Adv_\DHgr^\cdh(\advA)\coloneqq\Pr[\advA(\DHg,\DHp,\DHg^x,\DHg^y)=,\DHg^{xy}\mid (x,y)\getsr(\ZZ_\DHp^*)^2]\text{.}
\]

\paragraph{Decisional DH}
The advantage of an adversary~$\advA$ against the decisional Diffie-Hellman problem in group~$\DHgr$ is defined as:
\begin{align*}
    \Adv_\DHgr^\ddh(\advA)\coloneqq|&\Pr[\advA(\DHg,\DHp,\DHg^x,\DHg^y,\DHg^{xy})=1\mid (x,y)\getsr(\ZZ_\DHp^*)^2]\\
    &-\Pr[\advA(\DHg,\DHp,\DHg^x,\DHg^y,\DHg^z)=1\mid (x,y,z)\getsr(\ZZ_\DHp^*)^3]|\text{.}
\end{align*}

\subsection{Digital Signature}

\paragraph{Syntax}
A digital signature scheme $\SIG=(\SIGgen,\SIGsig,\SIGvfy)$ is a tuple of three algorithms with signing key space~$\sksp$, verification key space~$\vksp$, message space~$\msp$, and signature space~$\sigsp$:

\begin{itemize}
    \item $\SIGgen: \emptyset \tor \sksp\times\vksp$
    \item $\SIGsig: \sksp\times\msp \tor \sigsp$
    \item $\SIGvfy: \vksp\times\msp\times\sigsp \to \Bool$
\end{itemize}

\paragraph{Correctness}
A digital signature scheme $\SIG$ is correct if
\[
\Pr[\SIGvfy(\vk,m,\SIGsig(\sk,m))=\T\mid (\sk,\vk)\getsr\SIGgen,m\getsr\msp]=1\text{.}
\]

\paragraph{Security}
The advantage of an adversary~$\advA$ against digital signature scheme $\SIG$ in game $\SUFCMA$ from Figure~\ref{fig:sig:suf} is defined as:
\[
\Adv_\SIG^\sufcma(\advA)\coloneqq\Pr[\SUFCMA_{\SIG}(\advA)=1]\text{.}
\]

\begin{figure}[!ht]
    \centering
    \nicoresetlinenr%
    \fbox{%
        \scalebox{\codescalefactor}{%
            %!TEX root=../main.tex
\markersetlen{ndL}{110pt}%
\markersetlen{ndR}{120pt}%
\newcommand{\MS}{\mathit{MS}}%
\begin{tabular}[t]{ll}
    \nicodemusbox{\markerlenndL}{%
        \textbf{Game} $\SUFCMA_{\SIG}(\advA)$
        \begin{nicodemus}
            \item $\MS\gets\emptyset$
            \item $(\sk,\vk)\getsr\SIGgen$
            \item Invoke $\advA(\vk)$
            \item Stop with~$0$
        \end{nicodemus}%
        \medskip
        
        \textbf{Oracle} $\Osig(m)$
        \begin{nicodemus}
            \item $\sig\getsr\SIGsig(\sk,m)$
            \item $\MS\gets\MS\cup\{(m,\sig)\}$
            \item Return $\sig$
        \end{nicodemus}%
    }%
    &
    \nicodemusbox{\markerlenndR}{%
        \textbf{Oracle} $\Ovfy(m,\sig)$
        \begin{nicodemus}
            \item $b\gets\SIGvfy(k,m,\sig)$
            \item If $(m,\sig)\notin\MS\land b=\T$:
            \item \quad Stop with~$1$
            \item Return $b$
        \end{nicodemus}%
    }%
\end{tabular}%%
        }%
    }
    \caption{%
        Game $\SUFCMA$ for digital signature scheme~$\SIG$.
    }
    \label{fig:sig:suf}
\end{figure}

\subsection{Identity-Based KEM}

\paragraph{Syntax}
An identity-based KEM $\IKEM=(\IKEMgen,\IKEMenc,\IKEMdel,\IKEMdec)$ is a tuple of four algorithms with encapsulation key space~$\eksp$, decapsulation key space~$\dksp$, identity space~$\idsp$, ciphertext space~$\csp$, and symmetric key space~$\ksp$:

\begin{itemize}
    \item $\IKEMgen: \emptyset \tor \dksp\times\eksp$
    \item $\IKEMenc: \eksp\times(\idsp)^+ \tor \ksp\times\csp$
    \item $\IKEMdel: \dksp\times\idsp \tor \dksp$
    \item $\IKEMdec: \dksp\times\csp \to \ksp$
\end{itemize}


\section{Stateful Primitives}

\subsection{Forward-Secure KEM}



\subsection{Updatable KEM}

\paragraph{Syntax}
An updatable KEM $\UKEM=(\UKEMgen,\UKEMenc,\UKEMdec)$ is a tuple of three algorithms with encapsulation key space~$\eksp$, decapsulation key space~$\dksp$, ciphertext space~$\csp$, and symmetric key space~$\ksp$:

\begin{itemize}
    \item $\UKEMgen: \emptyset \tor \dksp\times\eksp$
    \item $\UKEMenc: \eksp \tor \eksp\times\ksp\times\csp$
    \item $\UKEMdec: \dksp\times\csp \to \dksp\times\ksp$
\end{itemize}

\paragraph{Construction}


\subsection{Key Exchange}

\subsubsection{Two-Pass Key Exchange}


\paragraph{Construction: DH Key Exchange}

\paragraph{Construction: X3DH}

\subsubsection{Authenticated Key Exchange}

\paragraph{Syntax}

\paragraph{Construction: Signature-Based Authentication}

\paragraph{Construction: KEM-Based Authentication}

\paragraph{Construction: TLS}

\paragraph{Construction: Noise Framework}


\subsection{Ratcheted Key Exchange (RKE)}

\subsubsection{Unidirectional RKE}

\paragraph{Syntax}
A unidirectional key exchange $\URKE=(\URKEinit,\URKEsnd,\URKErcv)$ is a tuple of three algorithms with state space~$\stsp$, ciphertext space~$\csp$, and symmetric key space~$\ksp$:

\begin{itemize}
    \item $\URKEinit: \emptyset \tor \stsp\times\stsp$
    \item $\URKEsnd: \stsp \tor \stsp\times\ksp\times\csp$
    \item $\URKErcv: \stsp\times\csp \to \ksp$
\end{itemize}

\paragraph{Construction}

\subsubsection{Sesquidirectional RKE}

\subsubsection{Bidirectional RKE}

\paragraph{Construction: Double Ratchet}

\subsubsection{Group RKE}

\paragraph{Construction: Sender Key Mechanism}

\paragraph{Construction: Tree-Based DH}

\paragraph{Construction: Tree KEM}

\bibliography{cryptobib/abbrev3,cryptobib/crypto,extra_bib.bib}

\end{document}