\documentclass[a4paper,orivec]{llncs}

\newcommand{\alert}[1]{{\color{red}#1}}

\usepackage{fullpage}
\pagestyle{plain}
\setcounter{tocdepth}{2}

%!TEX root=main.tex

\usepackage{lmodern}
\usepackage[T1]{fontenc}
\usepackage[utf8]{inputenc}

\usepackage{amsfonts,amsmath,amssymb,mathtools}

\usepackage{graphicx}

\usepackage{array}

\usepackage{color,xcolor}
\definecolor{gray}{gray}{0.5}
\definecolor{darkblue}{rgb}{0,0,0.5}
\definecolor{darkgreen}{rgb}{0,0.5,0}
\usepackage[colorlinks=true,linkcolor=darkblue,urlcolor=darkblue,citecolor=darkgreen,pdftitle={Cryptographic Communication Protocols: Key Exchange and Channels}]{hyperref}

\usepackage[strings]{underscore}

\newcommand{\codescalefactor}{0.8}
\newcommand{\tikzscalefactor}{0.8}

\usepackage{nicodemus}
\usepackage{bpmarker}

%draw dashed boxes
\usepackage{dashbox}
\setlength{\dashlength}{4pt}
\setlength{\dashdash}{2pt}

%tikz and libraries
\usepackage{tikz}
\usetikzlibrary{arrows.meta, positioning, calc, trees, shapes, intersections, tikzmark, decorations.pathmorphing}

%some symbols (e.g. lightning bolt)
\usepackage{marvosym}

%placing correlated figures next to each other
\usepackage{subcaption}

%nicer in-line fractions (used for in line case distinction for functions with multiple possible outputs)
%https://mirror.physik.tu-berlin.de/pub/CTAN/macros/latex/contrib/xfrac/xfrac.pdf
\usepackage{xfrac}
%!TEX root=main.tex

\newlength\itemizeskip%used for aligning text in itemize

\let\oldparagraph=\paragraph
\renewcommand\paragraph[1]{\oldparagraph{#1.}}

\newcolumntype{C}[1]{>{\centering\arraybackslash\hspace{0pt}}p{#1}}

%provides \intertext analogue for lists
%\listintertext*{text} inserts text flush with left margin
%\listintertext{text} inserts text flush with left margin of the list environment one higher (if it exists)
%copied from https://tex.stackexchange.com/questions/135726/intertext-like-command-in-enumerate-environment
\makeatletter
\newcommand{\listintertext}{\@ifstar\listintertext@\listintertext@@}
\newcommand{\listintertext@}[1]{% \listintertext*{#1}
  \hspace*{-\@totalleftmargin}#1}
\newcommand{\listintertext@@}[1]{% \listintertext{#1}
  \hspace{-\leftmargin}#1}
\makeatother


\newcommand{\algbox}[2]{\fbox{\parbox{#1}{#2}}}
\newcommand{\arrbox}[2]{\parbox{#1}{\centering#2}}

\newcommand{\getsr}{\gets_\$}
\newcommand{\tor}{\to_\$}
\newcommand{\es}{\epsilon}
\newcommand{\T}{\mathtt{tru}}
\newcommand{\F}{\mathtt{fal}}
\newcommand{\secp}{\kappa}

\newcommand{\ZZ}{\mathbb{Z}}
\newcommand{\BB}{\{0,1\}}
\newcommand{\Bool}{\{\T,\F\}}
\newcommand{\NN}{\mathbb{N}}
\newcommand{\PS}{\mathcal{P}}

\newcommand{\advA}{\mathcal{A}}
\newcommand{\advB}{\mathcal{B}}
\newcommand{\advC}{\mathcal{C}}
\newcommand{\advD}{\mathcal{D}}
\newcommand{\Adv}{\mathrm{Adv}}

\newcommand{\APKE}{\mathrm{APKE}}
\renewcommand{\AE}{\mathrm{AE}}
\newcommand{\IKE}{\mathrm{IKE}}


%VARIABLES
\newcommand{\sk}{\mathit{sk}}
\newcommand{\st}{\mathit{st}}
\newcommand{\vk}{\mathit{vk}}
\newcommand{\pk}{\mathit{pk}}
\newcommand{\ek}{\mathit{ek}}
\newcommand{\dk}{\mathit{dk}}
\newcommand{\rk}{\mathit{rk}}
\newcommand{\ck}{\mathit{ck}}
\newcommand{\mk}{\mathit{mk}}
\newcommand{\mtag}{\tau}
\newcommand{\sig}{\sigma}
\newcommand{\ad}{\mathit{ad}}

%SPACES
\newcommand{\sksp}{\mathcal{SK}}
\newcommand{\pksp}{\mathcal{PK}}
\newcommand{\vksp}{\mathcal{VK}}
\newcommand{\eksp}{\mathcal{EK}}
\newcommand{\dksp}{\mathcal{DK}}
\newcommand{\stsp}{\mathcal{ST}}
\newcommand{\ksp}{\mathcal{K}}
\newcommand{\rsp}{\mathcal{R}}
\newcommand{\msp}{\mathcal{M}}
\newcommand{\csp}{\mathcal{C}}
\newcommand{\tsp}{\mathcal{T}}
\newcommand{\sigsp}{\Sigma}
\newcommand{\adsp}{\mathcal{AD}}
\newcommand{\idsp}{\mathcal{ID}}

%ORACLES
\newcommand{\ROh}{\mathrm{H}}

\newcommand{\CORR}{\mathrm{CORR}}
\newcommand{\IND}{{\mathrm{IND}}}
\newcommand{\ind}{{\mathrm{ind}}}
\newcommand{\INDD}{{\mathrm{IND}\$}}
\newcommand{\indd}{{\mathrm{ind}\$}}
\newcommand{\INDCCA}{{\mathrm{IND}\text{-}\mathrm{CCA}}}
\newcommand{\indcca}{{\mathrm{ind}\text{-}\mathrm{cca}}}
\newcommand{\INDCPA}{{\mathrm{IND}\text{-}\mathrm{CPA}}}
\newcommand{\indcpa}{{\mathrm{ind}\text{-}\mathrm{cpa}}}
\newcommand{\OWCCA}{{\mathrm{OW}\text{-}\mathrm{CCA}}}
\newcommand{\owcca}{{\mathrm{ow}\text{-}\mathrm{cca}}}
\newcommand{\OWCPA}{{\mathrm{OW}\text{-}\mathrm{CPA}}}
\newcommand{\owcpa}{{\mathrm{ow}\text{-}\mathrm{cpa}}}
\newcommand{\SUFCMA}{{\mathrm{SUF}\text{-}\mathrm{CMA}}}
\newcommand{\sufcma}{{\mathrm{suf}\text{-}\mathrm{cma}}}
\newcommand{\ANON}{\mathrm{ANON}}

\newcommand{\Ogen}{\mathrm{Gen}}
\newcommand{\Oinit}{\mathrm{Init}}
\newcommand{\Oenc}{\mathrm{Enc}}
\newcommand{\Odec}{\mathrm{Dec}}
\newcommand{\Oup}{\mathrm{Up}}
\newcommand{\OupPK}{\mathrm{UpPK}}
\newcommand{\OupSK}{\mathrm{UpSK}}

\newcommand{\Otag}{\mathrm{Tag}}
\newcommand{\Osig}{\mathrm{Sign}}
\newcommand{\Ovfy}{\mathrm{Vfy}}

\newcommand{\Osnd}{\mathrm{Snd}}
\newcommand{\Orsp}{\mathrm{Rsp}}
\newcommand{\Orcv}{\mathrm{Rcv}}
\newcommand{\Ochall}{\mathrm{Chall}}

\newcommand{\Ocorrupt}{\mathrm{Corrupt}}
\newcommand{\Oexp}{\mathrm{Expose}}
\newcommand{\Oexps}{\mathrm{ExposeS}}
\newcommand{\Oexpr}{\mathrm{ExposeR}}
\newcommand{\Oreveal}{\mathrm{Reveal}}



\iffalse
\newcommand{\full}{\!$\cdot$\,}
\newcommand{\core}{\hphantom{\!$\cdot$\,}}

\newcommand{\getscup}{\overset{\raisebox{-1pt}{\tiny$\;\cup$}}{\gets}}
\newcommand{\getsconcat}{\overset{\raisebox{-1pt}{\tiny$\;\shortparallel$}}{\gets}} % "concat assignment"

\newcommand{\bO}{\mathcal{O}}
\newcommand{\secp}{\lambda}

\fi

%DH
\newcommand{\DHgr}{\mathbb{G}}
\newcommand{\DHg}{g}
\newcommand{\DHp}{p}

\newcommand{\dlp}{\mathrm{dlp}}
\newcommand{\cdh}{\mathrm{cdh}}
\newcommand{\ddh}{\mathrm{ddh}}

%PRG and PRF
\newcommand{\PRF}{\mathrm{PRF}}
\newcommand{\PRG}{\mathrm{PRG}}

\newcommand{\PRGf}{\PRG.\mathrm{f}}
\newcommand{\PRFf}{\PRF.\mathrm{f}}

%KDF
\newcommand{\KDF}{\mathrm{KDF}}

%MAC and DIGITAL SIGNATURE
\newcommand{\MAC}{\mathrm{MAC}}
\newcommand{\SIG}{\mathrm{SIG}}

\newcommand{\MACgen}{\MAC.\mathrm{gen}}
\newcommand{\MACtag}{\MAC.\mathrm{tag}}
\newcommand{\MACvfy}{\MAC.\mathrm{vfy}}

\newcommand{\SIGgen}{\SIG.\mathrm{gen}}
\newcommand{\SIGsig}{\SIG.\mathrm{sig}}
\newcommand{\SIGvfy}{\SIG.\mathrm{vfy}}

%SYMMETRIC ENCRYPTION, PUBLIC-KEY ENCRYPTION, various KEMs, and KEY EXCHANGE
\newcommand{\SYE}{\mathrm{SE}}
\newcommand{\AEAD}{\mathrm{AEAD}}
\newcommand{\PKE}{\mathrm{PKE}}
\newcommand{\KEM}{\mathrm{KEM}}
\newcommand{\IKEM}{\mathrm{IKEM}}
\newcommand{\FKEM}{\mathrm{FKEM}}
\newcommand{\UKEM}{\mathrm{UKEM}}
\newcommand{\KUKEM}{\mathrm{KU}\text{-}\mathrm{KEM}} %not sure if this is the best way to typeset the "-" in KU-KEM 
\newcommand{\TKE}{\mathrm{TKE}}
\newcommand{\AKE}{\mathrm{AKE}}
\newcommand{\URKE}{\mathrm{URKE}}
\newcommand{\HIBE}{\mathrm{HIBE}}

\newcommand{\AEADgen}{\AEAD.\mathrm{gen}}
\newcommand{\AEADenc}{\AEAD.\mathrm{enc}}
\newcommand{\AEADdec}{\AEAD.\mathrm{dec}}

\newcommand{\SYEgen}{\SYE.\mathrm{gen}}
\newcommand{\SYEenc}{\SYE.\mathrm{enc}}
\newcommand{\SYEdec}{\SYE.\mathrm{dec}}

\newcommand{\PKEgen}{\PKE.\mathrm{gen}}
\newcommand{\PKEenc}{\PKE.\mathrm{enc}}
\newcommand{\PKEdec}{\PKE.\mathrm{dec}}

\newcommand{\KEMgen}{\KEM.\mathrm{gen}}
\newcommand{\KEMenc}{\KEM.\mathrm{enc}}
\newcommand{\KEMdec}{\KEM.\mathrm{dec}}

\newcommand{\IKEMgen}{\IKEM.\mathrm{gen}}
\newcommand{\IKEMenc}{\IKEM.\mathrm{enc}}
\newcommand{\IKEMdel}{\IKEM.\mathrm{del}}
\newcommand{\IKEMdec}{\IKEM.\mathrm{dec}}

\newcommand{\UKEMgen}{\UKEM.\mathrm{gen}}
\newcommand{\UKEMenc}{\UKEM.\mathrm{enc}}
\newcommand{\UKEMdec}{\UKEM.\mathrm{dec}}

\newcommand{\KUKEMgen}{\KUKEM.\mathrm{gen}}
\newcommand{\KUKEMenc}{\KUKEM.\mathrm{enc}}
\newcommand{\KUKEMdec}{\KUKEM.\mathrm{dec}}
\newcommand{\KUKEMup}{\KUKEM.\mathrm{up}}

\newcommand{\FKEMgen}{\FKEM.\mathrm{gen}}
\newcommand{\FKEMenc}{\FKEM.\mathrm{enc}}
\newcommand{\FKEMdec}{\FKEM.\mathrm{dec}}

\newcommand{\TKEsnd}{\TKE.\mathrm{snd}}
\newcommand{\TKErsp}{\TKE.\mathrm{rsp}}
\newcommand{\TKErcv}{\TKE.\mathrm{rcv}}

\newcommand{\AKEgen}{\AKE.\mathrm{gen}}
\newcommand{\AKEsnd}{\AKE.\mathrm{snd}}
\newcommand{\AKErsp}{\AKE.\mathrm{rsp}}
\newcommand{\AKErcv}{\AKE.\mathrm{rcv}}

\newcommand{\URKEinit}{\URKE.\mathrm{init}}
\newcommand{\URKEsnd}{\URKE.\mathrm{snd}}
\newcommand{\URKErcv}{\URKE.\mathrm{rcv}}

\newcommand{\HIBEgen}{\HIBE.\mathrm{gen}}
\newcommand{\HIBEenc}{\HIBE.\mathrm{enc}}
\newcommand{\HIBEdec}{\HIBE.\mathrm{dec}}
\newcommand{\HIBEdel}{\HIBE.\mathrm{del}}


\bibliographystyle{alpha}

\makeatletter
\renewcommand*\l@author[2]{}
\renewcommand*\l@title[2]{}
\makeatletter
\renewcommand{\contentsname}{}


\title{Cryptographic Communication Protocols:\\Key Exchange and Channels}
\author{Paul Rösler}

\institute{FAU Erlangen-Nürnberg}

\begin{document}

\maketitle
\begin{center}
    \today
\end{center}

\begingroup
\let\clearpage\relax
\tableofcontents
\endgroup


\section{Administrative Information}
\label{sec:admin}

\paragraph{Exercise}
The exercize is split into two parts.
Some tasks can be submitted online via the interactive StudOn quiz.
Solving these quizzes awards bonus points for the final exam, if it is passed (at most a 0.7 grade improvement).
Some tasks are not part of the quiz and will be discussed in the following exercise sessions.
These tasks do not count towards bonus points, but are still important to form a deeper understanding of the more complex systems.

\paragraph{Recordings}
Lecture recordings from 2024 are available at \url{https://video.cs.fau.de/by-lecture/CCP/2024s}.
The exercise is not recorded.

\section{Preliminary Remarks}
This document is not (yet) a full course script.
Instead, it is meant as an additional resource that systematizes the considered primitives, definitions, and constructions.
The author invites readers to submit comments or pull requests via the GitHub Repository \url{https://github.com/roeslpa/ccpScript}.


\section{Notation}

\begin{tabular}{|l|p{13cm}|}\hline
    \textbf{Symbol} & \textbf{Meaning}\\\hline
    $\T$ / $\F$ & Boolean values true and false\\
    $\gets$ / $\to$ & Assigns a constant expression or the output of a deterministic algorithm\\
    $\getsr$ / $\tor$ & Assigns a random value uniformly sampled from a finite set or the output of a probabilistic algorithm\\
    $\PS(X)$ & Power set of set~$X$\\
    $(X)^+$ & Set of non-trivial concatenations of elements from set~$X$\\
    $X[\cdot]\gets x$ & Assigns all entries of array~$X$ with default value~$x$\\
    Invoke $X$ & Executes algorithm $X$\\
    Stop with $x$ & Terminates the running experiment with final output~$x$\\
    Require $x$ & Ends the algorithm, resp.~oracle, if expression~$x=\F$\\\hline
\end{tabular}


\section{Overview}
\label{sec:overview}

The goal of this course is to \emph{understand how to analyze the security of real-world protocols}.
\emph{Analyzing} comprises
\begin{enumerate}
    \item Methodically Defining Security
    \item Identifying the Cryptographic Core of Protocols
    \item Systematically Finding Attacks against Protocols
    \item Formally Proving Security of Protocols
\end{enumerate}
This course focuses on the following classes of \emph{real-world protocols} and their main building blocks:
\begin{itemize}
    \item Key Exchange Protocols
    \item Simple Communication Channels
    \item Messaging Protocols
    \item Group Membership Management Protocols
\end{itemize}

\subsection{First Informal Example: Diffie-Hellman Key Exchange}
\label{sec:overview:dhke}

Let $\DHgr=(\DHg,\DHp)$ be a group of prime order~$\DHp$ with generator~$\DHg$.
That is, $\DHgr$ is a set with a multiplicativly written operation~$\cdot$. A
simple example is the Group $\DHgr=(\ZZ_5^*,\cdot)$, where $\ZZ_5^* =
\{1,2,3,4\}$. Then a generator for $\DHgr$ is $\DHg=2$, since $2^0=1$, $2^1=2$,
$2^2=4$, $2^3=8\equiv3\mod 5$, and $2^4=16\equiv 1\mod 5$. (Applying the
group operation to the generator~$\DHg$ multiple times gives all group elements.)

Then, the Diffie-Hellman Key Exchange~(DHKE)~\cite{DifHel76}
between two users Alice and Bob works as follows:
\begin{enumerate}
    \item Alice samples $x\getsr\ZZ_\DHp^*$, locally stores $x$, and sends $X\gets(\DHg^x\mod\DHp)$ to Bob
    \item Bob samples $y\getsr\ZZ_\DHp^*$, computes $k\gets(X^y\mod\DHp)$, and sends $Y\gets(\DHg^y\mod\DHp)$ to Alice
    \item Alice computes $k\gets(Y^x\mod\DHp)$, and forgets $x$.
\end{enumerate}
Since $k\equiv X^y\equiv\DHg^{xy}\equiv Y^x\equiv k\mod \DHp$, Alice and Bob compute the same shared key.

\begin{figure}
    \centering
    \begin{tabular}{lC{3cm}l}
        \textbf{Alice} & & \textbf{Bob}\\
        $x\getsr\ZZ_\DHp^*$ & &\\
        $X\gets\DHg^x\mod\DHp$ & $\xrightarrow{X}$ & $y\getsr\ZZ_\DHp^*$\\
        & $\xleftarrow{Y}$ & $Y\gets \DHg^y\mod\DHp$\\
        $k\gets Y^x\mod\DHp$ & & $k\gets X^y\mod\DHp$\\
    \end{tabular}
    \caption{Diffie-Hellman Key Exchange.}
    \label{fig:dhke}
\end{figure}

\paragraph{Security of DHKE}
To prove security of the DHKE protocol, we use the \emph{Decisional Diffie-Hellman}~(DDH) problem (See Figure~\ref{fig:ddh_assumption}).
The advantage of an adversary~$\advA$ against the DDH problem in group~$\DHgr$ is defined as:
% \begin{align*}
%     \Adv_\DHgr^\ddh(\advA)\coloneqq|&\Pr[\advA(\DHg,\DHp,\DHg^x,\DHg^y,\DHg^{xy})=1\mid (x,y)\getsr(\ZZ_\DHp^*)^2]\\
%     &-\Pr[\advA(\DHg,\DHp,\DHg^x,\DHg^y,\DHg^z)=1\mid (x,y,z)\getsr(\ZZ_\DHp^*)^3]|\text{.}
% \end{align*}
%new advantage definition to be consistent with the lecture
\[\Adv_{\DHgr}^\ddh(\advA)\coloneqq\left|\Pr[\text{DDH}^0_{g,p}(\advA)=1]-\Pr[\text{DDH}^1_{g,p}(\advA)=1]\right|\]
\begin{figure}[t]
    \centering
    %!TEX root=../main.tex
\begin{tabular}{lll}
    \algbox{4.5cm}{%
        \textbf{DDH}$^b_{g,p}$\\
        $x,y,z \getsr \ZZ_p^*$\\
        $X\gets g^x, Y\gets g^y,Z_0\gets g^{xy}$\\
        $Z_1\gets g^z$\\~\\
        Return $b'$}&
    \arrbox{2cm}{%    
        ~\\~\\$\xrightarrow{(\DHg,\DHp,X,Y,Z_p)}$\\~\\
        $\xleftarrow{b'}$}&
    \algbox{3cm}{%
        \textbf{Adversary} $\advA$:\\~\\~\\~\\~\\}
\end{tabular}
    \caption{The Decisional Diffie-Hellman~(DDH) problem.}
    \label{fig:ddh_assumption}
\end{figure}
Or 
Intuitively, adversary~$\advA$ plays a game in which it sees both Diffie-Hellman shares $\DHg^x$ and $\DHg^y$ as well as either the real key~$\DHg^{xy}$~$(Z_0)$ or a random group element~$\DHg^z$~$(Z_1)$.
To win this game, the adversary has to \emph{decide} which of the two views they saw.
The advantage $\Adv_\DHgr^\ddh(\advA)$ denotes the probability of $\advA$ winning this game beyond pure guessing.

We define the \emph{Computational DH}~(CDH) and the \emph{Discrete Logarithm}~(DLog) problems for prime-order groups in Section~\ref{sec:asym:assumptions:dh}.

Based on the hardness of the best known attacks against the DDH problem from the long history of work in number theory, we assume that the DDH problem is \emph{hard} for security parameter~$\secp$.
This means that $\Adv_\DHgr^\ddh(\advA)$ is \emph{negligible} for all adversaries~$\advA$ that can be specified as \emph{probabilistic polynomial-time Turing machines} (PPT).

\paragraph{Simplifications}
For simplicity and clarity, we refrain from considering concrete group generation algorithms and we do not elaborate on hardness analyses of the DDH problem.
Furthermore, instead of treating security \emph{asymptotically} with respect to a security parameter~$\secp$, using the definitions of ``negligible'' and ``PPT adversaries'', we consider security \emph{concretely} in this document.
This means that all our reductionist security proofs provide concrete relations between the security of the analyzed protocols and the hardness of the underlying assumptions.

\paragraph{Real-World Remarks}
In practice, Elliptic Curves are often used to instantiate prime-order groups~$\DHgr$.
While the DHKE is one of the most widely used key exchange protocols in the real world, we want to mention that Shor's algorithm~\cite{FOCS:Shor94} solves the DDH problem in polynomial time using a quantum computer.
Thus, we only use the DHKE as a simple, introductory real-world example.

\paragraph{Intuitive Security Definition: Passive Adversaries}
A passive adversary~$\advA$ against the DHKE protocol is \emph{capable} of seeing the transcript of the interaction between Alice and Bob: $X=\DHg^x$ and $Y=\DHg^y$.
The \emph{goal} of the adversary is to derive information about the exchanged key~$k=\DHg^{xy}$ from that seen transcript.
To capture this goal, we challenge~$\advA$ eventually on key~$k$:
$\advA$ either sees the real key~$k_0=k=\DHg^{xy}$ or a randomly sampled element from the key space~$k_1\getsr\ksp$, where $k_1\gets\DHg^z$ and $z\getsr\ZZ_\DHp^*$.
If the adversary can distinguish these two cases, this is considered a \emph{successful} attack against the key exchange protocol.
In contrast, if we can prove that distinguishing the real key from a random key is hard, then we can treat the real key as if it was sampled randomly from a uniform distribution;
more intuitively, then we know that the adversary has no information about the real key, even when observing the interaction between Alice and Bob.

We emphasize that the above intuitive description of the adversarial capabilities, the attack goal, and the success criteria should only sketch the expected security requirements for the DHKE protocol.
As we will see, \emph{good} security definitions should not be specified with a concrete protocol design in mind.

\paragraph{Sketch of Security Proof}
The above described passive adversary~$\advA$ against DHKE obtains inputs $(X,Y)$ and $k_b$, where $X=\DHg^x$, $Y=\DHg^y$, $k_0=k=\DHg^{xy}$, $k_1=\DHg^z$, $(x,y,z)\in\ZZ_\DHp^*$, and $b\in\BB$.
If $\advA$, upon processing these inputs, outputs $b'$ such that $b=b'$, $\advA$ breaks the required security goal.

\begin{figure}
    \centering
    %!TEX root=../main.tex
\begin{tabular}{lllll}
    \algbox{3.5cm}{%
        \textbf{Challenger} $\advC_{\DHgr=(\DHg,\DHp)}^{\ddh,b}$:\\
        $(x,y,z)\getsr(\ZZ_\DHp^*)^3$\\
        $(X,Y)\gets(\DHg^x,\DHg^y)$\\
        $(Z_0,Z_1)\gets(\DHg^{xy},\DHg^z)$\\
        $Z\gets Z_b$\\\\
        Stop with $b''$}&
    \arrbox{2cm}{%
        $\xrightarrow{(\DHg,\DHp,X,Y,Z)}$\\~\\~\\
        $\xleftarrow{b''}$}&
    \algbox{3cm}{%
        \textbf{Reduction} $\advB^\ddh$:\\
        $k\gets Z$\\\\
        $b''\gets b'$}&
    \arrbox{2cm}{%
        $\xrightarrow{(\DHg,\DHp)}$\\
        $\xrightarrow{(X,Y)}$\\
        $\xrightarrow{k}$\\
        $\xleftarrow{b'}$}&
    \algbox{3cm}{%
        \textbf{Adversary} $\advA$\\\\\\}\\
\end{tabular}
    \caption{Sketch of reduction from Diffie-Hellman key exchange to DDH problem.}
    \label{fig:dhke:reduction}
\end{figure}

The primary proof method in this course is based on reductions via sequences of games~\cite{EPRINT:Shoup04} that we will specify in pseudo-code~\cite{EC:BelRog06}.
However, for the proof sketch of our simple example in which we reduce the security of DHKE to the hardness of the DDH problem, the schematic overview of reduction $\advB^\ddh$ in Figure~\ref{fig:dhke:reduction} suffices:
The reduction uses its DDH-challenge~$Z$ as the challenge key~$k$.
Once the DHKE-adversary~$\advA$ solves this challenge by successfully guessing bit~$b'$, the reduction can use this guess to break the DDH-challenge via bit~$b''$.
A formal proof remains a task for the interested reader.

\subsection{Second Example: Public-Key Encryption (PKE)}
\label{sec:overview:pke}

\paragraph{Syntax}
A public-key encryption scheme $\PKE=(\PKEgen,\PKEenc,\PKEdec)$ is a tuple of three algorithms with encryption key space~$\eksp$, decryption key space~$\dksp$, message space~$\msp$, and ciphertext space~$\csp$:

\begin{itemize}
    \item $\PKEgen: \emptyset \tor \dksp\times\eksp$
    \item $\PKEenc: \eksp\times\msp \tor \csp$
    \item $\PKEdec: \dksp\times\csp \to \msp$
\end{itemize}

\paragraph{Correctness}
A public-key encryption scheme $\PKE$ is correct if
\[
\Pr[\PKEdec(\dk,\PKEenc(\ek,m))=m\mid (\dk,\ek)\getsr\PKEgen,m\getsr\msp]=1\text{.}
\]

\paragraph{Passive Security}
We begin with a notion of passive security for PKE that is called \emph{Indistinguishability of Ciphertexts under Chosen Plaintext Attacks} (IND-CPA).
This models that adversaries only eavesdrop victims' ciphertexts for adversarially chosen messages---hence, \emph{chosen plaintext attacks}.
However, the adversary never observes the decryption of ciphertexts that were not created by the victims themselves.
The goal of the adversary is to decide which of two possible messages was encrypted in a challenge ciphertext---hence, \emph{indistinguishability of ciphertext}.

We introduce this passive notion of security mainly for didactic reasons.
Yet, we want to mention that the Fujisaki-Okamoto transform~\cite{C:FujOka99} lifts passively secure PKE schemes to actively secure ones.
Hence, this notion is indeed relevant in practice.

The advantage of an adversary~$\advA$ against public-key encryption scheme $\PKE$ in game $\INDCPA$ from Figure~\ref{fig:pke:ind:cpa} is defined as:
\[
\Adv_\PKE^\indcpa(\advA)\coloneqq\left|\Pr[\INDCPA_{\PKE}^0(\advA)=1]-\Pr[\INDCPA_{\PKE}^1(\advA)=1]\right|\text{.}
\]

\begin{figure}[!ht]
    \centering
    \nicoresetlinenr%
    \fbox{%
        \scalebox{\codescalefactor}{%
            %!TEX root=../main.tex
\markersetlen{ndL}{110pt}%
\markersetlen{ndR}{120pt}%
\newcommand{\CC}{\mathit{CC}}%
\begin{tabular}[t]{ll}
    \nicodemusbox{\markerlenndL}{%
        \textbf{Game} $\INDCPA_{\PKE}^b(\advA)$
        \begin{nicodemus}
            \item $(\dk,\ek)\getsr\PKEgen$
            \item $b'\getsr\advA(\ek)$
            \item Stop with~$b'$
        \end{nicodemus}%
    }%
    &
    \nicodemusbox{\markerlenndR}{%
        \textbf{Oracle} $\Ochall(m_0,m_1)$
        \begin{nicodemus}
            \item Require $\{m_0,m_1\}\subseteq\msp$
            \item $c\getsr\PKEenc(\ek,m_b)$
            \item Return~$c$
        \end{nicodemus}%
    }%
\end{tabular}%%
        }%
    }
    \caption{%
        Games $\INDCPA$ for public-key encryption scheme~$\PKE$.
    }
    \label{fig:pke:ind:cpa}
\end{figure}

\paragraph{Active Security}
The ability to actively manipulate ciphertexts and observe the victims' reaction on the decryption is modeled by providing a decryption oracle.
This additional oracle returns decrypted messages for all ciphertext queries except for those that correspond to challenge ciphertexts.
If also decryptions of challenge ciphertexts were given to the adversary, the adversary could trivially win the game for any functional PKE construction.
Therefore, we call such attack strategies \emph{trivial attacks} or \emph{trivial winning conditions}.

\begin{figure}[!ht]
    \centering
    \nicoresetlinenr%
    \fbox{%
        \scalebox{\codescalefactor}{%
            %!TEX root=../main.tex
\markersetlen{ndL}{110pt}%
\markersetlen{ndR}{120pt}%
\newcommand{\CC}{\mathit{CC}}%
\begin{tabular}[t]{ll}
    \nicodemusbox{\markerlenndL}{%
        \textbf{Game} $\INDCCA_{\PKE}^b(\advA)$
        \begin{nicodemus}
            \item $\CC\gets\emptyset$
            \item $(\dk,\ek)\getsr\PKEgen$
            \item $b'\getsr\advA(\ek)$
            \item Stop with~$b'$
        \end{nicodemus}%
    }%
    &
    \nicodemusbox{\markerlenndR}{%
        \textbf{Oracle} $\Ochall(m_0,m_1)$
        \begin{nicodemus}
            \item Require $\{m_0,m_1\}\subseteq\msp$
            \item $c\getsr\PKEenc(\ek,m_b)$
            \item $\CC\gets\CC\cup\{c\}$
            \item Return~$c$
        \end{nicodemus}%
        \medskip
        
        \textbf{Oracle} $\Odec(c)$
        \begin{nicodemus}
            \item Require $c\notin\CC$
            \item $m\gets\PKEdec(\dk,c)$
            \item Return $m$
        \end{nicodemus}%
    }%
\end{tabular}%%
        }%
    }
    \caption{%
        Games $\INDCCA$ for public-key encryption scheme~$\PKE$.
    }
    \label{fig:pke:ind}
\end{figure}

The advantage of an adversary~$\advA$ against public-key encryption scheme $\PKE$ in game $\INDCCA$ from Figure~\ref{fig:pke:ind} is defined as:
\[
\Adv_\PKE^\indcca(\advA)\coloneqq\left|\Pr[\INDCCA_{\PKE}^0(\advA)=1]-\Pr[\INDCCA_{\PKE}^1(\advA)=1]\right|\text{.}
\]

\paragraph{ElGamal Encryption}
As shown in Figure~\ref{fig:pke:const:elgamal}, the ElGamal public-key encryption scheme~\cite{ElGamal85} at its core uses the Diffie-Hellman key exchange.
Intuitively, the ElGamal PKE key pair~$(\dk,\ek)$ is Alice's key pair in the DHKE.
For ElGamal encryption, one samples a fresh, ephemeral DHKE key pair, which corresponds to Bob's key in the DHKE.
Using Alice's DHKE public key and Bob's DHKE secret key, one computes the shared key and multiplies the message to that shared key.
On ElGamal decryption, the same shared key is computed and, by multiplying with its inverse, the original, encrypted message is derived.

That decryption yields the original message can be seen from the following equation:
\[m\gets\frac{c_2}{c_1^\dk}=\frac{\ek^x\cdot m}{(g^x)^\dk}=\frac{g^{x\dk}\cdot m}{g^{x\dk}}=m\]
\begin{figure}[!ht]
    \centering
    \nicoresetlinenr%
    \fbox{%
        \scalebox{\codescalefactor}{%
            %!TEX root=../main.tex
\markersetlen{ndL}{110pt}%
\markersetlen{ndM}{120pt}%
\markersetlen{ndR}{120pt}%
\begin{tabular}[t]{lll}
    \nicodemusbox{\markerlenndL}{%
        \textbf{Proc} $\PKEgen$
        \begin{nicodemus}
            \item $\dk\getsr\ZZ_\DHp^*$
            \item $\ek\gets\DHg^\dk$
            \item Return $(\dk,\ek)$
        \end{nicodemus}%
    }%
    &
    \nicodemusbox{\markerlenndM}{%
        \textbf{Proc} $\PKEenc(\ek,m)$
        \begin{nicodemus}
            \item $x\getsr\ZZ_\DHp^*$
            \item $c_1\gets\DHg^x$
            \item $c_2\gets\ek^x\cdot m$
            \item $c\gets(c_1,c_2)$
            \item Return~$c$
        \end{nicodemus}%
    }%
    &
    \nicodemusbox{\markerlenndR}{%
        \textbf{Proc} $\PKEdec(\dk,c)$
        \begin{nicodemus}
            \item $(c_1,c_2)\gets c$
            \item $m\gets c_2/c_1^\dk$
            \item Return $m$
        \end{nicodemus}%
    }%
\end{tabular}%%
        }%
    }
    \caption{%
        ElGamal public-key encryption scheme~$\PKE$~\cite{ElGamal85}.
    }
    \label{fig:pke:const:elgamal}
\end{figure}

\paragraph{Security of ElGamal}
To prove IND-CPA security of the ElGamal encryption scheme, we assume that there is a successful adversary~$\advA$ against the IND-CPA security of $\PKE$ (Figure~\ref{fig:elgamal:ind}).
We then construct another game $G_1^b$ (Figure~\ref{fig:elgamal:g1}) that is a small modification of the ElGamal IND-CPA game.
There can be no adversary against $G_1^b$ that does better than just guessing the bit $b$.
This is because every value exposed to the adversary is independent of $b$.
For $\pk$ and $c_1$ this is obvious and since $z$ is uniformly sampled from $\ZZ_p^*$ $c_2$ is also random.

To show that games $G_1^b$ and $\INDCPA_\PKE^b$ are indistinguishable, we turn every successful distinguisher~$\advD$ into a successful adversary against the DDH problem (Figure~\ref{fig:elgamal:reduction}).
The distinguisher~$\advD$ is an algorithm that outputs $d\gets\{0,1\}$ depending on which game ($\INDCPA_\PKE^b$ or $G_1^b$) it is playing against.

\begin{figure}[!ht]
    \centering
    \begin{minipage}{0.49\textwidth}
        \centering
        %!TEX root=../main.tex
\begin{tabular}{lll}
    \algbox{3.5cm}{%
        \textbf{Game} $\INDCPA_\PKE^b$:\\
        $\dk\gets\ZZ_p^*,\pk\gets\DHg^\dk$\\
        $y\getsr\ZZ_p^*$\\
        $c_1\gets\DHg^y, c_2^1\gets\DHg^{\dk\cdot y}$\\
        $c_2\gets c_2^1\cdot m_b$\\
        $c^*\gets(c_1,c_2)$\\
        Return $b'$}&
    \arrbox{1cm}{%
        $\xrightarrow{\pk}$\\   
        $\xleftarrow{m_0,m_1}$\\
        $\xrightarrow{c^*}$\\
        $\xleftarrow{b'}$}&
    \algbox{2.5cm}{%
        \textbf{Adversary} $\advA$\\~\\~\\~\\~\\~\\}
\end{tabular}
        \caption{Game $\INDCPA_\PKE^b$}
        \label{fig:elgamal:ind}
    \end{minipage}
    \hfill
    \begin{minipage}{0.49\textwidth}
        \centering
        %!TEX root=../main.tex
\begin{tabular}{lll}
    \algbox{3.5cm}{%
        \textbf{Game} $G_1^b$:\\
        $\dk\gets\ZZ_p^*,\pk\gets\DHg^\dk$\\
        $y,{\color{red}z}\getsr\ZZ_p^*$\\
        $c_1\gets\DHg^y, c_2^1\gets{\color{red}\DHg^z}$\\
        $c_2\gets c_2^1\cdot m_b$\\
        $c^*\gets(c_1,c_2)$\\
        Return $b'$}&
    \arrbox{1cm}{%
        $\xrightarrow{\pk}$\\
        $\xleftarrow{m_0,m_1}$\\
        $\xrightarrow{c^*}$\\
        $\xleftarrow{b'}$}&
    \algbox{2.5cm}{%
        \textbf{Adversary} $\advA$\\~\\~\\~\\~\\~\\}
\end{tabular}
        \caption{Game $G_1^b$}
        \label{fig:elgamal:g1}
    \end{minipage}
\end{figure}
\begin{figure}[!ht]
    \centering
    %!TEX root=../main.tex
\begin{tabular}{lllll}
    \algbox{4cm}{%
        \textbf{Game} $\advC_{\DHgr=(\DHg,\DHp)}^{\ddh,b}$:\\
        $x,y,z\getsr(\ZZ_\DHp^*)^3$\\
        $X\gets\DHg^x,Y\gets\DHg^y, Z_0\gets\DHg^{xy}$\\
        $Z_1\gets\DHg^z, Z\gets Z_{b'}$\\~\\~\\
        Return $b''$
        }&
    \arrbox{2cm}{%
        $\xrightarrow{(\DHg,\DHp,X,Y,Z)}$\\~\\~\\
        $\xleftarrow{b''}$}&
    \algbox{3cm}{%
        \textbf{Reduction} $\advB^\ddh$:\\
        $\pk\gets X$\\
        $c_1\gets Y, c_2^1\gets Z$\\
        $c_2\gets c_2^1\cdot m_b$\\
        $c^*\gets(c_1,c_2)$\\
        $b''\gets d$\\
        Return $b''$}&
    \arrbox{2cm}{%
        $\xrightarrow{\pk}$\\
        $\xleftarrow{m_0,m_1}$\\
        $\xrightarrow{c^*}$\\
        $\xleftarrow{d}$}&
    \algbox{3cm}{%
        \textbf{Adversary} $\advD$\\~\\~\\~\\~\\~\\} 
\end{tabular}
    \caption{Sketch of reduction from distinguisher $\advD$ to DDH problem.}
    \label{fig:elgamal:reduction}
\end{figure}



\section{Game-Based Definitions}
As one can see from the minimal differences between the \emph{passive} security definition and the \emph{active} security definition for PKE from Section~\ref{sec:overview:pke}, definitions of security can be modular with respect to a pattern that is based on a small core.
We will use the example of \emph{Key Encapsulation Mechanisms}~(KEM) to introduce a methodology for defining security systematically.
This methodology is closely related to a concept called \emph{Indistinguishability up to Correctness} by Rogaway and Zhang~\cite{C:RogZha18}.

\subsection{Syntax}
The first step to formally consider a cryptographic primitive is defining its \emph{syntax}.
The syntax specifies the algorithms of a primitive by defining their inputs and outputs.

A key encapsulation mechanism $\KEM=(\KEMgen,\KEMenc,\KEMdec)$ is a tuple of three algorithms with encapsulation key space~$\eksp$, decapsulation key space~$\dksp$, ciphertext space~$\csp$, and symmetric key space~$\ksp$:
\begin{itemize}
    \item $\KEMgen: \emptyset \tor \dksp\times\eksp$
    \item $\KEMenc: \eksp \tor \ksp\times\csp$
    \item $\KEMdec: \dksp\times\csp \to \ksp$
\end{itemize}

Intuitively a KEM can be considered a special case of PKE that is limited to always encrypt (aka.\ encapsulate) fresh symmetric keys.
Instead of taking the symmetric keys as input, the encapsulation algorithm of a KEM generates them internally.

\paragraph{Construction: ElGamal KEM}
For illustration, we give the ElGamal KEM as a simple example of a KEM in Figure~\ref{fig:kem:const:elgamal}.

\begin{figure}[!ht]
    \centering
    \nicoresetlinenr%
    \fbox{%
        \scalebox{\codescalefactor}{%
            %!TEX root=../main.tex
\markersetlen{ndL}{110pt}%
\markersetlen{ndM}{120pt}%
\markersetlen{ndR}{120pt}%
\begin{tabular}[t]{lll}
    \nicodemusbox{\markerlenndL}{%
        \textbf{Proc} $\KEMgen$
        \begin{nicodemus}
            \item $\dk\getsr\ZZ_\DHp^*$
            \item $\ek\gets\DHg^\dk$
            \item Return $(\dk,\ek)$
        \end{nicodemus}%
    }%
    &
    \nicodemusbox{\markerlenndM}{%
        \textbf{Proc} $\KEMenc(\ek)$
        \begin{nicodemus}
            \item $x\getsr\ZZ_\DHp^*$
            \item $k\gets\ek^x$
            \item $c\gets\DHg^x$
            \item Return~$(k,c)$
        \end{nicodemus}%
    }%
    &
    \nicodemusbox{\markerlenndR}{%
        \textbf{Proc} $\KEMdec(\dk,c)$
        \begin{nicodemus}
            \item $k\gets c^\dk$
            \item Return $k$
        \end{nicodemus}%
    }%
\end{tabular}%%
        }%
    }
    \caption{%
        ElGamal key encapsulation mechanism~$\KEM$~\cite{ElGamal85}.
    }
    \label{fig:kem:const:elgamal}
\end{figure}

\subsection{Correctness}
Before specifying the expected security guarantees, we define the \emph{correctness} requirements.
Typically, these requirements capture the functionality that a primitive should offer when being executed in an \emph{honest} environment in which all traffic is delivered as intended.
The literature on distributed systems developed two complementary notions of correctness:
\begin{itemize}
    \item \textbf{Safety:} Nothing ``bad'' should happen (e.g., encapsulation and decapsulation for the same ciphertext should not compute different symmetric keys)
    \item \textbf{Liveness:} The ``good'' event always happens (e.g., encapsulation and decapsulation always output a non-trivial key)
\end{itemize}

A correctness definition in the sense of \emph{safety} typically suffices for the ultimate goal of defining and analyzing security.
Furthermore, specifying \emph{liveness} requirements for sophisticated primitives sometimes leads to complex, incomprehensible definitions.

A key encapsulation mechanism $\KEM$ is correct if
\[
\Pr[\KEMdec(\dk,c)=k\mid (\dk,\ek)\getsr\KEMgen,(k,c)\getsr\KEMenc(\ek))=m]=1\text{.}
\]
Equivalently, a key encapsulation mechanism $\KEM$ is correct if $\Pr[\CORR_{\KEM}(\advA)=0]=1$ for all adversaries~$\advA$, where game~$\CORR$ is defined in Figure~\ref{fig:kem:corr}.

\begin{figure}[!ht]
    \centering
    \nicoresetlinenr%
    \fbox{%
        \scalebox{\codescalefactor}{%
            %!TEX root=../main.tex
\markersetlen{ndL}{110pt}%
\markersetlen{ndR}{100pt}%
\newcommand{\CK}{\mathit{CK}}%
\begin{tabular}[t]{ll}
    \nicodemusbox{\markerlenndL}{%
        \textbf{Game} $\CORR_{\KEM}(\advA)$
        \begin{nicodemus}
            \item $\CK[\cdot]\gets\bot$
            \item $(\dk,\ek)\getsr\KEMgen$
            \item Invoke $\advA(\ek)$
            \item Stop with~$0$
        \end{nicodemus}%
        \medskip
        
        \textbf{Oracle} $\Oenc$
        \begin{nicodemus}
            \item $(k,c)\getsr\KEMenc(\ek)$
            \item $\CK[c]\gets k$
            \item Return~$c$
        \end{nicodemus}%
    }%
    &
    \nicodemusbox{\markerlenndR}{%
        \textbf{Oracle} $\Odec(c)$
        \begin{nicodemus}
            \item $k'\gets\KEMdec(\dk,c)$
            \item If $\CK[c]\notin\{k',\bot\}$:
            \item \quad Stop with $1$
            \item Return $k'$
        \end{nicodemus}%
    }%
\end{tabular}%%
        }%
    }
    \caption{%
        Game $\CORR$ for key encapsulation mechanism~$\KEM$.
    }
    \label{fig:kem:corr}
\end{figure}

\subsection{Adversarial Capabilities}
Defining \emph{security} can be divided into three components:
modeling the \emph{capabilities of adversaries},
specifying the \emph{adversaries' goal}, and
identifying attack strategies that are \emph{trivially} successful against every possible construction.
The former two components are typically influenced by intuitive purpose of the considered primitive and how this primitive is used in practice;
the latter one is almost entirely determined by all remaining definitional steps.

\subsubsection{Typical Capabilities}
We begin with considering typical adversarial capabilities against KEMs:
\begin{itemize}
    \item \textbf{Chosen-Plaintext Attacks:}
    The adversary can see the ciphertexts of encapsulated keys
    \item \textbf{Chosen-Ciphertext Attacks:}
    The adversary can see the decapsulated keys for chosen ciphertexts
    \item \textbf{Secret Key Corruption:}
    The adversary can learn the entire decapsulation key
    \item \textbf{Leakage of Information during Algorithm Execution:}
    The adversary can learn secret bits of variables processed by the encapsulation or decapsulation algorithm
    \item \textbf{Subversion of Algorithms:}
    The adversary can modify the encapsulation or decapsulation algorithm
    \item Etc.
\end{itemize}

\paragraph{Chosen-Plaintext Attacks}
The most standard adversarial capabilities that are considered realistic and defendable for KEMs are chosen-plaintext and chosen-ciphertext attacks (CPA resp.\ CCA).

Consider the adversarial capabilities shown in Figure~\ref{fig:adv_capabilities}.
Here, the adversary can see the values in the dashed boxes, as well as send values (in red).
Observing the values $\ek$ and $c$ is possible by eavesdropping, but the fact that $k'$ is visible might be surprising at first.
This is because an active adversary can send $c'$ and observe certain \emph{side-channels} to learn information about $k'$.
E.g. the adversary can measure the time it takes to decapsulate $c'$ or send an invalid $c'$ and observe the error message.
To model this capability we provide the adversary with the ability to choose ciphertexts for decapsulation and provide it with the decryption oracle (CCA).

\begin{figure}[!ht]
    \centering
    %!TEX root=../main.tex
\begin{tabular}{lcl}
    \underline{Alice} & \underline{$\advA$} & \underline{Bob}\\
    $(k,c)\getsr\KEMenc(\ek)$ & \arrbox{2cm}{$\xleftarrow{\dbox{$\ek$}}$} & $(\dk,\ek)\getsr\KEMgen$\\
    & $\xrightarrow{\dbox{$c$}}$ & $k\gets\KEMdec(\dk,c)$\\
    & {\color{red}$\xrightarrow{c'}$} & $\dbox{$k'$}\gets\KEMdec(\dk,c')$
\end{tabular}%
    \caption{%
        A sketch of common adversarial capabilities.
    }
    \label{fig:adv_capabilities}
\end{figure}

\paragraph{Chosen-Ciphertext Attacks}
The game $\INDCCA$ is defined in Figure~\ref{fig:kem:ind}.

\begin{figure}[!ht]
    \centering
    \nicoresetlinenr%
    \fbox{%
        \scalebox{\codescalefactor}{%
            %!TEX root=../main.tex
\markersetlen{ndL}{110pt}%
\markersetlen{ndR}{120pt}%
\newcommand{\CC}{\mathit{CC}}%
\begin{tabular}[t]{ll}
    \nicodemusbox{\markerlenndL}{%
        \textbf{Game} $\INDCCA_{\KEM}^b(\advA)$
        \begin{nicodemus}
            \item $\CC\gets\emptyset$
            \item $(\dk,\ek)\getsr\KEMgen$
            \item $b'\getsr\advA(\ek)$
            \item Stop with~$b'$
        \end{nicodemus}%
    }%
    &
    \nicodemusbox{\markerlenndR}{%
        \textbf{Oracle} $\Ochall$
        \begin{nicodemus}
            \item $(k_0,c)\getsr\KEMenc(\ek)$
            \item $k_1\getsr\ksp$
            \item $\CC\gets\CC\cup\{c\}$
            \item Return~$(k_b,c)$
        \end{nicodemus}%
        \medskip
        
        \textbf{Oracle} $\Odec(c)$
        \begin{nicodemus}
            \item Require $c\notin\CC$
            \item $k\gets\KEMdec(\dk,c)$
            \item Return $k$
        \end{nicodemus}%
    }%
\end{tabular}%%
        }%
    }
    \caption{%
        Games $\INDCCA$ for key encapsulation mechanism~$\KEM$.
    }
    \label{fig:kem:ind}
\end{figure}

\subsection{Adversarial Goal}
The adversarial goal is the second component of defining security.
It describes what the adversary is trying to learn. 
The adversarial goal is typically denoted in the first part of the name of the security game (e.g. \emph{IND}-CPA or \emph{OW}-CCA)

\subsubsection{Typical Goals}
For KEMs, the typical adversarial goals are either indistinguishability of ciphertexts or one-way security.

\paragraph{One-Way Security}
One-way security~(OW-security) is the intuitive definition of the adversarial goal. 
For a KEM, OW-security defines this to be learning the encapsulated key $k$.
The security game $\OWCCA$ is defined in Figure~\ref{fig:kem:ow:cca} and the $\OWCPA$ game follows from it by excluding the decryption oracle.

\begin{figure}[!ht]
    \centering
    \nicoresetlinenr%
    \fbox{%
        \scalebox{\codescalefactor}{%
            %!TEX root=../main.tex
\markersetlen{ndL}{110pt}%
\markersetlen{ndR}{120pt}%
\newcommand{\CC}{\mathit{CC}}%
\begin{tabular}[t]{ll}
    \nicodemusbox{\markerlenndL}{%
        \textbf{Game} $\OWCCA_{\KEM}(\advA)$
        \begin{nicodemus}
            \item $\CC\gets\emptyset$
            \item $K\gets\emptyset$
            \item $(\dk,\ek)\getsr\KEMgen$
            \item $k'\getsr\advA(\ek)$
            \item Stop with~$k'\in K$
        \end{nicodemus}%
    }%
    &
    \nicodemusbox{\markerlenndR}{%
        \textbf{Oracle} $\Ochall$
        \begin{nicodemus}
            \item $(k,c)\getsr\KEMenc(\ek)$
            \item $K\gets K\cup\{k\}$
            \item $\CC\gets\CC\cup\{c\}$
            \item Return~$(c)$
        \end{nicodemus}%
        \medskip
        
        \textbf{Oracle} $\Odec(c)$
        \begin{nicodemus}
            \item Require $c\notin\CC$
            \item $k\gets\KEMdec(\dk,c)$
            \item Return $k$
        \end{nicodemus}%
    }%
\end{tabular}%%
        }%
    }
    \caption{%
        One-Way CCA security game for $\KEM$.
    }
    \label{fig:kem:ow:cca}
\end{figure}

\paragraph{Indistinguishability of Ciphertexts}
As we have seen in Section~\ref{sec:overview:pke}, we sometimes want \emph{indistinguishability} of the generated ciphertext and a randomly chosen ciphertext.
The reason for this is that it gives us a stronger security guarantee. 
Namely, if a construction is IND secure, the adversary is unable to learn a single bit of the secret.
The $\INDCCA$ game is defined in Figure~\ref{fig:kem:ind}

\subsection{Trivial Winning Strategies}\label{sec:kem:trivial_attacks}
So far, we have done some unexplained bookkeeping in our security games. The reason for that are \emph{trivial winning strategies}.
These are strategies or attacks that work against every correct construction, simply because the constructions have to give sensible responses.
For our KEM there are two trivial winning conditions we need to guard against (Figure~\ref{kem:triv}).

\begin{figure}[!ht]%
    \centering
    %!TEX root=../main.tex
\parbox[t]{3.5cm}{%
    $\OWCCA$:
    \begin{enumerate}[topsep=0pt]
        \item $c\getsr\Ochall$
        \item $k\gets\Odec(c)$
        \item Stop with $k$
    \end{enumerate}    
}\parbox[t]{3.5cm}{%
    $\INDCCA$:
    \begin{enumerate}[topsep=0pt]
        \item $(k,c)\getsr\Ochall$
        \item $k'\gets\Odec(c)$
        \item Stop with $k\not= k'$
    \end{enumerate}
} 
    \caption{Trivial winning strategies against $\OWCCA$ and $\INDCCA$ $\KEM$ games.}
    \label{kem:triv}
\end{figure}

Now that the trivial winning conditions are defined, we can specify the advantage terms for each game.
For the OW-CPA and OW-CCA games that is:
\begin{align*}
    \Adv_\KEM^\owcpa(\advA)\coloneqq&\Pr[\OWCPA_{\KEM}(\advA)=1]\\
    \Adv_\KEM^\owcca(\advA)\coloneqq&\Pr[\OWCCA_{\KEM}(\advA)=1]
\end{align*}
And the advantages for the IND games are defined as:
\begin{align*}
    \Adv_\KEM^\indcpa(\advA)\coloneqq&\left|\Pr[\INDCPA_{\KEM}^0(\advA)=1]-\Pr[\INDCPA_{\KEM}^1(\advA)=1]\right|\\
    \Adv_\KEM^\indcca(\advA)\coloneqq&\left|\Pr[\INDCCA_{\KEM}^0(\advA)=1]-\Pr[\INDCCA_{\KEM}^1(\advA)=1]\right|
\end{align*}

If there exists a provably secure construction for a definition, then there are no more open trivial winning conditions.

\paragraph{Multi-Instance Security}
Definitions of security, such as the IND-CCA game, differ depending on the context.
One example of that would be the multi-instance security game (Figure~\ref{fig:kem:ind:mi:corrupt}).
This game is called \emph{multi-instance} because it captures a setting in which many instances are running concurrently.
This can be the case in TLS for example, where many connections are established at the same time.
If we were to use the IND-CCA game from Figure~\ref{fig:kem:ind}, then we would have to use the \emph{hybrid argument} to prove security.
The hybrid argument can be used to prove security for multiple instances, by showing step by step that each instance is secure and then showing the indistinguishability between each step.
Doing so however, yields a factor $q$ in the final security bound, where $q$ is the number of instances.
To avoid this degeneration of the security bound, we can use the multi-instance game, for which sometimes tighter security bounds exist.

\begin{figure}[!ht]
    \centering
    \nicoresetlinenr%
    \fbox{%
        \scalebox{\codescalefactor}{%
            %!TEX root=../main.tex
\markersetlen{ndL}{130pt}%
\markersetlen{ndR}{120pt}%
\newcommand{\CC}{\mathit{CC}}%
\newcommand{\CR}{\mathit{CR}}%
\begin{tabular}[t]{ll}
    \nicodemusbox{\markerlenndL}{%
        \textbf{Game} $\INDCCA_{\KEM}^b(\advA)$
        \begin{nicodemus}
            \item $n\gets0$
            \item $\CR\gets\emptyset$
            \item $b'\getsr\advA$
            \item Require $\forall i\in\CR:\CC_i=\emptyset$
            \item Stop with~$b'$
        \end{nicodemus}%
        \medskip
        
        \textbf{Oracle} $\Ogen$
        \begin{nicodemus}
            \item $\CC_n\gets\emptyset$
            \item $(\dk_n,\ek_n)\getsr\KEMgen$
            \item $n\gets n+1$
            \item Return $\ek_{n-1}$
        \end{nicodemus}%
    }%
    &
    \nicodemusbox{\markerlenndR}{%
        \textbf{Oracle} $\Ochall(i)$
        \begin{nicodemus}
            \item Require $i\in[n]$
            \item $(k_0,c)\getsr\KEMenc(\ek_i)$
            \item $k_1\getsr\ksp$
            \item $\CC_i\gets\CC_i\cup\{c\}$
            \item Return~$(k_b,c)$
        \end{nicodemus}%
        \medskip
        
        \textbf{Oracle} $\Odec(i,c)$
        \begin{nicodemus}
            \item Require $i\in[n]\land c\notin\CC_i$
            \item $k\gets\KEMdec(\dk_i,c)$
            \item Return $k$
        \end{nicodemus}%
        \medskip
        
        \textbf{Oracle} $\Ocorrupt(i)$
        \begin{nicodemus}
            \item Require $i\in[n]$
            \item $\CR\gets\CR\cup i$
            \item Return $\dk$
        \end{nicodemus}%
    }%
\end{tabular}%%
        }%
    }
    \caption{%
        Multi-instance games $\INDCCA$ for key encapsulation mechanism~$\KEM$.
    }
    \label{fig:kem:ind:mi:corrupt}
\end{figure}

\section{Symmetric Primitives}


\subsection{Pseudo-Random Generator}


\paragraph{Syntax}
A pseudo-random generator $\PRG=\PRGf$ is an algorithm with key space~$\ksp$ and output space~$\rsp$:

\begin{itemize}
    \item $\PRGf: \ksp \to \rsp$
\end{itemize}

\paragraph{Security}
The advantage of an adversary~$\advA$ against pseudo-random generator $\PRG$ in game $\IND$ from Figure~\ref{fig:prg:ind} is defined as:
\[
\Adv_\PRG^\ind(\advA)\coloneqq\left|\Pr[\IND_{\PRG}^0(\advA)=1]-\Pr[\IND_{\PRG}^1(\advA)=1]\right|\text{.}
\]

\begin{figure}[!ht]
    \centering
    \nicoresetlinenr%
    \fbox{%
        \scalebox{\codescalefactor}{%
            %!TEX root=../main.tex
\markersetlen{ndL}{110pt}%
\newcommand{\CC}{\mathit{CC}}%
\begin{tabular}[t]{l}
    \nicodemusbox{\markerlenndL}{%
        \textbf{Game} $\IND_{\PRG}^b(\advA)$
        \begin{nicodemus}
            \item $k\getsr\ksp$
            \item $r_0\gets\PRGf(k)$
            \item $r_1\getsr\rsp$
            \item $b'\getsr\advA(r_b)$
            \item Stop with~$b'$
        \end{nicodemus}%
    }%
\end{tabular}%%
        }%
    }
    \caption{%
        Games $\IND$ for pseudo-random generator~$\PRG$.
    }
    \label{fig:prg:ind}
\end{figure}

\subsection{Pseudo-Random Function}

\paragraph{Syntax}
A pseudo-random function $\PRF=\PRFf$ is an algorithm with key space~$\ksp$, input space~$\msp$ and output space~$\rsp$:

\begin{itemize}
    \item $\PRFf: \ksp\times\msp \to \rsp$
\end{itemize}

\paragraph{Security}
The advantage of an adversary~$\advA$ against pseudo-random function $\PRF$ in game $\IND$ from Figure~\ref{fig:prf:ind} is defined as:
\[
\Adv_\PRF^\ind(\advA)\coloneqq\left|\Pr[\IND_{\PRF}^0(\advA)=1]-\Pr[\IND_{\PRF}^1(\advA)=1]\right|\text{.}
\]

\begin{figure}[!ht]
    \centering
    \nicoresetlinenr%
    \fbox{%
        \scalebox{\codescalefactor}{%
            %!TEX root=../main.tex
\markersetlen{ndL}{110pt}%
\markersetlen{ndR}{120pt}%
\newcommand{\CC}{\mathit{CC}}%
\begin{tabular}[t]{ll}
    \nicodemusbox{\markerlenndL}{%
        \textbf{Game} $\IND_{\PRG}^b(\advA)$
        \begin{nicodemus}
            \item $R[\cdot]\gets\bot$
            \item $k\getsr\ksp$
            \item $b'\getsr\advA$
            \item Stop with~$b'$
        \end{nicodemus}%
    }%
    &
    \nicodemusbox{\markerlenndR}{%
        \textbf{Oracle} $\Ochall(m)$
        \begin{nicodemus}
            \item $r_0\gets\PRFf(k,m)$
            \item If $R[m]=\bot$: $R[m]\getsr\rsp$
            \item $r_1\gets R[m]$
            \item Return~$r_b$
        \end{nicodemus}%
    }%
\end{tabular}%%
        }%
    }
    \caption{%
        Games $\IND$ for pseudo-random function~$\PRF$.
    }
    \label{fig:prf:ind}
\end{figure}

\subsection{Message Authentication Code (MAC)}

\paragraph{Syntax}
A message authentication code $\MAC=(\MACgen,\MACtag,\MACvfy)$ is a tuple of three algorithms with key space~$\ksp$, message space~$\msp$, and tag space~$\tsp$:

\begin{itemize}
    \item $\MACgen: \emptyset \tor \ksp$
    \item $\MACtag: \ksp\times\msp \tor \tsp$
    \item $\MACvfy: \ksp\times\msp\times\tsp \to \Bool$
\end{itemize}

\paragraph{Correctness}
A message authentication code $\MAC$ is correct if
\[
\Pr[\MACvfy(k,m,\MACtag(k,m))=\T\mid k\getsr\MACgen,m\getsr\msp]=1\text{.}
\]

\paragraph{Security: Strong Existential Unforgeability under Chosen-Message Attacks}
The advantage of an adversary~$\advA$ against message authentication code $\MAC$ in game $\SUFCMA$ from Figure~\ref{fig:mac:suf} is defined as:
\[
\Adv_\MAC^\sufcma(\advA)\coloneqq\Pr[\SUFCMA_{\MAC}(\advA)=1]\text{.}
\]

\begin{figure}[!ht]
    \centering
    \nicoresetlinenr%
    \fbox{%
        \scalebox{\codescalefactor}{%
            %!TEX root=../main.tex
\markersetlen{ndL}{110pt}%
\markersetlen{ndR}{120pt}%
\newcommand{\MT}{\mathit{MT}}%
\begin{tabular}[t]{ll}
    \nicodemusbox{\markerlenndL}{%
        \textbf{Game} $\SUFCMA_{\MAC}(\advA)$
        \begin{nicodemus}
            \item $\MT\gets\emptyset$
            \item $k\getsr\MACgen$
            \item Invoke $\advA$
            \item Stop with~$0$
        \end{nicodemus}%
        \medskip
        
        \textbf{Oracle} $\Otag(m)$
        \begin{nicodemus}
            \item $\mtag\getsr\MACtag(k,m)$
            \item $\MT\gets\MT\cup\{(m,\mtag)\}$
            \item Return $\mtag$
        \end{nicodemus}%
    }%
    &
    \nicodemusbox{\markerlenndR}{%
        \textbf{Oracle} $\Ovfy(m,\mtag)$
        \begin{nicodemus}
            \item $b\gets\MACvfy(k,m,\mtag)$
            \item If $(m,\mtag)\notin\MT\land b=\T$:
            \item \quad Stop with~$1$
            \item Return $b$
        \end{nicodemus}%
    }%
\end{tabular}%%
        }%
    }
    \caption{%
        Game $\SUFCMA$ for message authentication code~$\MAC$.
    }
    \label{fig:mac:suf}
\end{figure}


\subsection{Symmetric Encryption (SE)}

\subsubsection{Probabilistic SE}

\paragraph{Syntax}
A probabilistic symmetric encryption scheme $\SYE=(\SYEgen,\SYEenc,\SYEdec)$ is a tuple of three algorithms with key space~$\ksp$, message space~$\msp$, and ciphertext space~$\csp$:

\begin{itemize}
    \item $\SYEgen: \emptyset \tor \ksp$
    \item $\SYEenc: \ksp\times\msp \tor \csp$
    \item $\SYEdec: \ksp\times\csp \to \msp$
\end{itemize}

\paragraph{Correctness}
A symmetric encryption scheme $\SYE$ is correct if 
\[
\Pr[\SYEdec(k,\SYEenc(k,m))=m\mid k\getsr\SYEgen,m\getsr\msp]=1\text{.}
\]
Equivalently, a symmetric encryption scheme $\SYE$ is correct if $\Pr[\CORR_{\SYE}(\advA)=0]=1$ for all adversaries~$\advA$, where game~$\CORR$ is defined in Figure~\ref{fig:sym:enc:corr:prob}.

\begin{figure}[!ht]
    \centering
    \nicoresetlinenr%
    \fbox{%
        \scalebox{\codescalefactor}{%
            %!TEX root=../main.tex
\markersetlen{ndL}{100pt}%
\markersetlen{ndR}{100pt}%
\newcommand{\CM}{\mathit{CM}}%
\begin{tabular}[t]{ll}
    \nicodemusbox{\markerlenndL}{%
        \textbf{Game} $\CORR_{\SYE}(\advA)$
        \begin{nicodemus}
            \item $\CM[\cdot]\gets\bot$
            \item $k\getsr\SYEgen$
            \item Invoke $\advA$
            \item Stop with~$0$
        \end{nicodemus}%
        \medskip
        
        \textbf{Oracle} $\Oenc(m)$
        \begin{nicodemus}
            \item Require $m\in\msp$
            \item $c\getsr\SYEenc(k,m)$
            \item $\CM[c]\gets m$
            \item Return~$c$
        \end{nicodemus}%
    }%
    &
    \nicodemusbox{\markerlenndR}{%
        \textbf{Oracle} $\Odec(c)$
        \begin{nicodemus}
            \item $m'\gets\SYEdec(k,c)$
            \item If $\CM[c]\notin\{m',\bot\}$:
            \item \quad Stop with $1$
            \item Return $m'$
        \end{nicodemus}%
    }%
\end{tabular}%%
        }%
    }
    \caption{%
        Game $\CORR$ for probabilistic encryption scheme~$\SYE$.
    }
    \label{fig:sym:enc:corr:prob}
\end{figure}

\paragraph{Security: One-Wayness under Chosen-Ciphertext Attacks}
The advantage of an adversary~$\advA$ against symmetric encryption scheme $\SYE$ in game $\OWCCA$ from Figure~\ref{fig:sym:enc:ow:prob} is defined as:
\[
\Adv_\SYE^\owcca(\advA)\coloneqq\Pr[\OWCCA_{\SYE}(\advA)=1]\text{.}
\]

\begin{figure}[!ht]
    \centering
    \nicoresetlinenr%
    \fbox{%
        \scalebox{\codescalefactor}{%
            %!TEX root=../main.tex
\markersetlen{ndL}{100pt}%
\markersetlen{ndR}{130pt}%
\newcommand{\CM}{\mathit{CM}}%
\begin{tabular}[t]{ll}
    \nicodemusbox{\markerlenndL}{%
        \textbf{Game} $\OWCCA_{\SYE}(\advA)$
        \begin{nicodemus}
            \item $\CM\gets\emptyset$
            \item $k\getsr\SYEgen$
            \item $(c,m)\getsr\advA$
            \item If $(c,m)\in\CM$:
            \item \quad Stop with~$1$
            \item Stop with~$0$
        \end{nicodemus}%
        \medskip
        
        \textbf{Oracle} $\Oenc(m)$
        \begin{nicodemus}
            \item $c\getsr\SYEenc(k,m)$
            \item Return~$c$
        \end{nicodemus}%
    }%
    &
    \nicodemusbox{\markerlenndR}{%
        \textbf{Oracle} $\Ochall()$
        \begin{nicodemus}
            \item $m\getsr\msp$
            \item $c\getsr\SYEenc(k,m)$
            \item $\CM\gets\CM\cup\{(c,m)\}$
            \item Return~$c$
        \end{nicodemus}%
        \medskip

        \textbf{Oracle} $\Odec(c)$
        \begin{nicodemus}
            \item Require $\nexists m':(c,m')\in\CM$
            \item $m\gets\SYEdec(k,c)$
            \item Return $m$
        \end{nicodemus}%
    }%
\end{tabular}%%
        }%
    }
    \caption{%
        Game $\OWCCA$ for probabilistic symmetric encryption scheme~$\SYE$.
    }
    \label{fig:sym:enc:ow:prob}
\end{figure}

\paragraph{Security: Indistinguishability under Chosen-Ciphertext Attacks}
The advantage of an adversary~$\advA$ against symmetric encryption scheme $\SYE$ in game $\INDCCA$ from Figure~\ref{fig:sym:enc:ind:prob} is defined as:
\[
\Adv_\SYE^\indcca(\advA)\coloneqq\left|\Pr[\INDCCA_{\SYE}^0(\advA)=1]-\Pr[\INDCCA_{\SYE}^1(\advA)=1]\right|\text{.}
\]

\begin{figure}[!ht]
    \centering
    \nicoresetlinenr%
    \fbox{%
        \scalebox{\codescalefactor}{%
            %!TEX root=../main.tex
\markersetlen{ndL}{100pt}%
\markersetlen{ndR}{130pt}%
\newcommand{\CC}{\mathit{CC}}%
\begin{tabular}[t]{ll}
    \nicodemusbox{\markerlenndL}{%
        \textbf{Game} $\INDCCA_{\SYE}^b(\advA)$
        \begin{nicodemus}
            \item $\CC\gets\emptyset$
            \item $k\getsr\SYEgen$
            \item $b'\getsr\advA$
            \item Stop with~$b'$
        \end{nicodemus}%
        \medskip
        
        \textbf{Oracle} $\Oenc(m)$
        \begin{nicodemus}
            \item $c\getsr\SYEenc(k,m)$
            \item Return~$c$
        \end{nicodemus}%
    }%
    &
    \nicodemusbox{\markerlenndR}{%
        \textbf{Oracle} $\Ochall(m_0,m_1)$
        \begin{nicodemus}
            \item Require $\{m_0,m_1\}\subseteq\msp$
            \item $c\getsr\SYEenc(k,m_b)$
            \item $\CC\gets\CC\cup\{c\}$
            \item Return~$c$
        \end{nicodemus}%
        \medskip
        
        \textbf{Oracle} $\Odec(c)$
        \begin{nicodemus}
            \item Require $c\notin\CC$
            \item $m\gets\SYEdec(k,c)$
            \item Return $m$
        \end{nicodemus}%
    }%
\end{tabular}%%
        }%
    }
    \caption{%
        Games $\INDCCA$ for probabilistic symmetric encryption scheme~$\SYE$.
    }
    \label{fig:sym:enc:ind:prob}
\end{figure}


\subsubsection{Authenticated Encryption with Associated Data}

\paragraph{Syntax}
An authenticated encryption scheme with associated data $\AEAD=(\AEADgen,\AEADenc,\AEADdec)$ is a tuple of three algorithms with key space~$\ksp$, message space~$\msp$, associated-data space~$\adsp$, and ciphertext space~$\csp$:

\begin{itemize}
    \item $\AEADgen: \emptyset \tor \ksp$
    \item $\AEADenc: \ksp\times\msp\times\adsp \tor \csp$
    \item $\AEADdec: \ksp\times\adsp\times\csp \to \msp\cup\{\bot\}$
\end{itemize}

\paragraph{Correctness}
An authenticated encryption scheme with associated data $\AEAD$ is correct if 
\[
\Pr[\AEADdec(k,\ad,\AEADenc(k,m,\ad))=m\mid k\getsr\SYEgen,m\getsr\msp,\ad\getsr\adsp]=1\text{.}
\]
Equivalently, an authenticated encryption scheme with associated data $\AEAD$ is correct if $\Pr[\CORR_{\AEAD}(\advA)=0]=1$ for all adversaries~$\advA$, where game~$\CORR$ is defined in Figure~\ref{fig:sym:aenc:corr}.

We define the standard security notions $\INDD$ and $\SUFCMA$ adapted from the seminal work by Rogaway~\cite{CCS:Rogaway02} in the following paragraphs.

\begin{figure}[!ht]
    \centering
    \nicoresetlinenr%
    \fbox{%
        \scalebox{\codescalefactor}{%
            %!TEX root=../main.tex
\markersetlen{ndL}{120pt}%
\markersetlen{ndR}{120pt}%
\newcommand{\CM}{\mathit{CM}}%
\begin{tabular}[t]{ll}
    \nicodemusbox{\markerlenndL}{%
        \textbf{Game} $\CORR_{\AEAD}(\advA)$
        \begin{nicodemus}
            \item $\CM[\cdot]\gets\bot$
            \item $k\getsr\AEADgen$
            \item Invoke $\advA$
            \item Stop with~$0$
        \end{nicodemus}%
        \medskip
        
        \textbf{Oracle} $\Oenc(m,\ad)$
        \begin{nicodemus}
            \item Require $m\in\msp$
            \item $c\getsr\AEADenc(k,m,\ad)$
            \item $\CM[\ad,c]\gets m$
            \item Return~$c$
        \end{nicodemus}%
    }%
    &
    \nicodemusbox{\markerlenndR}{%
        \textbf{Oracle} $\Odec(c,\ad)$
        \begin{nicodemus}
            \item $m'\gets\AEADdec(k,\ad,c)$
            \item If $\CM[\ad,c]\notin\{m',\bot\}$:
            \item \quad Stop with $1$
            \item Return $m'$
        \end{nicodemus}%
    }%
\end{tabular}%%
        }%
    }
    \caption{%
        Game $\CORR$ for authenticated encryption scheme with associated data~$\AEAD$.
    }
    \label{fig:sym:aenc:corr}
\end{figure}

\paragraph{Security: Indistinguishability of Ciphertexts from Randomness under Chosen-Plaintext Attacks}
The advantage of an adversary~$\advA$ against authenticated encryption scheme with associated data $\AEAD$ in game $\INDD$ from Figure~\ref{fig:sym:aenc:indd} is defined as:
\[
\Adv_\AEAD^\indd(\advA)\coloneqq\left|\Pr[\INDD_{\AEAD}^0(\advA)=1]-\Pr[\INDD_{\AEAD}^1(\advA)=1]\right|\text{.}
\]

\begin{figure}[!ht]
    \centering
    \nicoresetlinenr%
    \fbox{%
        \scalebox{\codescalefactor}{%
            %!TEX root=../main.tex
\markersetlen{ndL}{120pt}%
\markersetlen{ndR}{120pt}%
\begin{tabular}[t]{ll}
    \nicodemusbox{\markerlenndL}{%
        \textbf{Game} $\INDD_{\AEAD}^b(\advA)$
        \begin{nicodemus}
            \item $k\getsr\AEADgen$
            \item $b'\getsr\advA$
            \item Stop with~$b'$
        \end{nicodemus}%
        \medskip
        
        \textbf{Oracle} $\Oenc(m,\ad)$
        \begin{nicodemus}
            \item $c\getsr\AEADenc(k,m,\ad)$
            \item Return~$c$
        \end{nicodemus}%
    }%
    &
    \nicodemusbox{\markerlenndR}{%
        \textbf{Oracle} $\Ochall(m,\ad)$
        \begin{nicodemus}
            \item Require $m\in\msp$
            \item $c_0\getsr\AEADenc(k,m,\ad)$
            \item $c_1\getsr\BB^{|c_0|}$
            \item Return~$c_b$
        \end{nicodemus}%
    }%
\end{tabular}%%
        }%
    }
    \caption{%
        Games $\INDD$ for authenticated encryption scheme with associated data~$\AEAD$.
    }
    \label{fig:sym:aenc:indd}
\end{figure}

\paragraph{Security: Strong Existential Unforgeability under Chosen-Message Attacks}
The advantage of an adversary~$\advA$ against authenticated encryption scheme with associated data $\AEAD$ in game $\SUFCMA$ from Figure~\ref{fig:sym:aenc:suf} is defined as:
\[
\Adv_\AEAD^\sufcma(\advA)\coloneqq\Pr[\SUFCMA_{\AEAD}^0(\advA)=1]\text{.}
\]

\begin{figure}[!ht]
    \centering
    \nicoresetlinenr%
    \fbox{%
        \scalebox{\codescalefactor}{%
            %!TEX root=../main.tex
\markersetlen{ndL}{120pt}%
\markersetlen{ndR}{130pt}%
\newcommand{\MC}{\mathit{MC}}%
\begin{tabular}[t]{ll}
    \nicodemusbox{\markerlenndL}{%
        \textbf{Game} $\SUFCMA_{\AEAD}(\advA)$
        \begin{nicodemus}
            \item $\MC\gets\emptyset$
            \item $k\getsr\AEADgen$
            \item Invoke $\advA$
            \item Stop with~$0$
        \end{nicodemus}%
        \medskip
        
        \textbf{Oracle} $\Oenc(m,\ad)$
        \begin{nicodemus}
            \item $c\getsr\AEADenc(k,m,\ad)$
            \item $\MC\gets\MC\cup\{(m,\ad,c)\}$
            \item Return $c$
        \end{nicodemus}%
    }%
    &
    \nicodemusbox{\markerlenndR}{%
        \textbf{Oracle} $\Odec(\ad,c)$
        \begin{nicodemus}
            \item $m\gets\AEADdec(k,\ad,c)$
            \item If $(m,\ad,c)\notin\MC\land m\neq\bot$:
            \item \quad Stop with~$1$
            \item Return $m$
        \end{nicodemus}%
    }%
\end{tabular}%%
        }%
    }
    \caption{%
        Game $\SUFCMA$ for authenticated encryption scheme with associated data~$\AEAD$.
    }
    \label{fig:sym:aenc:suf}
\end{figure}




\section{Asymmetric Primitives}
\label{sec:asym}

\subsection{Assumptions}
\label{sec:asym:assumptions}

\subsubsection{Diffie-Hellman (DH)}
\label{sec:asym:assumptions:dh}
Let $\DHgr=(\DHg,\DHp)$ be a group of prime order~$\DHp$ with generator~$\DHg$.

\paragraph{Discrete Logarithm}
The advantage of an adversary~$\advA$ against the discrete logarithm problem in group~$\DHgr$ is defined as:
\[
\Adv_\DHgr^\dlp(\advA)\coloneqq\Pr[\advA(\DHg,\DHp,\DHg^x)=x\mid x\getsr\ZZ_\DHp^*]\text{.}
\]


\paragraph{Computational DH}
The advantage of an adversary~$\advA$ against the computational Diffie-Hellman problem in group~$\DHgr$ is defined as:
\[
\Adv_\DHgr^\cdh(\advA)\coloneqq\Pr[\advA(\DHg,\DHp,\DHg^x,\DHg^y)=,\DHg^{xy}\mid (x,y)\getsr(\ZZ_\DHp^*)^2]\text{.}
\]

\paragraph{Decisional DH}
The advantage of an adversary~$\advA$ against the decisional Diffie-Hellman problem in group~$\DHgr$ is defined as:
\begin{align*}
    \Adv_\DHgr^\ddh(\advA)\coloneqq|&\Pr[\advA(\DHg,\DHp,\DHg^x,\DHg^y,\DHg^{xy})=1\mid (x,y)\getsr(\ZZ_\DHp^*)^2]\\
    &-\Pr[\advA(\DHg,\DHp,\DHg^x,\DHg^y,\DHg^z)=1\mid (x,y,z)\getsr(\ZZ_\DHp^*)^3]|\text{.}
\end{align*}

\subsection{Digital Signature}

\paragraph{Syntax}
A digital signature scheme $\SIG=(\SIGgen,\SIGsig,\SIGvfy)$ is a tuple of three algorithms with signing key space~$\sksp$, verification key space~$\vksp$, message space~$\msp$, and signature space~$\sigsp$:

\begin{itemize}
    \item $\SIGgen: \emptyset \tor \sksp\times\vksp$
    \item $\SIGsig: \sksp\times\msp \tor \sigsp$
    \item $\SIGvfy: \vksp\times\msp\times\sigsp \to \Bool$
\end{itemize}

\paragraph{Correctness}
A digital signature scheme $\SIG$ is correct if
\[
\Pr[\SIGvfy(\vk,m,\SIGsig(\sk,m))=\T\mid (\sk,\vk)\getsr\SIGgen,m\getsr\msp]=1\text{.}
\]

\paragraph{Security}
The advantage of an adversary~$\advA$ against digital signature scheme $\SIG$ in game $\SUFCMA$ from Figure~\ref{fig:sig:suf} is defined as:
\[
\Adv_\SIG^\sufcma(\advA)\coloneqq\Pr[\SUFCMA_{\SIG}(\advA)=1]\text{.}
\]

\begin{figure}[!ht]
    \centering
    \nicoresetlinenr%
    \fbox{%
        \scalebox{\codescalefactor}{%
            %!TEX root=../main.tex
\markersetlen{ndL}{110pt}%
\markersetlen{ndR}{120pt}%
\newcommand{\MS}{\mathit{MS}}%
\begin{tabular}[t]{ll}
    \nicodemusbox{\markerlenndL}{%
        \textbf{Game} $\SUFCMA_{\SIG}(\advA)$
        \begin{nicodemus}
            \item $\MS\gets\emptyset$
            \item $(\sk,\vk)\getsr\SIGgen$
            \item Invoke $\advA(\vk)$
            \item Stop with~$0$
        \end{nicodemus}%
        \medskip
        
        \textbf{Oracle} $\Osig(m)$
        \begin{nicodemus}
            \item $\sig\getsr\SIGsig(\sk,m)$
            \item $\MS\gets\MS\cup\{(m,\sig)\}$
            \item Return $\sig$
        \end{nicodemus}%
    }%
    &
    \nicodemusbox{\markerlenndR}{%
        \textbf{Oracle} $\Ovfy(m,\sig)$
        \begin{nicodemus}
            \item $b\gets\SIGvfy(k,m,\sig)$
            \item If $(m,\sig)\notin\MS\land b=\T$:
            \item \quad Stop with~$1$
            \item Return $b$
        \end{nicodemus}%
    }%
\end{tabular}%%
        }%
    }
    \caption{%
        Game $\SUFCMA$ for digital signature scheme~$\SIG$.
    }
    \label{fig:sig:suf}
\end{figure}

\subsection{Identity-Based KEM}

\paragraph{Syntax}
An identity-based KEM $\IKEM=(\IKEMgen,\IKEMenc,\IKEMdel,\IKEMdec)$ is a tuple of four algorithms with encapsulation key space~$\eksp$, decapsulation key space~$\dksp$, identity space~$\idsp$, ciphertext space~$\csp$, and symmetric key space~$\ksp$:

\begin{itemize}
    \item $\IKEMgen: \emptyset \tor \dksp\times\eksp$
    \item $\IKEMenc: \eksp\times(\idsp)^+ \tor \ksp\times\csp$
    \item $\IKEMdel: \dksp\times\idsp \tor \dksp$
    \item $\IKEMdec: \dksp\times\csp \to \ksp$
\end{itemize}


\section{Stateful Primitives}

\subsection{Forward-Secure KEM}

\paragraph{Syntax}
A forward-secure KEM $\FKEM=(\FKEMgen,\FKEMenc,\FKEMdec)$ is a tuple of three algorithms with encapsulation key space~$\eksp$, decapsulation key space~$\dksp$, ciphertext space~$\csp$, and symmetric key space~$\ksp$:

\begin{itemize}
    \item $\FKEMgen: \emptyset \tor \dksp\times\eksp$
    \item $\FKEMenc: \eksp \tor \ksp\times\csp$
    \item $\FKEMdec: \dksp\times\csp \to \dksp\times\ksp$
\end{itemize}

\paragraph{Correctness}
A forward-secure KEM $\FKEM$ is correct if $\Pr[\CORR_{\FKEM}(\advA)=0]=1$ for all adversaries~$\advA$, where game~$\CORR$ is defined in Figure~\ref{fig:fkem:corr}.

\begin{figure}[!ht]
    \centering
    \nicoresetlinenr%
    \fbox{%
        \scalebox{\codescalefactor}{%
            %!TEX root=../main.tex
\markersetlen{ndL}{120pt}%
\markersetlen{ndR}{130pt}%
\newcommand{\CK}{\mathit{CK}}%
\begin{tabular}[t]{ll}
    \nicodemusbox{\markerlenndL}{%
        \textbf{Game} $\CORR_{\FKEM}(\advA)$
        \begin{nicodemus}
            \item $\CK[\cdot]\gets\bot$
            \item $(\dk,\ek)\getsr\FKEMgen$
            \item Invoke $\advA(\ek)$
            \item Stop with~$0$
        \end{nicodemus}%
        \medskip
        
        \textbf{Oracle} $\Oenc$
        \begin{nicodemus}
            \item $(k,c)\getsr\FKEMenc(\ek)$
            \item If $\CK[c]\neq\diamond$: $\CK[c]\gets k$
            \item Return~$c$
        \end{nicodemus}%
    }%
    &
    \nicodemusbox{\markerlenndR}{%
        \textbf{Oracle} $\Odec(c)$
        \begin{nicodemus}
            \item $(\dk,k')\gets\FKEMdec(\dk,c)$
            \item If $\CK[c]\notin\{k',\bot,\diamond\}$:
            \item \quad Stop with $1$
            \item If $\CK[c]\neq\bot$: $\CK[c]\gets \diamond$
            \item Return $k'$
        \end{nicodemus}%
    }%
\end{tabular}%%
        }%
    }
    \caption{%
        Game $\CORR$ for forward-secure KEM~$\FKEM$.
    }
    \label{fig:fkem:corr}
\end{figure}

\paragraph{Security}
The advantage of an adversary~$\advA$ against forward-secure KEM $\FKEM$ in game $\INDCCA$ from Figure~\ref{fig:fkem:ind} is defined as:
\[
\Adv_\FKEM^\indcca(\advA)\coloneqq\left|\Pr[\INDCCA_{\FKEM}^0(\advA)=1]-\Pr[\INDCCA_{\FKEM}^1(\advA)=1]\right|\text{.}
\]

\begin{figure}[!ht]
    \centering
    \nicoresetlinenr%
    \fbox{%
        \scalebox{\codescalefactor}{%
            %!TEX root=../main.tex
\markersetlen{ndL}{130pt}%
\markersetlen{ndR}{130pt}%
\newcommand{\CC}{\mathit{CC}}%
\begin{tabular}[t]{ll}
    \nicodemusbox{\markerlenndL}{%
        \textbf{Game} $\INDCCA_{\FKEM}^b(\advA)$
        \begin{nicodemus}
            \item $\CC\gets\emptyset$
            \item $(\dk,\ek)\getsr\FKEMgen$
            \item $b'\getsr\advA(\ek)$
            \item Stop with~$b'$
        \end{nicodemus}%
        \medskip
        
        \textbf{Oracle} $\Ochall$
        \begin{nicodemus}
            \item $(k_0,c)\getsr\FKEMenc(\ek)$
            \item $k_1\getsr\ksp$
            \item $\CC\gets\CC\cup\{c\}$
            \item Return~$(k_b,c)$
        \end{nicodemus}%
    }%
    &
    \nicodemusbox{\markerlenndR}{%
        \textbf{Oracle} $\Odec(c)$
        \begin{nicodemus}
            \item $(\dk,k)\gets\FKEMdec(\dk,c)$
            \item If $c\in\CC$: $k\gets\bot$
            \item $\CC\gets\CC\setminus\{c\}$
            \item Return $k$
        \end{nicodemus}%
        \medskip
        
        \textbf{Oracle} $\Ocorrupt$
        \begin{nicodemus}
            \item Require $\CC=\emptyset$
            \item Return~$\dk$
        \end{nicodemus}%
    }%
\end{tabular}%%
        }%
    }
    \caption{%
        Games $\INDCCA$ for forward-secure KEM~$\FKEM$.
    }
    \label{fig:fkem:ind}
\end{figure}

\paragraph{Construction}


\subsection{Updatable KEM}

\paragraph{Syntax}
An updatable KEM $\UKEM=(\UKEMgen,\UKEMenc,\UKEMdec)$ is a tuple of three algorithms with encapsulation key space~$\eksp$, decapsulation key space~$\dksp$, ciphertext space~$\csp$, and symmetric key space~$\ksp$:

\begin{itemize}
    \item $\UKEMgen: \emptyset \tor \dksp\times\eksp$
    \item $\UKEMenc: \eksp \tor \eksp\times\ksp\times\csp$
    \item $\UKEMdec: \dksp\times\csp \to \dksp\times\ksp$
\end{itemize}

\paragraph{Construction}


\subsection{Key Exchange}

\subsubsection{Two-Pass Key Exchange}

\paragraph{Syntax}
A two-pass key exchange $\TKE=(\TKEsnd,\TKErsp,\TKErcv)$ is a tuple of three algorithms with state space~$\stsp$, initial ciphertext space~$\csp_I$, response ciphertext space~$\csp_R$, and symmetric key space~$\ksp$:

\begin{itemize}
    \item $\TKEsnd: \emptyset \tor \stsp\times\csp_I$
    \item $\TKErsp: \csp_I \tor \ksp\times\csp_R$
    \item $\TKErcv: \stsp\times\csp_R \to \ksp$
\end{itemize}

\paragraph{Construction: DH Key Exchange}

\paragraph{Construction: X3DH}

\subsubsection{Authenticated Key Exchange}

\paragraph{Syntax}
An authenticated key exchange $\AKE=(\AKEgen,\allowbreak\AKEsnd,\allowbreak\AKErsp_R,\allowbreak\AKErsp_I,\allowbreak\AKErcv)$ is a tuple of five algorithms with secret key space~$\sksp$, public key space~$\pksp$, state space~$\stsp$, ciphertext space~$\csp$, and symmetric key space~$\ksp$:

\begin{itemize}
    \item $\AKEgen: \emptyset \tor \sksp\times\pksp$
    \item $\AKEsnd: \sksp\times\pksp \tor \stsp\times\csp$
    \item $\AKErsp_R: \sksp\times\csp \tor \stsp\times\csp$
    \item $\AKErsp_I: \sksp\times\stsp\times\csp \tor \ksp\times\csp$
    \item $\AKErcv: \sksp\times\stsp\times\csp \to \pksp\times\ksp$
\end{itemize}

\paragraph{Construction: Signature-Based Authentication}

\paragraph{Construction: KEM-Based Authentication}

\paragraph{Construction: TLS}

\paragraph{Construction: Noise Framework}


\subsection{Ratcheted Key Exchange (RKE)}

\subsubsection{Unidirectional RKE}

\paragraph{Syntax}
A unidirectional key exchange $\URKE=(\URKEinit,\URKEsnd,\URKErcv)$ is a tuple of three algorithms with state space~$\stsp$, ciphertext space~$\csp$, and symmetric key space~$\ksp$:

\begin{itemize}
    \item $\URKEinit: \emptyset \tor \stsp\times\stsp$
    \item $\URKEsnd: \stsp \tor \stsp\times\ksp\times\csp$
    \item $\URKErcv: \stsp\times\csp \to \ksp$
\end{itemize}

\paragraph{Construction}

\subsubsection{Sesquidirectional RKE}

\subsubsection{Bidirectional RKE}

\paragraph{Construction: Double Ratchet}

\subsubsection{Group RKE}

\paragraph{Construction: Sender Key Mechanism}

\paragraph{Construction: Tree-Based DH}

\paragraph{Construction: Tree KEM}

\bibliography{cryptobib/abbrev3,cryptobib/crypto}

\end{document}