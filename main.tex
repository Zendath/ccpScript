\documentclass[a4paper,orivec]{llncs}

\newcommand{\alert}[1]{{\color{red}#1}}

\usepackage{fullpage}
\pagestyle{plain}
\setcounter{tocdepth}{2}

%!TEX root=main.tex

\usepackage{lmodern}
\usepackage[T1]{fontenc}
\usepackage[utf8]{inputenc}

\usepackage{amsfonts,amsmath,amssymb,mathtools}

\usepackage{graphicx}

\usepackage{array}

\usepackage{color,xcolor}
\definecolor{gray}{gray}{0.5}
\definecolor{darkblue}{rgb}{0,0,0.5}
\definecolor{darkgreen}{rgb}{0,0.5,0}
\usepackage[colorlinks=true,linkcolor=darkblue,urlcolor=darkblue,citecolor=darkgreen,pdftitle={Cryptographic Communication Protocols: Key Exchange and Channels}]{hyperref}

\usepackage[strings]{underscore}

\newcommand{\codescalefactor}{0.8}
\newcommand{\tikzscalefactor}{0.8}

\usepackage{nicodemus}
\usepackage{bpmarker}

%draw dashed boxes
\usepackage{dashbox}
\setlength{\dashlength}{4pt}
\setlength{\dashdash}{2pt}

%tikz and libraries
\usepackage{tikz}
\usetikzlibrary{arrows.meta, positioning, calc, trees, shapes, intersections, tikzmark, decorations.pathmorphing}

%some symbols (e.g. lightning bolt)
\usepackage{marvosym}

%placing correlated figures next to each other
\usepackage{subcaption}

%nicer in-line fractions (used for in line case distinction for functions with multiple possible outputs)
%https://mirror.physik.tu-berlin.de/pub/CTAN/macros/latex/contrib/xfrac/xfrac.pdf
\usepackage{xfrac}
%!TEX root=main.tex

\newlength\itemizeskip%used for aligning text in itemize

\let\oldparagraph=\paragraph
\renewcommand\paragraph[1]{\oldparagraph{#1.}}

\newcolumntype{C}[1]{>{\centering\arraybackslash\hspace{0pt}}p{#1}}

%provides \intertext analogue for lists
%\listintertext*{text} inserts text flush with left margin
%\listintertext{text} inserts text flush with left margin of the list environment one higher (if it exists)
%copied from https://tex.stackexchange.com/questions/135726/intertext-like-command-in-enumerate-environment
\makeatletter
\newcommand{\listintertext}{\@ifstar\listintertext@\listintertext@@}
\newcommand{\listintertext@}[1]{% \listintertext*{#1}
  \hspace*{-\@totalleftmargin}#1}
\newcommand{\listintertext@@}[1]{% \listintertext{#1}
  \hspace{-\leftmargin}#1}
\makeatother


\newcommand{\algbox}[2]{\fbox{\parbox{#1}{#2}}}
\newcommand{\arrbox}[2]{\parbox{#1}{\centering#2}}

\newcommand{\getsr}{\gets_\$}
\newcommand{\tor}{\to_\$}
\newcommand{\es}{\epsilon}
\newcommand{\T}{\mathtt{tru}}
\newcommand{\F}{\mathtt{fal}}
\newcommand{\secp}{\kappa}

\newcommand{\ZZ}{\mathbb{Z}}
\newcommand{\BB}{\{0,1\}}
\newcommand{\Bool}{\{\T,\F\}}
\newcommand{\NN}{\mathbb{N}}
\newcommand{\PS}{\mathcal{P}}

\newcommand{\advA}{\mathcal{A}}
\newcommand{\advB}{\mathcal{B}}
\newcommand{\advC}{\mathcal{C}}
\newcommand{\advD}{\mathcal{D}}
\newcommand{\Adv}{\mathrm{Adv}}

\newcommand{\APKE}{\mathrm{APKE}}
\renewcommand{\AE}{\mathrm{AE}}
\newcommand{\IKE}{\mathrm{IKE}}


%VARIABLES
\newcommand{\sk}{\mathit{sk}}
\newcommand{\st}{\mathit{st}}
\newcommand{\vk}{\mathit{vk}}
\newcommand{\pk}{\mathit{pk}}
\newcommand{\ek}{\mathit{ek}}
\newcommand{\dk}{\mathit{dk}}
\newcommand{\rk}{\mathit{rk}}
\newcommand{\ck}{\mathit{ck}}
\newcommand{\mk}{\mathit{mk}}
\newcommand{\mtag}{\tau}
\newcommand{\sig}{\sigma}
\newcommand{\ad}{\mathit{ad}}

%SPACES
\newcommand{\sksp}{\mathcal{SK}}
\newcommand{\pksp}{\mathcal{PK}}
\newcommand{\vksp}{\mathcal{VK}}
\newcommand{\eksp}{\mathcal{EK}}
\newcommand{\dksp}{\mathcal{DK}}
\newcommand{\stsp}{\mathcal{ST}}
\newcommand{\ksp}{\mathcal{K}}
\newcommand{\rsp}{\mathcal{R}}
\newcommand{\msp}{\mathcal{M}}
\newcommand{\csp}{\mathcal{C}}
\newcommand{\tsp}{\mathcal{T}}
\newcommand{\sigsp}{\Sigma}
\newcommand{\adsp}{\mathcal{AD}}
\newcommand{\idsp}{\mathcal{ID}}

%ORACLES
\newcommand{\ROh}{\mathrm{H}}

\newcommand{\CORR}{\mathrm{CORR}}
\newcommand{\IND}{{\mathrm{IND}}}
\newcommand{\ind}{{\mathrm{ind}}}
\newcommand{\INDD}{{\mathrm{IND}\$}}
\newcommand{\indd}{{\mathrm{ind}\$}}
\newcommand{\INDCCA}{{\mathrm{IND}\text{-}\mathrm{CCA}}}
\newcommand{\indcca}{{\mathrm{ind}\text{-}\mathrm{cca}}}
\newcommand{\INDCPA}{{\mathrm{IND}\text{-}\mathrm{CPA}}}
\newcommand{\indcpa}{{\mathrm{ind}\text{-}\mathrm{cpa}}}
\newcommand{\OWCCA}{{\mathrm{OW}\text{-}\mathrm{CCA}}}
\newcommand{\owcca}{{\mathrm{ow}\text{-}\mathrm{cca}}}
\newcommand{\OWCPA}{{\mathrm{OW}\text{-}\mathrm{CPA}}}
\newcommand{\owcpa}{{\mathrm{ow}\text{-}\mathrm{cpa}}}
\newcommand{\SUFCMA}{{\mathrm{SUF}\text{-}\mathrm{CMA}}}
\newcommand{\sufcma}{{\mathrm{suf}\text{-}\mathrm{cma}}}
\newcommand{\ANON}{\mathrm{ANON}}

\newcommand{\Ogen}{\mathrm{Gen}}
\newcommand{\Oinit}{\mathrm{Init}}
\newcommand{\Oenc}{\mathrm{Enc}}
\newcommand{\Odec}{\mathrm{Dec}}
\newcommand{\Oup}{\mathrm{Up}}
\newcommand{\OupPK}{\mathrm{UpPK}}
\newcommand{\OupSK}{\mathrm{UpSK}}

\newcommand{\Otag}{\mathrm{Tag}}
\newcommand{\Osig}{\mathrm{Sign}}
\newcommand{\Ovfy}{\mathrm{Vfy}}

\newcommand{\Osnd}{\mathrm{Snd}}
\newcommand{\Orsp}{\mathrm{Rsp}}
\newcommand{\Orcv}{\mathrm{Rcv}}
\newcommand{\Ochall}{\mathrm{Chall}}

\newcommand{\Ocorrupt}{\mathrm{Corrupt}}
\newcommand{\Oexp}{\mathrm{Expose}}
\newcommand{\Oexps}{\mathrm{ExposeS}}
\newcommand{\Oexpr}{\mathrm{ExposeR}}
\newcommand{\Oreveal}{\mathrm{Reveal}}



\iffalse
\newcommand{\full}{\!$\cdot$\,}
\newcommand{\core}{\hphantom{\!$\cdot$\,}}

\newcommand{\getscup}{\overset{\raisebox{-1pt}{\tiny$\;\cup$}}{\gets}}
\newcommand{\getsconcat}{\overset{\raisebox{-1pt}{\tiny$\;\shortparallel$}}{\gets}} % "concat assignment"

\newcommand{\bO}{\mathcal{O}}
\newcommand{\secp}{\lambda}

\fi

%DH
\newcommand{\DHgr}{\mathbb{G}}
\newcommand{\DHg}{g}
\newcommand{\DHp}{p}

\newcommand{\dlp}{\mathrm{dlp}}
\newcommand{\cdh}{\mathrm{cdh}}
\newcommand{\ddh}{\mathrm{ddh}}

%PRG and PRF
\newcommand{\PRF}{\mathrm{PRF}}
\newcommand{\PRG}{\mathrm{PRG}}

\newcommand{\PRGf}{\PRG.\mathrm{f}}
\newcommand{\PRFf}{\PRF.\mathrm{f}}

%KDF
\newcommand{\KDF}{\mathrm{KDF}}

%MAC and DIGITAL SIGNATURE
\newcommand{\MAC}{\mathrm{MAC}}
\newcommand{\SIG}{\mathrm{SIG}}

\newcommand{\MACgen}{\MAC.\mathrm{gen}}
\newcommand{\MACtag}{\MAC.\mathrm{tag}}
\newcommand{\MACvfy}{\MAC.\mathrm{vfy}}

\newcommand{\SIGgen}{\SIG.\mathrm{gen}}
\newcommand{\SIGsig}{\SIG.\mathrm{sig}}
\newcommand{\SIGvfy}{\SIG.\mathrm{vfy}}

%SYMMETRIC ENCRYPTION, PUBLIC-KEY ENCRYPTION, various KEMs, and KEY EXCHANGE
\newcommand{\SYE}{\mathrm{SE}}
\newcommand{\AEAD}{\mathrm{AEAD}}
\newcommand{\PKE}{\mathrm{PKE}}
\newcommand{\KEM}{\mathrm{KEM}}
\newcommand{\IKEM}{\mathrm{IKEM}}
\newcommand{\FKEM}{\mathrm{FKEM}}
\newcommand{\UKEM}{\mathrm{UKEM}}
\newcommand{\KUKEM}{\mathrm{KU}\text{-}\mathrm{KEM}} %not sure if this is the best way to typeset the "-" in KU-KEM 
\newcommand{\TKE}{\mathrm{TKE}}
\newcommand{\AKE}{\mathrm{AKE}}
\newcommand{\URKE}{\mathrm{URKE}}
\newcommand{\HIBE}{\mathrm{HIBE}}

\newcommand{\AEADgen}{\AEAD.\mathrm{gen}}
\newcommand{\AEADenc}{\AEAD.\mathrm{enc}}
\newcommand{\AEADdec}{\AEAD.\mathrm{dec}}

\newcommand{\SYEgen}{\SYE.\mathrm{gen}}
\newcommand{\SYEenc}{\SYE.\mathrm{enc}}
\newcommand{\SYEdec}{\SYE.\mathrm{dec}}

\newcommand{\PKEgen}{\PKE.\mathrm{gen}}
\newcommand{\PKEenc}{\PKE.\mathrm{enc}}
\newcommand{\PKEdec}{\PKE.\mathrm{dec}}

\newcommand{\KEMgen}{\KEM.\mathrm{gen}}
\newcommand{\KEMenc}{\KEM.\mathrm{enc}}
\newcommand{\KEMdec}{\KEM.\mathrm{dec}}

\newcommand{\IKEMgen}{\IKEM.\mathrm{gen}}
\newcommand{\IKEMenc}{\IKEM.\mathrm{enc}}
\newcommand{\IKEMdel}{\IKEM.\mathrm{del}}
\newcommand{\IKEMdec}{\IKEM.\mathrm{dec}}

\newcommand{\UKEMgen}{\UKEM.\mathrm{gen}}
\newcommand{\UKEMenc}{\UKEM.\mathrm{enc}}
\newcommand{\UKEMdec}{\UKEM.\mathrm{dec}}

\newcommand{\KUKEMgen}{\KUKEM.\mathrm{gen}}
\newcommand{\KUKEMenc}{\KUKEM.\mathrm{enc}}
\newcommand{\KUKEMdec}{\KUKEM.\mathrm{dec}}
\newcommand{\KUKEMup}{\KUKEM.\mathrm{up}}

\newcommand{\FKEMgen}{\FKEM.\mathrm{gen}}
\newcommand{\FKEMenc}{\FKEM.\mathrm{enc}}
\newcommand{\FKEMdec}{\FKEM.\mathrm{dec}}

\newcommand{\TKEsnd}{\TKE.\mathrm{snd}}
\newcommand{\TKErsp}{\TKE.\mathrm{rsp}}
\newcommand{\TKErcv}{\TKE.\mathrm{rcv}}

\newcommand{\AKEgen}{\AKE.\mathrm{gen}}
\newcommand{\AKEsnd}{\AKE.\mathrm{snd}}
\newcommand{\AKErsp}{\AKE.\mathrm{rsp}}
\newcommand{\AKErcv}{\AKE.\mathrm{rcv}}

\newcommand{\URKEinit}{\URKE.\mathrm{init}}
\newcommand{\URKEsnd}{\URKE.\mathrm{snd}}
\newcommand{\URKErcv}{\URKE.\mathrm{rcv}}

\newcommand{\HIBEgen}{\HIBE.\mathrm{gen}}
\newcommand{\HIBEenc}{\HIBE.\mathrm{enc}}
\newcommand{\HIBEdec}{\HIBE.\mathrm{dec}}
\newcommand{\HIBEdel}{\HIBE.\mathrm{del}}


\bibliographystyle{alpha}

\makeatletter
\renewcommand*\l@author[2]{}
\renewcommand*\l@title[2]{}
\makeatletter
\renewcommand{\contentsname}{}


\title{Cryptographic Communication Protocols:\\Key Exchange and Channels}
\author{Paul Rösler}

\institute{FAU Erlangen-Nürnberg}

\begin{document}

\maketitle
\begin{center}
    \today
\end{center}

\begingroup
\let\clearpage\relax
\tableofcontents
\endgroup

\section{Preliminary Remarks}
This document is not (yet) a full course script.
Instead, it is meant as an additional resource that systematizes the considered primitives, definitions, and constructions.
The author invites readers to submit comments or pull requests via the GitHub Repository \url{https://github.com/roeslpa/ccpScript}.


\section{Notation}

\begin{tabular}{|l|p{13cm}|}\hline
    \textbf{Symbol} & \textbf{Meaning}\\\hline
    $\T$ / $\F$ & Boolean values true and false\\
    $\gets$ / $\to$ & Assigns a constant expression or a deterministic algorithm\\
    $\getsr$ / $\tor$ & Assigns a random value uniformly sampled from a finite set or the output of a probabilistic algorithm\\
    $\PS(X)$ & Power set of set~$X$\\
    $(X)^+$ & Set of non-trivial concatenations of elements from set~$X$\\
    $X[\cdot]\gets x$ & Assigns all entries of array~$X$ with default value~$x$\\
    Invoke $X$ & Executes algorithm $X$\\
    Stop with $x$ & Terminates the running experiment with final output~$x$\\
    Require $x$ & Ends the algorithm, resp.~oracle, if expression~$x=\T$\\\hline
\end{tabular}

\section{Game-Based Definitions}

\subsection{Syntax}

\subsection{Correctness}

\subsection{Adversarial Capabilities}

\subsubsection{Typical Capabilities}

\paragraph{Chosen-Plaintext Attacks}

\paragraph{Chosen-Ciphertext Attacks}

\paragraph{Replayable Chosen-Ciphertext Attacks}

\paragraph{Exposure of Secrets}

\paragraph{Compromised Randomness}

\subsection{Adversarial Goal}

\subsubsection{Typical Goals}

\paragraph{One-Way Security}

\paragraph{Indistinguishability of Ciphertexts}

\subsection{Trivial Winning Strategies}


\section{Symmetric Primitives}


\subsection{Pseudo-Random Generator}


\paragraph{Syntax}
A pseudo-random generator $\PRG=\PRGf$ is an algorithm with key space~$\ksp$ and output space~$\rsp$:

\begin{itemize}
    \item $\PRGf: \ksp \to \rsp$
\end{itemize}

\subsection{Pseudo-Random Function}

\paragraph{Syntax}
A pseudo-random function $\PRF=\PRFf$ is an algorithm with key space~$\ksp$, input space~$\msp$ and output space~$\rsp$:

\begin{itemize}
    \item $\PRGf: \ksp\times\msp \to \rsp$
\end{itemize}

\subsection{Message Authentication Code (MAC)}

\paragraph{Syntax}
A message authentication code $\MAC=(\MACgen,\MACtag,\MACvfy)$ is a tuple of three algorithms with key space~$\ksp$, message space~$\msp$, and tag space~$\tsp$:

\begin{itemize}
    \item $\MACgen: \emptyset \tor \ksp$
    \item $\MACtag: \ksp\times\msp \tor \tsp$
    \item $\MACvfy: \ksp\times\msp\times\tsp \to \Bool$
\end{itemize}

\paragraph{Correctness}
A message authentication code $\MAC$ is correct if
\[
\Pr[\MACvfy(k,m,\MACtag(k,m))=\T\mid k\getsr\MACgen,m\getsr\msp]=1\text{.}
\]

\paragraph{Security: Strong Existential Unforgeability under Chosen-Message Attacks}
The advantage of an adversary~$\advA$ against message authentication code $\MAC$ in game $\SUFCMA$ from Figure~\ref{fig:mac:suf} is defined as:
\[
\Adv_\MAC^\sufcma(\advA)\coloneqq\Pr[\SUFCMA_{\MAC}(\advA)=1]\text{.}
\]

\begin{figure}[!ht]
    \centering
    \nicoresetlinenr%
    \fbox{%
        \scalebox{\codescalefactor}{%
            %!TEX root=../main.tex
\markersetlen{ndL}{110pt}%
\markersetlen{ndR}{120pt}%
\newcommand{\MT}{\mathit{MT}}%
\begin{tabular}[t]{ll}
    \nicodemusbox{\markerlenndL}{%
        \textbf{Game} $\SUFCMA_{\MAC}(\advA)$
        \begin{nicodemus}
            \item $\MT\gets\emptyset$
            \item $k\getsr\MACgen$
            \item Invoke $\advA$
            \item Stop with~$0$
        \end{nicodemus}%
        \medskip
        
        \textbf{Oracle} $\Otag(m)$
        \begin{nicodemus}
            \item $\mtag\getsr\MACtag(k,m)$
            \item $\MT\gets\MT\cup\{(m,\mtag)\}$
            \item Return $\mtag$
        \end{nicodemus}%
    }%
    &
    \nicodemusbox{\markerlenndR}{%
        \textbf{Oracle} $\Ovfy(m,\mtag)$
        \begin{nicodemus}
            \item $b\gets\MACvfy(k,m,\mtag)$
            \item If $(m,\mtag)\notin\MT\land b=\T$:
            \item \quad Stop with~$1$
            \item Return $b$
        \end{nicodemus}%
    }%
\end{tabular}%%
        }%
    }
    \caption{%
        Game $\SUFCMA$ for message authentication code~$\MAC$.
    }
    \label{fig:mac:suf}
\end{figure}


\subsection{Symmetric Encryption (SE)}

\subsubsection{Probabilistic SE}

\paragraph{Syntax}
A probabilistic symmetric encryption scheme $\SYE=(\SYEgen,\SYEenc,\SYEdec)$ is a tuple of three algorithms with key space~$\ksp$, message space~$\msp$, and ciphertext space~$\csp$:

\begin{itemize}
    \item $\SYEgen: \emptyset \tor \ksp$
    \item $\SYEenc: \ksp\times\msp \tor \csp$
    \item $\SYEdec: \ksp\times\csp \to \msp$
\end{itemize}

\paragraph{Correctness}
A symmetric encryption scheme $\SYE$ is correct if 
\[
\Pr[\SYEdec(k,\SYEenc(k,m))=m\mid k\getsr\SYEgen,m\getsr\msp]=1\text{.}
\]
Equivalently, a symmetric encryption scheme $\SYE$ is correct if $\Pr[\CORR_{\SYE}(\advA)=0]=1$ for all adversaries~$\advA$, where game~$\CORR$ is defined in Figure~\ref{fig:sym:enc:corr:prob}.

\begin{figure}[!ht]
    \centering
    \nicoresetlinenr%
    \fbox{%
        \scalebox{\codescalefactor}{%
            %!TEX root=../main.tex
\markersetlen{ndL}{100pt}%
\markersetlen{ndR}{100pt}%
\newcommand{\CM}{\mathit{CM}}%
\begin{tabular}[t]{ll}
    \nicodemusbox{\markerlenndL}{%
        \textbf{Game} $\CORR_{\SYE}(\advA)$
        \begin{nicodemus}
            \item $\CM[\cdot]\gets\bot$
            \item $k\getsr\SYEgen$
            \item Invoke $\advA$
            \item Stop with~$0$
        \end{nicodemus}%
        \medskip
        
        \textbf{Oracle} $\Oenc(m)$
        \begin{nicodemus}
            \item Require $m\in\msp$
            \item $c\getsr\SYEenc(k,m)$
            \item $\CM[c]\gets m$
            \item Return~$c$
        \end{nicodemus}%
    }%
    &
    \nicodemusbox{\markerlenndR}{%
        \textbf{Oracle} $\Odec(c)$
        \begin{nicodemus}
            \item $m'\gets\SYEdec(k,c)$
            \item If $\CM[c]\notin\{m',\bot\}$:
            \item \quad Stop with $1$
            \item Return $m'$
        \end{nicodemus}%
    }%
\end{tabular}%%
        }%
    }
    \caption{%
        Game $\CORR$ for probabilistic encryption scheme~$\SYE$.
    }
    \label{fig:sym:enc:corr:prob}
\end{figure}

\paragraph{Security: One-Wayness under Chosen-Ciphertext Attacks}
The advantage of an adversary~$\advA$ against symmetric encryption scheme $\SYE$ in game $\OWCCA$ from Figure~\ref{fig:sym:enc:ow:prob} is defined as:
\[
\Adv_\SYE^\owcca(\advA)\coloneqq\Pr[\OWCCA_{\SYE}(\advA)=1]\text{.}
\]

\begin{figure}[!ht]
    \centering
    \nicoresetlinenr%
    \fbox{%
        \scalebox{\codescalefactor}{%
            %!TEX root=../main.tex
\markersetlen{ndL}{100pt}%
\markersetlen{ndR}{130pt}%
\newcommand{\CM}{\mathit{CM}}%
\begin{tabular}[t]{ll}
    \nicodemusbox{\markerlenndL}{%
        \textbf{Game} $\OWCCA_{\SYE}(\advA)$
        \begin{nicodemus}
            \item $\CM\gets\emptyset$
            \item $k\getsr\SYEgen$
            \item $(c,m)\getsr\advA$
            \item If $(c,m)\in\CM$:
            \item \quad Stop with~$1$
            \item Stop with~$0$
        \end{nicodemus}%
        \medskip
        
        \textbf{Oracle} $\Oenc(m)$
        \begin{nicodemus}
            \item $c\getsr\SYEenc(k,m)$
            \item Return~$c$
        \end{nicodemus}%
    }%
    &
    \nicodemusbox{\markerlenndR}{%
        \textbf{Oracle} $\Ochall()$
        \begin{nicodemus}
            \item $m\getsr\msp$
            \item $c\getsr\SYEenc(k,m)$
            \item $\CM\gets\CM\cup\{(c,m)\}$
            \item Return~$c$
        \end{nicodemus}%
        \medskip

        \textbf{Oracle} $\Odec(c)$
        \begin{nicodemus}
            \item Require $\nexists m':(c,m')\in\CM$
            \item $m\gets\SYEdec(k,c)$
            \item Return $m$
        \end{nicodemus}%
    }%
\end{tabular}%%
        }%
    }
    \caption{%
        Game $\OWCCA$ for probabilistic symmetric encryption scheme~$\SYE$.
    }
    \label{fig:sym:enc:ow:prob}
\end{figure}

\paragraph{Security: Indistinguishability under Chosen-Ciphertext Attacks}
The advantage of an adversary~$\advA$ against symmetric encryption scheme $\SYE$ in game $\INDCCA$ from Figure~\ref{fig:sym:enc:ind:prob} is defined as:
\[
\Adv_\SYE^\indcca(\advA)\coloneqq\left|\Pr[\INDCCA_{\SYE}^0(\advA)=1]-\Pr[\INDCCA_{\SYE}^1(\advA)=1]\right|\text{.}
\]

\begin{figure}[!ht]
    \centering
    \nicoresetlinenr%
    \fbox{%
        \scalebox{\codescalefactor}{%
            %!TEX root=../main.tex
\markersetlen{ndL}{100pt}%
\markersetlen{ndR}{130pt}%
\newcommand{\CC}{\mathit{CC}}%
\begin{tabular}[t]{ll}
    \nicodemusbox{\markerlenndL}{%
        \textbf{Game} $\INDCCA_{\SYE}^b(\advA)$
        \begin{nicodemus}
            \item $\CC\gets\emptyset$
            \item $k\getsr\SYEgen$
            \item $b'\getsr\advA$
            \item Stop with~$b'$
        \end{nicodemus}%
        \medskip
        
        \textbf{Oracle} $\Oenc(m)$
        \begin{nicodemus}
            \item $c\getsr\SYEenc(k,m)$
            \item Return~$c$
        \end{nicodemus}%
    }%
    &
    \nicodemusbox{\markerlenndR}{%
        \textbf{Oracle} $\Ochall(m_0,m_1)$
        \begin{nicodemus}
            \item Require $\{m_0,m_1\}\subseteq\msp$
            \item $c\getsr\SYEenc(k,m_b)$
            \item $\CC\gets\CC\cup\{c\}$
            \item Return~$c$
        \end{nicodemus}%
        \medskip
        
        \textbf{Oracle} $\Odec(c)$
        \begin{nicodemus}
            \item Require $c\notin\CC$
            \item $m\gets\SYEdec(k,c)$
            \item Return $m$
        \end{nicodemus}%
    }%
\end{tabular}%%
        }%
    }
    \caption{%
        Games $\INDCCA$ for probabilistic symmetric encryption scheme~$\SYE$.
    }
    \label{fig:sym:enc:ind:prob}
\end{figure}



\section{Asymmetric Primitives}


\subsection{Assumptions}

\subsubsection{Diffie-Hellman (DH)}
Let $\DHgr=(\DHg,\DHp)$ be a group of prime order~$\DHp$ with generator~$\DHg$.

\paragraph{Discrete Logarithm}
The advantage of an adversary~$\advA$ against the discrete logarithm problem in group~$\DHgr$ is defined as:
\[
\Adv_\DHgr^\dlp(\advA)\coloneqq\Pr[\advA(\DHg,\DHp,\DHg^x)=x\mid x\getsr\ZZ_\DHp^*]\text{.}
\]


\paragraph{Computational DH}
The advantage of an adversary~$\advA$ against the computational Diffie-Hellman problem in group~$\DHgr$ is defined as:
\[
\Adv_\DHgr^\cdh(\advA)\coloneqq\Pr[\advA(\DHg,\DHp,\DHg^x,\DHg^y)=,\DHg^{xy}\mid (x,y)\getsr(\ZZ_\DHp^*)^2]\text{.}
\]

\paragraph{Decisional DH}
The advantage of an adversary~$\advA$ against the decicional Diffie-Hellman problem in group~$\DHgr$ is defined as:
\begin{align*}
    \Adv_\DHgr^\ddh(\advA)\coloneqq|&\Pr[\advA(\DHg,\DHp,\DHg^x,\DHg^y,\DHg^{xy})=1\mid (x,y)\getsr(\ZZ_\DHp^*)^2]\\
    &-\Pr[\advA(\DHg,\DHp,\DHg^x,\DHg^y,\DHg^z)=1\mid (x,y,z)\getsr(\ZZ_\DHp^*)^3]|\text{.}
\end{align*}

\subsection{Digital Signature}

\paragraph{Syntax}
A digital signature scheme $\SIG=(\SIGgen,\SIGsig,\SIGvfy)$ is a tuple of three algorithms with signing key space~$\sksp$, verification key space~$\vksp$, message space~$\msp$, and signature space~$\sigsp$:

\begin{itemize}
    \item $\SIGgen: \emptyset \tor \sksp\times\vksp$
    \item $\SIGsig: \sksp\times\msp \tor \sigsp$
    \item $\SIGvfy: \vksp\times\msp\times\sigsp \to \Bool$
\end{itemize}

\paragraph{Correctness}
A digital signature scheme $\SIG$ is correct if
\[
\Pr[\SIGvfy(\vk,m,\SIGsig(\sk,m))=\T\mid (\sk,\vk)\getsr\SIGgen,m\getsr\msp]=1\text{.}
\]

\paragraph{Security}
The advantage of an adversary~$\advA$ against digital signature scheme $\SIG$ in game $\SUFCMA$ from Figure~\ref{fig:sig:suf} is defined as:
\[
\Adv_\SIG^\sufcma(\advA)\coloneqq\Pr[\SUFCMA_{\SIG}(\advA)=1]\text{.}
\]

\begin{figure}[!ht]
    \centering
    \nicoresetlinenr%
    \fbox{%
        \scalebox{\codescalefactor}{%
            %!TEX root=../main.tex
\markersetlen{ndL}{110pt}%
\markersetlen{ndR}{120pt}%
\newcommand{\MS}{\mathit{MS}}%
\begin{tabular}[t]{ll}
    \nicodemusbox{\markerlenndL}{%
        \textbf{Game} $\SUFCMA_{\SIG}(\advA)$
        \begin{nicodemus}
            \item $\MS\gets\emptyset$
            \item $(\sk,\vk)\getsr\SIGgen$
            \item Invoke $\advA(\vk)$
            \item Stop with~$0$
        \end{nicodemus}%
        \medskip
        
        \textbf{Oracle} $\Osig(m)$
        \begin{nicodemus}
            \item $\sig\getsr\SIGsig(\sk,m)$
            \item $\MS\gets\MS\cup\{(m,\sig)\}$
            \item Return $\sig$
        \end{nicodemus}%
    }%
    &
    \nicodemusbox{\markerlenndR}{%
        \textbf{Oracle} $\Ovfy(m,\sig)$
        \begin{nicodemus}
            \item $b\gets\SIGvfy(k,m,\sig)$
            \item If $(m,\sig)\notin\MS\land b=\T$:
            \item \quad Stop with~$1$
            \item Return $b$
        \end{nicodemus}%
    }%
\end{tabular}%%
        }%
    }
    \caption{%
        Game $\SUFCMA$ for digital signature scheme~$\SIG$.
    }
    \label{fig:sig:suf}
\end{figure}

\subsection{Public-Key Encryption (PKE)}

\paragraph{Syntax}
A public-key encryption scheme $\PKE=(\PKEgen,\PKEenc,\PKEdec)$ is a tuple of three algorithms with encryption key space~$\eksp$, decryption key space~$\dksp$, message space~$\msp$, and ciphertext space~$\csp$:

\begin{itemize}
    \item $\PKEgen: \emptyset \tor \dksp\times\eksp$
    \item $\PKEenc: \eksp\times\msp \tor \csp$
    \item $\PKEdec: \dksp\times\csp \to \msp$
\end{itemize}

\paragraph{Correctness}
A public-key encryption scheme $\PKE$ is correct if
\[
\Pr[\PKEdec(\dk,\PKEenc(\ek,m))=m\mid (\dk,\ek)\getsr\PKEgen,m\getsr\msp]=1\text{.}
\]

\paragraph{Security}
The advantage of an adversary~$\advA$ against public-key encryption scheme $\PKE$ in game $\INDCCA$ from Figure~\ref{fig:pke:ind} is defined as:
\[
\Adv_\PKE^\indcca(\advA)\coloneqq\left|\Pr[\INDCCA_{\PKE}^0(\advA)=1]-\Pr[\INDCCA_{\PKE}^1(\advA)=1]\right|\text{.}
\]

\begin{figure}[!ht]
    \centering
    \nicoresetlinenr%
    \fbox{%
        \scalebox{\codescalefactor}{%
            %!TEX root=../main.tex
\markersetlen{ndL}{110pt}%
\markersetlen{ndR}{120pt}%
\newcommand{\CC}{\mathit{CC}}%
\begin{tabular}[t]{ll}
    \nicodemusbox{\markerlenndL}{%
        \textbf{Game} $\INDCCA_{\PKE}^b(\advA)$
        \begin{nicodemus}
            \item $\CC\gets\emptyset$
            \item $(\dk,\ek)\getsr\PKEgen$
            \item $b'\getsr\advA(\ek)$
            \item Stop with~$b'$
        \end{nicodemus}%
    }%
    &
    \nicodemusbox{\markerlenndR}{%
        \textbf{Oracle} $\Ochall(m_0,m_1)$
        \begin{nicodemus}
            \item Require $\{m_0,m_1\}\subseteq\msp$
            \item $c\getsr\PKEenc(\ek,m_b)$
            \item $\CC\gets\CC\cup\{c\}$
            \item Return~$c$
        \end{nicodemus}%
        \medskip
        
        \textbf{Oracle} $\Odec(c)$
        \begin{nicodemus}
            \item Require $c\notin\CC$
            \item $m\gets\PKEdec(\dk,c)$
            \item Return $m$
        \end{nicodemus}%
    }%
\end{tabular}%%
        }%
    }
    \caption{%
        Games $\INDCCA$ for public-key encryption scheme~$\PKE$.
    }
    \label{fig:pke:ind}
\end{figure}

\subsection{Key Encapsulation Mechanism (KEM)}

\paragraph{Syntax}
A key encapsulation mechanism $\KEM=(\KEMgen,\KEMenc,\KEMdec)$ is a tuple of three algorithms with encapsulation key space~$\eksp$, decapsulation key space~$\dksp$, ciphertext space~$\csp$, and symmetric key space~$\ksp$:

\begin{itemize}
    \item $\KEMgen: \emptyset \tor \dksp\times\eksp$
    \item $\KEMenc: \eksp \tor \ksp\times\csp$
    \item $\KEMdec: \dksp\times\csp \to \ksp$
\end{itemize}

\paragraph{Correctness}
A key encapsulation mechanism $\KEM$ is correct if
\[
\Pr[\KEMdec(\dk,c)=k\mid (\dk,\ek)\getsr\KEMgen,(k,c)\getsr\KEMenc(\ek))=m]=1\text{.}
\]
Equivalently, a key encapsulation mechanism $\KEM$ is correct if $\Pr[\CORR_{\KEM}(\advA)=0]=1$ for all adversaries~$\advA$, where game~$\CORR$ is defined in Figure~\ref{fig:kem:corr}.

\begin{figure}[!ht]
    \centering
    \nicoresetlinenr%
    \fbox{%
        \scalebox{\codescalefactor}{%
            %!TEX root=../main.tex
\markersetlen{ndL}{110pt}%
\markersetlen{ndR}{100pt}%
\newcommand{\CK}{\mathit{CK}}%
\begin{tabular}[t]{ll}
    \nicodemusbox{\markerlenndL}{%
        \textbf{Game} $\CORR_{\KEM}(\advA)$
        \begin{nicodemus}
            \item $\CK[\cdot]\gets\bot$
            \item $(\dk,\ek)\getsr\KEMgen$
            \item Invoke $\advA(\ek)$
            \item Stop with~$0$
        \end{nicodemus}%
        \medskip
        
        \textbf{Oracle} $\Oenc$
        \begin{nicodemus}
            \item $(k,c)\getsr\KEMenc(\ek)$
            \item $\CK[c]\gets k$
            \item Return~$c$
        \end{nicodemus}%
    }%
    &
    \nicodemusbox{\markerlenndR}{%
        \textbf{Oracle} $\Odec(c)$
        \begin{nicodemus}
            \item $k'\gets\KEMdec(\dk,c)$
            \item If $\CK[c]\notin\{k',\bot\}$:
            \item \quad Stop with $1$
            \item Return $k'$
        \end{nicodemus}%
    }%
\end{tabular}%%
        }%
    }
    \caption{%
        Game $\CORR$ for key encapsulation mechanism~$\KEM$.
    }
    \label{fig:kem:corr}
\end{figure}

\paragraph{Security}
The advantage of an adversary~$\advA$ against key encapsulation mechanism $\KEM$ in game $\INDCCA$ from Figure~\ref{fig:kem:ind} is defined as:
\[
\Adv_\KEM^\indcca(\advA)\coloneqq\left|\Pr[\INDCCA_{\KEM}^0(\advA)=1]-\Pr[\INDCCA_{\KEM}^1(\advA)=1]\right|\text{.}
\]

\begin{figure}[!ht]
    \centering
    \nicoresetlinenr%
    \fbox{%
        \scalebox{\codescalefactor}{%
            %!TEX root=../main.tex
\markersetlen{ndL}{110pt}%
\markersetlen{ndR}{120pt}%
\newcommand{\CC}{\mathit{CC}}%
\begin{tabular}[t]{ll}
    \nicodemusbox{\markerlenndL}{%
        \textbf{Game} $\INDCCA_{\KEM}^b(\advA)$
        \begin{nicodemus}
            \item $\CC\gets\emptyset$
            \item $(\dk,\ek)\getsr\KEMgen$
            \item $b'\getsr\advA(\ek)$
            \item Stop with~$b'$
        \end{nicodemus}%
    }%
    &
    \nicodemusbox{\markerlenndR}{%
        \textbf{Oracle} $\Ochall$
        \begin{nicodemus}
            \item $(k_0,c)\getsr\KEMenc(\ek)$
            \item $k_1\getsr\ksp$
            \item $\CC\gets\CC\cup\{c\}$
            \item Return~$(k_b,c)$
        \end{nicodemus}%
        \medskip
        
        \textbf{Oracle} $\Odec(c)$
        \begin{nicodemus}
            \item Require $c\notin\CC$
            \item $k\gets\KEMdec(\dk,c)$
            \item Return $k$
        \end{nicodemus}%
    }%
\end{tabular}%%
        }%
    }
    \caption{%
        Games $\INDCCA$ for key encapsulation mechanism~$\KEM$.
    }
    \label{fig:kem:ind}
\end{figure}

\paragraph{Construction: ElGamal KEM}


\subsection{Identity-Based KEM}

\paragraph{Syntax}
An identity-based KEM $\IKEM=(\IKEMgen,\IKEMenc,\IKEMdel,\IKEMdec)$ is a tuple of four algorithms with encapsulation key space~$\eksp$, decapsulation key space~$\dksp$, identity space~$\idsp$, ciphertext space~$\csp$, and symmetric key space~$\ksp$:

\begin{itemize}
    \item $\IKEMgen: \emptyset \tor \dksp\times\eksp$
    \item $\IKEMenc: \eksp\times(\idsp)^+ \tor \ksp\times\csp$
    \item $\IKEMdel: \dksp\times\idsp \tor \dksp$
    \item $\IKEMdec: \dksp\times\csp \to \ksp$
\end{itemize}


\section{Stateful Primitives}

\subsection{Forward-Secure KEM}

\paragraph{Syntax}
A forward-secure KEM $\FKEM=(\FKEMgen,\FKEMenc,\FKEMdec)$ is a tuple of three algorithms with encapsulation key space~$\eksp$, decapsulation key space~$\dksp$, ciphertext space~$\csp$, and symmetric key space~$\ksp$:

\begin{itemize}
    \item $\FKEMgen: \emptyset \tor \dksp\times\eksp$
    \item $\FKEMenc: \eksp \tor \ksp\times\csp$
    \item $\FKEMdec: \dksp\times\csp \to \dksp\times\ksp$
\end{itemize}

\paragraph{Correctness}
A forward-secure KEM $\FKEM$ is correct if $\Pr[\CORR_{\FKEM}(\advA)=0]=1$ for all adversaries~$\advA$, where game~$\CORR$ is defined in Figure~\ref{fig:fkem:corr}.

\begin{figure}[!ht]
    \centering
    \nicoresetlinenr%
    \fbox{%
        \scalebox{\codescalefactor}{%
            %!TEX root=../main.tex
\markersetlen{ndL}{120pt}%
\markersetlen{ndR}{130pt}%
\newcommand{\CK}{\mathit{CK}}%
\begin{tabular}[t]{ll}
    \nicodemusbox{\markerlenndL}{%
        \textbf{Game} $\CORR_{\FKEM}(\advA)$
        \begin{nicodemus}
            \item $\CK[\cdot]\gets\bot$
            \item $(\dk,\ek)\getsr\FKEMgen$
            \item Invoke $\advA(\ek)$
            \item Stop with~$0$
        \end{nicodemus}%
        \medskip
        
        \textbf{Oracle} $\Oenc$
        \begin{nicodemus}
            \item $(k,c)\getsr\FKEMenc(\ek)$
            \item If $\CK[c]\neq\diamond$: $\CK[c]\gets k$
            \item Return~$c$
        \end{nicodemus}%
    }%
    &
    \nicodemusbox{\markerlenndR}{%
        \textbf{Oracle} $\Odec(c)$
        \begin{nicodemus}
            \item $(\dk,k')\gets\FKEMdec(\dk,c)$
            \item If $\CK[c]\notin\{k',\bot,\diamond\}$:
            \item \quad Stop with $1$
            \item If $\CK[c]\neq\bot$: $\CK[c]\gets \diamond$
            \item Return $k'$
        \end{nicodemus}%
    }%
\end{tabular}%%
        }%
    }
    \caption{%
        Game $\CORR$ for forward-secure KEM~$\FKEM$.
    }
    \label{fig:fkem:corr}
\end{figure}

\paragraph{Security}
The advantage of an adversary~$\advA$ against forward-secure KEM $\FKEM$ in game $\INDCCA$ from Figure~\ref{fig:fkem:ind} is defined as:
\[
\Adv_\FKEM^\indcca(\advA)\coloneqq\left|\Pr[\INDCCA_{\FKEM}^0(\advA)=1]-\Pr[\INDCCA_{\FKEM}^1(\advA)=1]\right|\text{.}
\]

\begin{figure}[!ht]
    \centering
    \nicoresetlinenr%
    \fbox{%
        \scalebox{\codescalefactor}{%
            %!TEX root=../main.tex
\markersetlen{ndL}{130pt}%
\markersetlen{ndR}{130pt}%
\newcommand{\CC}{\mathit{CC}}%
\begin{tabular}[t]{ll}
    \nicodemusbox{\markerlenndL}{%
        \textbf{Game} $\INDCCA_{\FKEM}^b(\advA)$
        \begin{nicodemus}
            \item $\CC\gets\emptyset$
            \item $(\dk,\ek)\getsr\FKEMgen$
            \item $b'\getsr\advA(\ek)$
            \item Stop with~$b'$
        \end{nicodemus}%
        \medskip
        
        \textbf{Oracle} $\Ochall$
        \begin{nicodemus}
            \item $(k_0,c)\getsr\FKEMenc(\ek)$
            \item $k_1\getsr\ksp$
            \item $\CC\gets\CC\cup\{c\}$
            \item Return~$(k_b,c)$
        \end{nicodemus}%
    }%
    &
    \nicodemusbox{\markerlenndR}{%
        \textbf{Oracle} $\Odec(c)$
        \begin{nicodemus}
            \item $(\dk,k)\gets\FKEMdec(\dk,c)$
            \item If $c\in\CC$: $k\gets\bot$
            \item $\CC\gets\CC\setminus\{c\}$
            \item Return $k$
        \end{nicodemus}%
        \medskip
        
        \textbf{Oracle} $\Ocorrupt$
        \begin{nicodemus}
            \item Require $\CC=\emptyset$
            \item Return~$\dk$
        \end{nicodemus}%
    }%
\end{tabular}%%
        }%
    }
    \caption{%
        Games $\INDCCA$ for forward-secure KEM~$\FKEM$.
    }
    \label{fig:fkem:ind}
\end{figure}

\paragraph{Construction}


\subsection{Updatable KEM}

\paragraph{Syntax}
An updatable KEM $\UKEM=(\UKEMgen,\UKEMenc,\UKEMdec)$ is a tuple of three algorithms with encapsulation key space~$\eksp$, decapsulation key space~$\dksp$, ciphertext space~$\csp$, and symmetric key space~$\ksp$:

\begin{itemize}
    \item $\UKEMgen: \emptyset \tor \dksp\times\eksp$
    \item $\UKEMenc: \eksp \tor \eksp\times\ksp\times\csp$
    \item $\UKEMdec: \dksp\times\csp \to \dksp\times\ksp$
\end{itemize}

\paragraph{Construction}


\subsection{Key Exchange}

\subsubsection{Two-Pass Key Exchange}

\paragraph{Syntax}
A two-pass key exchange $\TKE=(\TKEsnd,\TKErsp,\TKErcv)$ is a tuple of three algorithms with state space~$\stsp$, initial ciphertext space~$\csp_I$, response ciphertext space~$\csp_R$, and symmetric key space~$\ksp$:

\begin{itemize}
    \item $\TKEsnd: \emptyset \tor \stsp\times\csp_I$
    \item $\TKErsp: \csp_I \tor \ksp\times\csp_R$
    \item $\TKErcv: \stsp\times\csp_R \to \ksp$
\end{itemize}

\paragraph{Construction: DH Key Exchange}

\paragraph{Construction: X3DH}

\subsubsection{Authenticated Key Exchange}

\paragraph{Syntax}
An authenticated key exchange $\AKE=(\AKEgen,\allowbreak\AKEsnd,\allowbreak\AKErsp_R,\allowbreak\AKErsp_I,\allowbreak\AKErcv)$ is a tuple of five algorithms with secret key space~$\sksp$, public key space~$\pksp$, state space~$\stsp$, ciphertext space~$\csp$, and symmetric key space~$\ksp$:

\begin{itemize}
    \item $\AKEgen: \emptyset \tor \sksp\times\pksp$
    \item $\AKEsnd: \sksp\times\pksp \tor \stsp\times\csp$
    \item $\AKErsp_R: \sksp\times\csp \tor \stsp\times\csp$
    \item $\AKErsp_I: \sksp\times\stsp\times\csp \tor \ksp\times\csp$
    \item $\AKErcv: \sksp\times\stsp\times\csp \to \pksp\times\ksp$
\end{itemize}

\paragraph{Construction: Signature-Based Authentication}

\paragraph{Construction: KEM-Based Authentication}

\paragraph{Construction: TLS}

\paragraph{Construction: Noise Framework}


\subsection{Ratcheted Key Exchange (RKE)}

\subsubsection{Unidirectional RKE}

\paragraph{Syntax}
A unidirectional key exchange $\URKE=(\URKEinit,\URKEsnd,\URKErcv)$ is a tuple of three algorithms with state space~$\stsp$, ciphertext space~$\csp$, and symmetric key space~$\ksp$:

\begin{itemize}
    \item $\URKEinit: \emptyset \tor \stsp\times\stsp$
    \item $\URKEsnd: \stsp \tor \stsp\times\ksp\times\csp$
    \item $\URKErcv: \stsp\times\csp \to \ksp$
\end{itemize}

\paragraph{Construction}

\subsubsection{Sesquidirectional RKE}

\subsubsection{Bidirectional RKE}

\paragraph{Construction: Double Ratchet}

\subsubsection{Group RKE}

\paragraph{Construction: Sender Key Mechanism}

\paragraph{Construction: Tree-Based DH}

\paragraph{Construction: Tree KEM}

\bibliography{cryptobib/abbrev3,cryptobib/crypto}

\end{document}