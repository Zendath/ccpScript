\section{Game-Based Definitions}
As one can see from the minimal differences between the \emph{passive} security definition and the \emph{active} security definition for PKE from Section~\ref{sec:overview:pke}, definitions of security can be modular with respect to a pattern that is based on a small core.
We will use the example of \emph{Key Encapsulation Mechanisms}~(KEM) to introduce a methodology for defining security systematically.
This methodology is closely related to a concept called \emph{Indistinguishability up to Correctness} by Rogaway and Zhang~\cite{C:RogZha18}.

\subsection{Syntax}
The first step to formally consider a cryptographic primitive is defining its \emph{syntax}.
The syntax specifies the algorithms of a primitive by defining their inputs and outputs.

A key encapsulation mechanism $\KEM=(\KEMgen,\KEMenc,\KEMdec)$ is a tuple of three algorithms with encapsulation key space~$\eksp$, decapsulation key space~$\dksp$, ciphertext space~$\csp$, and symmetric key space~$\ksp$:
\begin{itemize}
    \item $\KEMgen: \emptyset \tor \dksp\times\eksp$
    \item $\KEMenc: \eksp \tor \ksp\times\csp$
    \item $\KEMdec: \dksp\times\csp \to \ksp$
\end{itemize}

Intuitively a KEM can be considered a special case of PKE that is limited to always encrypt (aka.\ encapsulate) fresh symmetric keys.
Instead of taking the symmetric keys as input, the encapsulation algorithm of a KEM generates them internally.

\paragraph{Construction: ElGamal KEM}
For illustration, we give the ElGamal KEM as a simple example of a KEM in Figure~\ref{fig:kem:const:elgamal}.

\begin{figure}[!ht]
    \centering
    \nicoresetlinenr%
    \fbox{%
        \scalebox{\codescalefactor}{%
            %!TEX root=../main.tex
\markersetlen{ndL}{110pt}%
\markersetlen{ndM}{120pt}%
\markersetlen{ndR}{120pt}%
\begin{tabular}[t]{lll}
    \nicodemusbox{\markerlenndL}{%
        \textbf{Proc} $\KEMgen$
        \begin{nicodemus}
            \item $\dk\getsr\ZZ_\DHp^*$
            \item $\ek\gets\DHg^\dk$
            \item Return $(\dk,\ek)$
        \end{nicodemus}%
    }%
    &
    \nicodemusbox{\markerlenndM}{%
        \textbf{Proc} $\KEMenc(\ek)$
        \begin{nicodemus}
            \item $x\getsr\ZZ_\DHp^*$
            \item $k\gets\ek^x$
            \item $c\gets\DHg^x$
            \item Return~$(k,c)$
        \end{nicodemus}%
    }%
    &
    \nicodemusbox{\markerlenndR}{%
        \textbf{Proc} $\KEMdec(\dk,c)$
        \begin{nicodemus}
            \item $k\gets c^\dk$
            \item Return $k$
        \end{nicodemus}%
    }%
\end{tabular}%%
        }%
    }
    \caption{%
        ElGamal key encapsulation mechanism~$\KEM$~\cite{ElGamal85}.
    }
    \label{fig:kem:const:elgamal}
\end{figure}

\subsection{Correctness}
Before specifying the expected security guarantees, we define the \emph{correctness} requirements.
Typically, these requirements capture the functionality that a primitive should offer when being executed in an \emph{honest} environment in which all traffic is delivered as intended.
The literature on distributed systems developed two complementary notions of correctness:
\begin{itemize}
    \item \textbf{Safety:} Nothing ``bad'' should happen (e.g., encapsulation and decapsulation for the same ciphertext should not compute different symmetric keys)
    \item \textbf{Liveness:} The ``good'' event always happens (e.g., encapsulation and decapsulation always output a non-trivial key)
\end{itemize}

A correctness definition in the sense of \emph{safety} typically suffices for the ultimate goal of defining and analyzing security.
Furthermore, specifying \emph{liveness} requirements for sophisticated primitives sometimes leads to complex, incomprehensible definitions.

A key encapsulation mechanism $\KEM$ is correct if
\[
\Pr[\KEMdec(\dk,c)=k\mid (\dk,\ek)\getsr\KEMgen,(k,c)\getsr\KEMenc(\ek))=m]=1\text{.}
\]
Equivalently, a key encapsulation mechanism $\KEM$ is correct if $\Pr[\CORR_{\KEM}(\advA)=0]=1$ for all adversaries~$\advA$, where game~$\CORR$ is defined in Figure~\ref{fig:kem:corr}.

\begin{figure}[!ht]
    \centering
    \nicoresetlinenr%
    \fbox{%
        \scalebox{\codescalefactor}{%
            %!TEX root=../main.tex
\markersetlen{ndL}{110pt}%
\markersetlen{ndR}{100pt}%
\newcommand{\CK}{\mathit{CK}}%
\begin{tabular}[t]{ll}
    \nicodemusbox{\markerlenndL}{%
        \textbf{Game} $\CORR_{\KEM}(\advA)$
        \begin{nicodemus}
            \item $\CK[\cdot]\gets\bot$
            \item $(\dk,\ek)\getsr\KEMgen$
            \item Invoke $\advA(\ek)$
            \item Stop with~$0$
        \end{nicodemus}%
        \medskip
        
        \textbf{Oracle} $\Oenc$
        \begin{nicodemus}
            \item $(k,c)\getsr\KEMenc(\ek)$
            \item $\CK[c]\gets k$
            \item Return~$c$
        \end{nicodemus}%
    }%
    &
    \nicodemusbox{\markerlenndR}{%
        \textbf{Oracle} $\Odec(c)$
        \begin{nicodemus}
            \item $k'\gets\KEMdec(\dk,c)$
            \item If $\CK[c]\notin\{k',\bot\}$:
            \item \quad Stop with $1$
            \item Return $k'$
        \end{nicodemus}%
    }%
\end{tabular}%%
        }%
    }
    \caption{%
        Game $\CORR$ for key encapsulation mechanism~$\KEM$.
    }
    \label{fig:kem:corr}
\end{figure}

\subsection{Adversarial Capabilities}
Defining \emph{security} can be divided into three components:
modeling the \emph{capabilities of adversaries},
specifying the \emph{adversaries' goal}, and
identifying attack strategies that are \emph{trivially} successful against every possible construction.
The former two components are typically influenced by intuitive purpose of the considered primitive and how this primitive is used in practice;
the latter one is almost entirely determined by all remaining definitional steps.

\subsubsection{Typical Capabilities}
We begin with considering typical adversarial capabilities against KEMs:
\begin{itemize}
    \item \textbf{Chosen-Plaintext Attacks:}
    The adversary can see the ciphertexts of encapsulated keys
    \item \textbf{Chosen-Ciphertext Attacks:}
    The adversary can see the decapsulated keys for chosen ciphertexts
    \item \textbf{Secret Key Corruption:}
    The adversary can learn the entire decapsulation key
    \item \textbf{Leakage of Information during Algorithm Execution:}
    The adversary can learn secret bits of variables processed by the encapsulation or decapsulation algorithm
    \item \textbf{Subversion of Algorithms:}
    The adversary can modify the encapsulation or decapsulation algorithm
    \item Etc.
\end{itemize}

\paragraph{Chosen-Plaintext Attacks}
The most standard adversarial capabilities that are considered realistic and defendable for KEMs are chosen-plaintext and chosen-ciphertext attacks (CPA resp.\ CCA).

Consider the adversarial capabilities shown in Figure~\ref{fig:adv_capabilities}.
Here, the adversary can see the values in the dashed boxes, as well as send values (in red).
Observing the values $\ek$ and $c$ is possible by eavesdropping, but the fact that $k'$ is visible might be surprising at first.
This is because an active adversary can send $c'$ and observe certain \emph{side-channels} to learn information about $k'$.
E.g. the adversary can measure the time it takes to decapsulate $c'$ or send an invalid $c'$ and observe the error message.
To model this capability we provide the adversary with the ability to choose ciphertexts for decapsulation and provide it with the decryption oracle (CCA).

\begin{figure}[!ht]
    \centering
    %!TEX root=../main.tex
\begin{tabular}{lcl}
    \underline{Alice} & \underline{$\advA$} & \underline{Bob}\\
    $(k,c)\getsr\KEMenc(\ek)$ & \arrbox{2cm}{$\xleftarrow{\dbox{$\ek$}}$} & $(\dk,\ek)\getsr\KEMgen$\\
    & $\xrightarrow{\dbox{$c$}}$ & $k\gets\KEMdec(\dk,c)$\\
    & {\color{red}$\xrightarrow{c'}$} & $\dbox{$k'$}\gets\KEMdec(\dk,c')$
\end{tabular}%
    \caption{%
        A sketch of common adversarial capabilities.
    }
    \label{fig:adv_capabilities}
\end{figure}

\paragraph{Chosen-Ciphertext Attacks}
The game $\INDCCA$ is defined in Figure~\ref{fig:kem:ind}.

\begin{figure}[!ht]
    \centering
    \nicoresetlinenr%
    \fbox{%
        \scalebox{\codescalefactor}{%
            %!TEX root=../main.tex
\markersetlen{ndL}{110pt}%
\markersetlen{ndR}{120pt}%
\newcommand{\CC}{\mathit{CC}}%
\begin{tabular}[t]{ll}
    \nicodemusbox{\markerlenndL}{%
        \textbf{Game} $\INDCCA_{\KEM}^b(\advA)$
        \begin{nicodemus}
            \item $\CC\gets\emptyset$
            \item $(\dk,\ek)\getsr\KEMgen$
            \item $b'\getsr\advA(\ek)$
            \item Stop with~$b'$
        \end{nicodemus}%
    }%
    &
    \nicodemusbox{\markerlenndR}{%
        \textbf{Oracle} $\Ochall$
        \begin{nicodemus}
            \item $(k_0,c)\getsr\KEMenc(\ek)$
            \item $k_1\getsr\ksp$
            \item $\CC\gets\CC\cup\{c\}$
            \item Return~$(k_b,c)$
        \end{nicodemus}%
        \medskip
        
        \textbf{Oracle} $\Odec(c)$
        \begin{nicodemus}
            \item Require $c\notin\CC$
            \item $k\gets\KEMdec(\dk,c)$
            \item Return $k$
        \end{nicodemus}%
    }%
\end{tabular}%%
        }%
    }
    \caption{%
        Games $\INDCCA$ for key encapsulation mechanism~$\KEM$.
    }
    \label{fig:kem:ind}
\end{figure}

\subsection{Adversarial Goal}
The adversarial goal is the second component of defining security.
It describes what the adversary is trying to learn. 
The adversarial goal is typically denoted in the first part of the name of the security game (e.g. \emph{IND}-CPA or \emph{OW}-CCA)

\subsubsection{Typical Goals}
For KEMs, the typical adversarial goals are either indistinguishability of ciphertexts or one-way security.

\paragraph{One-Way Security}
One-way security~(OW-security) is the intuitive definition of the adversarial goal. 
For a KEM, OW-security defines this to be learning the encapsulated key $k$.
The security game $\OWCCA$ is defined in Figure~\ref{fig:kem:ow:cca} and the $\OWCPA$ game follows from it by excluding the decryption oracle.

\begin{figure}[!ht]
    \centering
    \nicoresetlinenr%
    \fbox{%
        \scalebox{\codescalefactor}{%
            %!TEX root=../main.tex
\markersetlen{ndL}{110pt}%
\markersetlen{ndR}{120pt}%
\newcommand{\CC}{\mathit{CC}}%
\begin{tabular}[t]{ll}
    \nicodemusbox{\markerlenndL}{%
        \textbf{Game} $\OWCCA_{\KEM}(\advA)$
        \begin{nicodemus}
            \item $\CC\gets\emptyset$
            \item $K\gets\emptyset$
            \item $(\dk,\ek)\getsr\KEMgen$
            \item $k'\getsr\advA(\ek)$
            \item Stop with~$k'\in K$
        \end{nicodemus}%
    }%
    &
    \nicodemusbox{\markerlenndR}{%
        \textbf{Oracle} $\Ochall$
        \begin{nicodemus}
            \item $(k,c)\getsr\KEMenc(\ek)$
            \item $K\gets K\cup\{k\}$
            \item $\CC\gets\CC\cup\{c\}$
            \item Return~$(c)$
        \end{nicodemus}%
        \medskip
        
        \textbf{Oracle} $\Odec(c)$
        \begin{nicodemus}
            \item Require $c\notin\CC$
            \item $k\gets\KEMdec(\dk,c)$
            \item Return $k$
        \end{nicodemus}%
    }%
\end{tabular}%%
        }%
    }
    \caption{%
        One-Way CCA security game for $\KEM$.
    }
    \label{fig:kem:ow:cca}
\end{figure}

\paragraph{Indistinguishability of Ciphertexts}
As we have seen in Section~\ref{sec:overview:pke}, we sometimes want \emph{indistinguishability} of the generated ciphertext and a randomly chosen ciphertext.
The reason for this is that it gives us a stronger security guarantee. 
Namely, if a construction is IND secure, the adversary is unable to learn a single bit of the secret.
The $\INDCCA$ game is defined in Figure~\ref{fig:kem:ind}

\subsection{Trivial Winning Strategies}\label{sec:kem:trivial_attacks}
So far, we have done some unexplained bookkeeping in our security games. The reason for that are \emph{trivial winning strategies}.
These are strategies or attacks that work against every correct construction, simply because the constructions have to give sensible responses.
For our KEM there are two trivial winning conditions we need to guard against (Figure~\ref{kem:triv}).

\begin{figure}[!ht]%
    \centering
    %!TEX root=../main.tex
\parbox[t]{3.5cm}{%
    $\OWCCA$:
    \begin{enumerate}[topsep=0pt]
        \item $c\getsr\Ochall$
        \item $k\gets\Odec(c)$
        \item Stop with $k$
    \end{enumerate}    
}\parbox[t]{3.5cm}{%
    $\INDCCA$:
    \begin{enumerate}[topsep=0pt]
        \item $(k,c)\getsr\Ochall$
        \item $k'\gets\Odec(c)$
        \item Stop with $k\not= k'$
    \end{enumerate}
} 
    \caption{Trivial winning strategies against $\OWCCA$ and $\INDCCA$ $\KEM$ games.}
    \label{kem:triv}
\end{figure}

Now that the trivial winning conditions are defined, we can specify the advantage terms for each game.
For the OW-CPA and OW-CCA games that is:
\begin{align*}
    \Adv_\KEM^\owcpa(\advA)\coloneqq&\Pr[\OWCPA_{\KEM}(\advA)=1]\\
    \Adv_\KEM^\owcca(\advA)\coloneqq&\Pr[\OWCCA_{\KEM}(\advA)=1]
\end{align*}
And the advantages for the IND games are defined as:
\begin{align*}
    \Adv_\KEM^\indcpa(\advA)\coloneqq&\left|\Pr[\INDCPA_{\KEM}^0(\advA)=1]-\Pr[\INDCPA_{\KEM}^1(\advA)=1]\right|\\
    \Adv_\KEM^\indcca(\advA)\coloneqq&\left|\Pr[\INDCCA_{\KEM}^0(\advA)=1]-\Pr[\INDCCA_{\KEM}^1(\advA)=1]\right|
\end{align*}

If there exists a provably secure construction for a definition, then there are no more open trivial winning conditions.

\paragraph{Multi-Instance Security}
Definitions of security, such as the IND-CCA game, differ depending on the context.
One example of that would be the multi-instance security game (Figure~\ref{fig:kem:ind:mi:corrupt}).
This game is called \emph{multi-instance} because it captures a setting in which many instances are running concurrently.
This can be the case in TLS for example, where many connections are established at the same time.
If we were to use the IND-CCA game from Figure~\ref{fig:kem:ind}, then we would have to use the \emph{hybrid argument} to prove security.
The hybrid argument can be used to prove security for multiple instances, by showing step by step that each instance is secure and then showing the indistinguishability between each step.
Doing so however, yields a factor $q$ in the final security bound, where $q$ is the number of instances.
To avoid this degeneration of the security bound, we can use the multi-instance game, for which sometimes tighter security bounds exist.

\begin{figure}[!ht]
    \centering
    \nicoresetlinenr%
    \fbox{%
        \scalebox{\codescalefactor}{%
            %!TEX root=../main.tex
\markersetlen{ndL}{130pt}%
\markersetlen{ndR}{120pt}%
\newcommand{\CC}{\mathit{CC}}%
\newcommand{\CR}{\mathit{CR}}%
\begin{tabular}[t]{ll}
    \nicodemusbox{\markerlenndL}{%
        \textbf{Game} $\INDCCA_{\KEM}^b(\advA)$
        \begin{nicodemus}
            \item $n\gets0$
            \item $\CR\gets\emptyset$
            \item $b'\getsr\advA$
            \item Require $\forall i\in\CR:\CC_i=\emptyset$
            \item Stop with~$b'$
        \end{nicodemus}%
        \medskip
        
        \textbf{Oracle} $\Ogen$
        \begin{nicodemus}
            \item $\CC_n\gets\emptyset$
            \item $(\dk_n,\ek_n)\getsr\KEMgen$
            \item $n\gets n+1$
            \item Return $\ek_{n-1}$
        \end{nicodemus}%
    }%
    &
    \nicodemusbox{\markerlenndR}{%
        \textbf{Oracle} $\Ochall(i)$
        \begin{nicodemus}
            \item Require $i\in[n]$
            \item $(k_0,c)\getsr\KEMenc(\ek_i)$
            \item $k_1\getsr\ksp$
            \item $\CC_i\gets\CC_i\cup\{c\}$
            \item Return~$(k_b,c)$
        \end{nicodemus}%
        \medskip
        
        \textbf{Oracle} $\Odec(i,c)$
        \begin{nicodemus}
            \item Require $i\in[n]\land c\notin\CC_i$
            \item $k\gets\KEMdec(\dk_i,c)$
            \item Return $k$
        \end{nicodemus}%
        \medskip
        
        \textbf{Oracle} $\Ocorrupt(i)$
        \begin{nicodemus}
            \item Require $i\in[n]$
            \item $\CR\gets\CR\cup i$
            \item Return $\dk$
        \end{nicodemus}%
    }%
\end{tabular}%%
        }%
    }
    \caption{%
        Multi-instance games $\INDCCA$ for key encapsulation mechanism~$\KEM$.
    }
    \label{fig:kem:ind:mi:corrupt}
\end{figure}