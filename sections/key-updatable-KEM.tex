\section{Key-Updatable Key Encapsulation Mechanism (KU-KEM)}

An overview of the interaction between Alice and Bob in a key-updatable key encapsulation mechanism ($\KUKEM$) is shown in Figure~\ref{fig:kukem:overview}. 

\begin{figure}[!ht]
    \centering
    % !TeX root = ..\..\main.tex
\begin{tabular}{cC{1cm}c}
    \textbf{Alice} &  & \textbf{Bob}\\
    & $\xlongleftarrow{\pk}$ & $(\sk, \pk)\getsr\KUKEMgen$\\
    $\pk\gets\KUKEMup(\pk, \ad)$ & & $\sk\gets\KUKEMup(\sk, \ad)$\\
    $(k, c)\getsr\KUKEMenc(\pk)$ & $\xlongrightarrow{c}$ & \\
    & & $k\gets\KUKEMdec(\sk, c)$\\ 
    $\pk\gets\KUKEMup(\pk, \ad)$ & & $\sk\gets\KUKEMup(\sk, \ad)$
\end{tabular}
    \caption{Overview of a key-updatable key encapsulation mechanism (KU-KEM).}
    \label{fig:kukem:overview}
\end{figure}

\paragraph{Syntax} An updatable KEM $\KUKEM=(\KUKEMgen,\KUKEMenc,\KUKEMdec, \KUKEMup)$ is a tuple of five algorithms ($\KUKEMup$ has two definitions) with secret key space~$\sksp$, public key space~$\pksp$, ciphertext space~$\csp$, symmetric key space~$\ksp$ and associated data space~$\adsp$:

\begin{itemize}
    \item $\KUKEMgen: \emptyset \tor \sksp\times\pksp$
    \item $\KUKEMenc: \pksp \tor \ksp\times\csp$
    \item $\KUKEMdec: \sksp\times\csp \to \dksp\times\ksp$
    \item $\KUKEMup: \pksp\times\adsp\to\pksp$
    \item $\KUKEMup: \sksp\times\adsp\to\sksp$
\end{itemize}

\paragraph{Correctness} A $\KUKEM$ construction is correct, if the following holds:
\begin{align*}
    \Pr[\KUKEMdec(\sk_n,c)=k\mid&(\sk_0,\pk_0)\getsr\KUKEMgen, \ad_i\getsr\adsp, \sk_i\gets\KUKEMup(\sk_{i-1},\ad_i),\\
     &\pk_i\gets\KUKEMup(\pk_{i-1},\ad_i), (k,c)\getsr\KUKEMenc(\pk_n)] = 1
\end{align*}

\paragraph{Security} We define the adversarial capabilities by specifying the oracles:

\begin{itemize}
    \item $\Oenc()\to(c,k_b)$
    \item $\OupPK(\ad)\to\pk$
    \item $\OupSK(\ad)\to\bot$
    \item $\Odec(c)\to\sfrac{k}{\bot}$, where $\bot$ is output when $c$ is a challenge.
    \item $\Ocorrupt\to\sk$
\end{itemize}  

The adversarial goal is to learn the challenge bit $b$. This models the confidentiality of the encapsulated keys.

\paragraph{Non-trivial attacks} In contrast to previous definitions, we do not give the trivial attacks here.
Instead, there are two security properties that should hold for a key-updatable KEM:
\begin{enumerate}
    \item 
    \begin{minipage}[t]{.5\linewidth}
        Previous keys $k$ should stay secure even if later secret keys $\sk$ are corrupted after intermediate updates.
    \end{minipage}\hfill
    \begin{minipage}[t][][b]{.5\linewidth}
        \centering
        \vspace{-.7\baselineskip} %the vertical alignment is not perfect, but skipping .7\baselineskip is pretty arbitrary
        % !TeX root = ..\..\main.tex
\begin{tabular}{cC{3cm}c}
    \textbf{Alice} &  & \textbf{Bob}\\
    $\pk$ & $\xlongrightarrow{\textcolor{green}{c}}$ & $\sk$\\
    & &$\downarrow$\\
    & &$\mathrm{up}(\ad)$\\
    & &$\downarrow$\\
    & &$\sk'$\makebox[0pt][l]{\quad\color{red}\Large\Lightning} %put it in zero width box, to prevent it from changing the centering alignment
\end{tabular}
    \end{minipage}
    \item \begin{minipage}[t]{.5\linewidth}
        Keys $k$ should stay secure even if incompatible secret keys are corrupted.
        Secret keys are incompatible if they are updated with associated data that differs from the public key updates.
    \end{minipage}\hfill
    \begin{minipage}[t][][b]{.5\linewidth}
        \centering
        \vspace{-.7\baselineskip}
        % !TeX root = ..\..\main.tex
\begin{tabular}{cC{3cm}c}
    \textbf{Alice} &  & \textbf{Bob}\\
    $\pk$ & & $\sk$\\
    $\downarrow$& &$\downarrow$\\
    $\mathrm{up}(\ad)$& $\ad\neq\ad^*$ &$\mathrm{up}(\ad^*)$\\
    $\downarrow$ & &$\downarrow$\\
    $\pk'$ & $\xlongrightarrow{\textcolor{green}{c}}$ &$\sk'$\makebox[0pt][l]{\quad\color{red}\Large\Lightning} %put it in zero width box, to prevent it from changing the centering alignment
\end{tabular}
    \end{minipage}
\end{enumerate}

\subsection{Construction} 
The idea of the construction is to use $\HIBE$~(Section~\ref{sec:hibe}) where $\KUKEMup(\pk,\ad)$ keeps a list of update strings and $\KUKEMup(\sk,\ad)$ extends the path in the $\HIBE$ identity hierarchy.
An outline of this idea is given in Figure~\ref{fig:kukem:construction:idea}.

\begin{figure}[!ht]
    \centering
    % !TeX root = ..\..\main.tex
\begin{tikzpicture}

    %code for the left update/pk-renewing operations
    \path let \p1=(-2,0) in %define point p1 --> \x1 is the x variable of p1, this allows for easier adjusting of textnodes x placement
        node[anchor=west] at (\x1, 0)    {$\pk_0=(\pk,\epsilon)$}
        node[anchor=west] at (\x1, -.5)  {$\mathrm{up}(\pk_0, \ad_1)$}
        node[anchor=west] at (\x1, -1)   {$\pk_1=(\pk,\ad_1)$}
        node[anchor=west] at (\x1, -1.5) {$\mathrm{up}(\pk_1, \ad_2)$}
        node[anchor=west] (lastTextNode) at (\x1, -2)   {$\pk_2=(\pk,\ad_1\|\ad_2)$}
        node[yshift=-.25cm] at (lastTextNode.south) {\vdots}; %place three vertical dots centered below the last text node

    %code for the right figure from the lecture (the delegation tree)
    \node at (4,0) {$\sk_0$}
        child [missing] %declare missing children --> otherwise the tree would be a straight line down
        child [missing]
        child[missing]
        child {node {$\sk_1$} 
            child [missing]
            child {node {$sk_2$} edge from parent node[right=5mm, midway] {$\mathrm{up}(\sk_1,\ad_2)$} %edge from parent node places a node at the connecting edge
                                                  node[left=1mm, midway] {$\mathrm{del}(\sk_0, \ad_2$)}} %we want two nodes per edge
            child [missing]
            child [missing] edge from parent node[right=9mm, midway] {$\mathrm{up}(\sk_0,\ad_1)$}
                                             node[left=1mm, midway] {$\mathrm{del}(\sk_0, \ad_2$)}}
        child [missing];
\end{tikzpicture}
    \caption{Basic Idea of constructing a $\KUKEM$ with $\HIBE$.}
    \label{fig:kukem:construction:idea}
\end{figure}

\paragraph{Observations} 
One problem with the construction is that the public key $\pk$ grows, although compression of $\pk$ is possible for some $\HIBE$ schemes.
Another problem is that the depth of the $\HIBE$ is linear in the number of updates. 
As a result, unbounded $\HIBE$ has to be used.

Balli et al.~\cite{AC:BalRosVau20} showed that any $\URKE$ scheme secure against weak randomness is equivalent to a $\KUKEM$ scheme.

At EuroCrypt 2023, Rösler et al.~\cite{EC:RosSlaStr23} presented two new constructions, showing that if the number of updates is bounded then $\KUKEM$ can be build from \emph{identity-based encryption}~(IBE).
If the number of updates is unbounded, then bounded-depth $\HIBE$ can be used. An example tree for the later case is given in Figure~\ref{fig:kukem:upibe}. \alert{JR: I don't quite understand how to interpret the tree. There should be some explanation here.}

\begin{figure}[!ht]
    \centering
    % !TeX root = ..\..\main.tex
\begin{tikzpicture}[
    % !changing the distances requires recalculating the positions p1 in the later draw commands (more information below)
    level 1/.style = {level distance = 1.5cm, sibling distance = 6cm},
    level 2/.style = {level distance = 1cm, sibling distance = 3cm},
    level 3/.style = {level distance = 1cm, sibling distance = 1.5cm},
    level 4/.style = {level distance = 1cm, sibling distance = .75cm},
    every child node/.style = {circle, draw, scale=.5},]

    % draw the nodes
    \node (n) {$\sk$} child foreach \x in {0,1}     %root
            {node {} child foreach \y in {0,1}      %level 1
                {node {} child foreach \z in {0,1}  %level 2
                    {node {} child                  %level 3
                        %draw the leaf nodes with a squiggly arrow
                        {node {} edge from parent[decorate, decoration={coil,aspect=0, amplitude=1pt, segment length=2pt}]}}}};
                    
    %draw the blue arrows (this can be thought of as being done in multiple parts):
    % 1. Draw the path from start node to end (this is the "(\p1) -- (n-X)--...--(n-X.south)" part)
    % 1. a) Since the arrow is shifted (step 4), the first line has to be shorter than the actual edge in the graph
    %       This is done by using "partway modifiers" (https://tikz.dev/tikz-coordinates#autosec-648). We name the start point with let \p1 =($(n-1.center)!<dist>!(n-1-2.center)$) in ...
    %       <dist> has to be calculated s.t. dist=sqrt( (xshift * relLineLength)^2 + (xshift^2) ), where
    %       relLineLength is level distance / (sibling distance / 2)
    %       xshift is the amount of shift in the x-direction (here: -2mm)
    %       e.g. if the line starts below a level 2 node, relLineLength=1/.75, dist=3.34mm
    % 2. Draw the end of the arrow that "bends" around the last node (this is done by adding a relative node with "-- +(.2, -.2)")
    % 3. Draw the rounded start of the arrow by goind .05 units to the left and .075 units down (with "($(\p1)+(-.05,-.075)$)" ) and then rounding the corner with "[rounded corners=1mm]--"
    % Shift everything by xshift with "transform canvas={xshift=-2mm}"

    %draw the three "inner" arrows (arrows 2,3 and 4 from the left) after the rules describe above
    \draw[blue, transform canvas={xshift=-2mm,}, ->] let \p1 =($(n-1-1.center)!3.34mm!(n-1-1-2.center)$) in
        ($(\p1)+(-.05,-.075)$) [rounded corners=1mm]-- (\p1) -- (n-1-1-2.center) -- (n-1-1-2-1.south) -- node[pos=1, below] {$2t$} +(.2, -.2);
    \draw[blue, transform canvas={xshift=-2mm}, ->] let \p1 =($(n-1.center)!2.41mm!(n-1-2.center)$) in
        ($(\p1)+(-.05,-.075)$) [rounded corners=1mm]-- (\p1) -- (n-1-2.center) -- (n-1-2-1.center) -- ( n-1-2-1-1.south) -- node[pos=1, below] {$3t$} +(.2, -.2);
    \draw[blue, transform canvas={xshift=-2mm}, ->] let \p1 =($(n-1-2.center)!3.34mm!(n-1-2-2.center)$) in
        ($(\p1)+(-.05,-.075)$) [rounded corners=1mm]-- (\p1) -- (n-1-2-2.center) -- ( n-1-2-2-1.south) -- node[pos=1, below] {$4t$} +(.2, -.2);
    
    %leftmost arrow; no bend at start
    %this is drawn in two parts: The first part is from the root to to beginning of the squiggly arrow, where a node "1" is placed
    %and the second part is from node 1 to the end of the squiggly arrow, where a node "t" is placed
    %when drawing node 1 we have to reverse the yshift of 1mm by placing a new node with "+(0, -.1)"
    \draw[blue, transform canvas={xshift=-2mm, yshift=1mm}, rounded corners=1mm, -|] (n.center) -- (n-1.center) -- (n-1-1.center) -- (n-1-1-1.center) -- node[pos=1, left] {$1$} +(0, -.1);
    \draw[blue, transform canvas={xshift=-2mm, yshift=1mm}, rounded corners=1mm, ->] ($(n-1-1-1.center)+(0,-.1)$) -- ([yshift=-2mm] n-1-1-1-1.south) -- node[pos=1, below] {$t$} +(.2, -.1);

    %draw the rightmost arrow
    %we place two nodes after the last segment: one node contains vertical dots and the other node contains the label $4t+x$
    \draw[blue, transform canvas={xshift=-2mm}] let \p1 =($(n.center)!4mm!(n-2.center)$) in
        ($(\p1)+(-.05,-.075)$) [rounded corners=1mm]-- (\p1) -- (n-2.center) -- (n-2-1.center) -- ( n-2-1-1.south) -- node[below] {\vdots} +(0, -.2) node[left] {$4t+x$};

    %draw the (l,t)-line to the right 
    %let p1 be the rightmost, lowermost node of the last level before the squiggly arrow shifted 1 unit to the right
    %let p2 be the rightmost node of the last level
    \draw[|-] let \p1=([xshift=1cm] n-2-2-2) in
        (\x1, 0) -- node[right, midway] {$l$} (\p1); %\x1 is x-coordinate of p1
    \draw[|-|] let \p1=($(n-2-2-2)+(1,0)$), \p2=(n-2-2-2-1) in
        (\p1) -- node[right, midway] {$t$} (\x1, \y2);

    \end{tikzpicture}

    \caption{Using bounded-depth HIBE to construct $\KUKEM$.}
    \label{fig:kukem:upibe}
\end{figure}